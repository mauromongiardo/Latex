\documentclass[]{article}
% Created by Mauro Mongiardo 28-12-16
% here all the necessary packages are included
\usepackage{psfrag}     % for using psfrag
\usepackage{tocbibind}  % to have the ref in the toc
\usepackage{cite}       % for sorting the citations

%\usepackage[latin1]{inputenc} 
\usepackage{tikz} 
\usepackage{circuitikz}
\usepackage{pgfplots}
\usepgfplotslibrary{smithchart}
\pgfplotsset{compat=1.13}

\usepackage{chapterbib}
%\usepackage[sectionbib]{chapterbib}
%\usepackage{amsmath}
\usepackage{amssymb,amsmath, esint, commath,amsbsy}
\usepackage{mathrsfs}
%\usepackage{bm}
\usepackage{graphicx,graphics}
\usepackage{a4}
\usepackage{appendix}
\usepackage{makeidx}

\usepackage{cite}
\usepackage{mathptmx}       % selects Times Roman as basic font
\usepackage{helvet}         % selects Helvetica as sans-serif font
\usepackage{courier}        % selects Courier as typewriter font
\usepackage{type1cm}        % activate if the above 3 fonts are
                            % not available on your system
%
\usepackage{multicol}        % used for the two-column index
\usepackage[bottom]{footmisc}% places footnotes at page bottom
\usepackage[ruled,linesnumbered,longend]{algorithm2e} %when including algorithms 

%\usepackage[authoryear]{natbib}

\usepackage{siunitx}
%\usepackage{siunitx}
%\usepackage[amssymb]{SIunits}

\usepackage{comment}
%If you want to turn comments on or off (so they appear or don�t appear in your final document) you have to place:
% \includecomment{comment}, or \excludecomment{comment}




% see the list of further useful packages
% in the Reference Guide

%\usepackage[T1]{fontenc}
%\usepackage[utf8]{luainputenc}
\usepackage{color}
\usepackage{listings}

\usepackage{setspace}
\usepackage{physics}

\usepackage{latexsym}
%\usepackage{url}
\usepackage{hyperref}

\usepackage{tikz-3dplot} %requires 3dplot.sty to be in same directory, or in your LaTeX installation
\usepackage{blindtext}
\usepackage[inline]{enumitem}
\usepackage{xcolor}

\usepackage{mathtools}

\usepackage{slashed} % for the Feynman slash notation

%Reference https://www.physicsforums.com/threads/latex-how-to-do-dirac-slash-notation.55561/

\DeclareMathOperator{\dalembert}{\Box}


% edited by Mauro 28-12-16
%
%% <local definitions>
\newcommand{\R}{\mathbb{R}}	
\newcommand{\C}{\mathbb{C}}
\newcommand{\HQ}{\mathbb{H}}
\newcommand{\N}{\mathbb{N}}
\newcommand{\be}{\begin{equation}}
\newcommand{\ee}{\end{equation}}	
\newcommand{\bea}{\begin{eqnarray}}
\newcommand{\eea}{\end{eqnarray}}	
\newcommand{\Pin}{\mathrm{Pin}}	
\newcommand{\Spin}{\mathrm{Spin}}
\renewcommand{\O}{\mathrm{O}}
\newcommand{\SO}{\mathrm{SO}}
\renewcommand{\eqref}[1]{(\ref{#1})}
\newcommand{\cl}[1]{\ensuremath{Cl(#1)}} % #1 stands for the values p,q. $\cl{p,q}$ produces 'Cl(p,q)'.
\newcommand{\gvec}[1]{\ensuremath{\mbox{\textbf{\textit{#1}}}}}
\newcommand{\vect}[1]{\ensuremath{\mbox{\textbf{\textit{#1}}}}}
%% </local definitions>

\newcommand{\Ba}[0]{\mathbf{a}}
\newcommand{\Bb}[0]{\mathbf{b}}
\newcommand{\Bc}[0]{\mathbf{c}}
\newcommand{\Bd}[0]{\mathbf{d}}
\newcommand{\Be}[0]{\mathbf{e}}
\newcommand{\Bf}[0]{\mathbf{f}}
\newcommand{\Bg}[0]{\mathbf{g}}
\newcommand{\Bh}[0]{\mathbf{h}}
\newcommand{\Bi}[0]{\mathbf{i}}
\newcommand{\Bj}[0]{\mathbf{j}}
\newcommand{\Bk}[0]{\mathbf{k}}
\newcommand{\Bl}[0]{\mathbf{l}}
\newcommand{\Bm}[0]{\mathbf{m}}
\newcommand{\Bn}[0]{\mathbf{n}}
\newcommand{\Bo}[0]{\mathbf{o}}
\newcommand{\Bp}[0]{\mathbf{p}}
\newcommand{\Bq}[0]{\mathbf{q}}
\newcommand{\Br}[0]{\mathbf{r}}
\newcommand{\Bs}[0]{\mathbf{s}}
\newcommand{\Bt}[0]{\mathbf{t}}
\newcommand{\Bu}[0]{\mathbf{u}}
\newcommand{\Bv}[0]{\mathbf{v}}
\newcommand{\Bw}[0]{\mathbf{w}}
\newcommand{\Bx}[0]{\mathbf{x}}
\newcommand{\By}[0]{\mathbf{y}}
\newcommand{\Bz}[0]{\mathbf{z}}
\newcommand{\BA}[0]{\mathbf{A}}
\newcommand{\BB}[0]{\mathbf{B}}
\newcommand{\BC}[0]{\mathbf{C}}
\newcommand{\BD}[0]{\mathbf{D}}
\newcommand{\BE}[0]{\mathbf{E}}
\newcommand{\BF}[0]{\mathbf{F}}
\newcommand{\BG}[0]{\mathbf{G}}
\newcommand{\BH}[0]{\mathbf{H}}
\newcommand{\BI}[0]{\mathbf{I}}
\newcommand{\BJ}[0]{\mathbf{J}}
\newcommand{\BK}[0]{\mathbf{K}}
\newcommand{\BL}[0]{\mathbf{L}}
\newcommand{\BM}[0]{\mathbf{M}}
\newcommand{\BN}[0]{\mathbf{N}}
\newcommand{\BO}[0]{\mathbf{O}}
\newcommand{\BP}[0]{\mathbf{P}}
\newcommand{\BQ}[0]{\mathbf{Q}}
\newcommand{\BR}[0]{\mathbf{R}}
\newcommand{\BS}[0]{\mathbf{S}}
\newcommand{\BT}[0]{\mathbf{T}}
\newcommand{\BU}[0]{\mathbf{U}}
\newcommand{\BV}[0]{\mathbf{V}}
\newcommand{\BW}[0]{\mathbf{W}}
\newcommand{\BX}[0]{\mathbf{X}}
\newcommand{\BY}[0]{\mathbf{Y}}
\newcommand{\BZ}[0]{\mathbf{Z}}

\newcommand{\ta}[0]{\tilde{a}}
\newcommand{\tb}[0]{\tilde{b}}
\newcommand{\tc}[0]{\tilde{c}}
\newcommand{\td}[0]{\tilde{d}}

\newcommand{\hA}[0]{\hat{A}}
\newcommand{\hB}[0]{\hat{B}}
\newcommand{\hH}[0]{\hat{H}}

\newcommand{\tA}[0]{\tilde{A}}
\newcommand{\tB}[0]{\tilde{B}}
\newcommand{\tF}[0]{\tilde{F}}
\newcommand{\tE}[0]{\tilde{E}}
\newcommand{\tH}[0]{\tilde{H}}

% spinors definition
\newcommand{\barJ}[0]{\bar{J}}
\newcommand{\barF}[0]{\bar{F}}
\newcommand{\barP}[0]{\bar{P}}
\newcommand{\barW}[0]{\bar{W}}



\newcommand{\tnabla}[0]{\tilde{\nabla}}
\newcommand{\tphi}[0]{\tilde{\phi}}
\newcommand{\tpsi}[0]{\tilde{\psi}}

%
\newcommand{\wavep}[0]{\partial^+}
\newcommand{\wavem}[0]{\partial^-}

\newcommand{\wavepp}[0]{\tilde{\partial}^+}
\newcommand{\wavemp}[0]{\tilde{\partial}^-}

\newcommand{\wavepd}[0]{\bar{\partial}^+}
\newcommand{\wavemd}[0]{\bar{\partial}^-}

\newcommand{\pbd}[0]{\bar{\partial}_d}

% frequency

\newcommand{\helmp}[0]{{\underline{\partial}}^+}
\newcommand{\helmm}[0]{{\underline{\partial}}^-}

\newcommand{\helmpp}[0]{{\underline{\tilde{\partial}}}^+}
\newcommand{\helmmp}[0]{{\underline{\tilde{\partial}}}^-}

\newcommand{\helmpd}[0]{{\underline{\bar{\partial}}}^+}
\newcommand{\helmmd}[0]{{\underline{\bar{\partial}}}^-}

\newcommand{\pbfd}[0]{{\underline{\bar{\partial}}}_d}




\def \figname {Figure}
\def \emode {E }
\def \hmode {H }
\def \temode {TE }
\def \tmmode {TM }
\def \temoden {TE${}_n$ }
\def \tmmoden {TM${}_n$ }
\def \temodemn {TE${}_{mn}$ }
\def \tmmodemn {TM${}_{mn}$ }


\begin{document}

\title{wxMaxima Files}
\author{Mauro Mongiardo}
\date{Today}
\maketitle


The files which start with 1 are for rectangular coordinates, 2 corresponds to cylindrical coordinates, 3 corresponds to spherical coordinates.


\section{Rectangular coordinates}


The most complete file is:
\begin{verbatim}
File: 1Pauli_nabla3rect.wxm
\end{verbatim}
Nablarect is the block that performs the nabla operator in rectangular coordinates.
Gradesep performs the grade extraction. 
It is suggested to start from here.

The file
\begin{verbatim}
File: 1Example3.10rect.wxm
\end{verbatim}
perform the gradient evaluation of a given function.

The file
\begin{verbatim}
File: 1Example3.11rect.wxm
\end{verbatim}
perform the div and curl evaluation of a given vector.
Conventional analysis is also used. 

The file
\begin{verbatim}
File: 1Gradient_rectangular.wxm
\end{verbatim}
perform the gradient evaluation of a givenfunction with Pauli matrices.
 
 
The file
\begin{verbatim}
File: 1nabla_rectangular.wxm
\end{verbatim}
perform the div and curl evaluation of a given vector with Pauli matrices.
 



\section{Cylindrical coordinates}

\subsection{Cylindrical Potential}
The most complete file is the following:
\begin{verbatim}
File: Nabla_cyl_4_blocks.wxm
\end{verbatim}

This file contains a block performing the nabla operator in cylindrical coordinates.
It also contains the Pauli matrices for rectangular and cylindrical coordinates.
It also contains a block Gradecyl which perform the grade operation in cylindrical coordinates.

At the end there are two examples. 
The first example deals with a charged line (Electrostatic field)
The second example deals with an infinite filament with current.

\subsection{Conventional and GA nabla for cylindrical coordinates}
\begin{verbatim}
File: 2Nabla_cyl.wxm
\end{verbatim}

This files shows how to the div , grad curl operation in conventional analysis and with Pauli matrices.



\subsection{example 3.10, exercise 3.10, exercise 3.13}
\begin{verbatim}
File: Esempio_3.10.wxm
\end{verbatim}

This file contains a block performing the nabla operator in cylindrical coordinates.
It also contains the Pauli matrices for rectangular and cylindrical coordinates.

It solves the example 3.10 of Ulaby which for a scalar function
\be
V = {V}_{0}\,{e}^{-2\,\rho}\,\mathrm{sin}\left( 3\,\phi\right) 
\ee
ask to evaluate the gradient.
Then performs the divergence and the curl in cylindrical coordinates.


\section{Spherical coordinates: Dipole}

The most complete file is 
\begin{verbatim}
File: 3Nabla_sph_2_blocks.wxm
\end{verbatim}
which contains a block for the nabla evaluation in spherical coordinates and a block for the grade retrieval.


A first example of application is given in 
\begin{verbatim}
File: 3Dipole_sph_static1.wxm
\end{verbatim}



The file
\begin{verbatim}
File: 3Nabla_sph_exc_2.wxm
\end{verbatim}
is an example for a particular vector.



\begin{verbatim}
File: Dipole_sph_1.wxm
\end{verbatim}

This file contains a block performing the nabla operator in spherical coordinates.
It also contains the Pauli matrices for rectangular and spherical coordinates.

We assume a potential of the type
\be
\frac{{e}^{-j\,k\,r}}{r} \Bz_0
\ee
which is the same of a short dipole.
Perform the nabla operator on this potential.

This file also contains a block Gradesph that performs the grade extraction.

The Electric field is evaluated from the potential as
\bea
\Phi & = & - \frac{v}{jk} \nabla \cdot \BA\nonumber \\
\BE & = & -j \, k \,  v \,  \BA - \nabla\Phi \nonumber
\eea


\end{document}