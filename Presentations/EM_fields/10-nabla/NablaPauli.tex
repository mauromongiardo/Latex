%\documentclass[10pt]{beamer}
\documentclass[handout,10pt]{beamer}

\usetheme[progressbar=frametitle]{metropolis}

\usepackage{booktabs}
\usepackage[scale=2]{ccicons}
\usepackage{comment}


\usepackage{amsmath}
\usepackage{pgfplots}
\usepgfplotslibrary{dateplot}

\usepackage{xspace}
\newcommand{\themename}{\textbf{\textsc{metropolis}}\xspace}

%\usepackage{placeins} %%%
\usepackage{subfig}
\usepackage{physics}
\usepackage{amssymb}

\usepackage{esint}

\usepackage{tikz}
\usepackage{circuitikz}
\usepackage{siunitx}
\usepackage{tikz-3dplot} %requires 3dplot.sty to be in same directory, or in your LaTeX installation


\usepackage{latexsym}
\usepackage{mathtools}
\usepackage{slashed} % for the Feynman slash notation

\usepackage{listings}

\usepackage{balance}
\usepackage{verbatim}


% edited by Mauro 28-12-16
%
%% <local definitions>
\newcommand{\R}{\mathbb{R}}	
\newcommand{\C}{\mathbb{C}}
\newcommand{\HQ}{\mathbb{H}}
\newcommand{\N}{\mathbb{N}}
\newcommand{\be}{\begin{equation}}
\newcommand{\ee}{\end{equation}}	
\newcommand{\bea}{\begin{eqnarray}}
\newcommand{\eea}{\end{eqnarray}}	
\newcommand{\Pin}{\mathrm{Pin}}	
\newcommand{\Spin}{\mathrm{Spin}}
\renewcommand{\O}{\mathrm{O}}
\newcommand{\SO}{\mathrm{SO}}
\renewcommand{\eqref}[1]{(\ref{#1})}
\newcommand{\cl}[1]{\ensuremath{Cl(#1)}} % #1 stands for the values p,q. $\cl{p,q}$ produces 'Cl(p,q)'.
\newcommand{\gvec}[1]{\ensuremath{\mbox{\textbf{\textit{#1}}}}}
\newcommand{\vect}[1]{\ensuremath{\mbox{\textbf{\textit{#1}}}}}
%% </local definitions>

\newcommand{\Ba}[0]{\mathbf{a}}
\newcommand{\Bb}[0]{\mathbf{b}}
\newcommand{\Bc}[0]{\mathbf{c}}
\newcommand{\Bd}[0]{\mathbf{d}}
\newcommand{\Be}[0]{\mathbf{e}}
\newcommand{\Bf}[0]{\mathbf{f}}
\newcommand{\Bg}[0]{\mathbf{g}}
\newcommand{\Bh}[0]{\mathbf{h}}
\newcommand{\Bi}[0]{\mathbf{i}}
\newcommand{\Bj}[0]{\mathbf{j}}
\newcommand{\Bk}[0]{\mathbf{k}}
\newcommand{\Bl}[0]{\mathbf{l}}
\newcommand{\Bm}[0]{\mathbf{m}}
\newcommand{\Bn}[0]{\mathbf{n}}
\newcommand{\Bo}[0]{\mathbf{o}}
\newcommand{\Bp}[0]{\mathbf{p}}
\newcommand{\Bq}[0]{\mathbf{q}}
\newcommand{\Br}[0]{\mathbf{r}}
\newcommand{\Bs}[0]{\mathbf{s}}
\newcommand{\Bt}[0]{\mathbf{t}}
\newcommand{\Bu}[0]{\mathbf{u}}
\newcommand{\Bv}[0]{\mathbf{v}}
\newcommand{\Bw}[0]{\mathbf{w}}
\newcommand{\Bx}[0]{\mathbf{x}}
\newcommand{\By}[0]{\mathbf{y}}
\newcommand{\Bz}[0]{\mathbf{z}}
\newcommand{\BA}[0]{\mathbf{A}}
\newcommand{\BB}[0]{\mathbf{B}}
\newcommand{\BC}[0]{\mathbf{C}}
\newcommand{\BD}[0]{\mathbf{D}}
\newcommand{\BE}[0]{\mathbf{E}}
\newcommand{\BF}[0]{\mathbf{F}}
\newcommand{\BG}[0]{\mathbf{G}}
\newcommand{\BH}[0]{\mathbf{H}}
\newcommand{\BI}[0]{\mathbf{I}}
\newcommand{\BJ}[0]{\mathbf{J}}
\newcommand{\BK}[0]{\mathbf{K}}
\newcommand{\BL}[0]{\mathbf{L}}
\newcommand{\BM}[0]{\mathbf{M}}
\newcommand{\BN}[0]{\mathbf{N}}
\newcommand{\BO}[0]{\mathbf{O}}
\newcommand{\BP}[0]{\mathbf{P}}
\newcommand{\BQ}[0]{\mathbf{Q}}
\newcommand{\BR}[0]{\mathbf{R}}
\newcommand{\BS}[0]{\mathbf{S}}
\newcommand{\BT}[0]{\mathbf{T}}
\newcommand{\BU}[0]{\mathbf{U}}
\newcommand{\BV}[0]{\mathbf{V}}
\newcommand{\BW}[0]{\mathbf{W}}
\newcommand{\BX}[0]{\mathbf{X}}
\newcommand{\BY}[0]{\mathbf{Y}}
\newcommand{\BZ}[0]{\mathbf{Z}}

\newcommand{\ta}[0]{\tilde{a}}
\newcommand{\tb}[0]{\tilde{b}}
\newcommand{\tc}[0]{\tilde{c}}
\newcommand{\td}[0]{\tilde{d}}

\newcommand{\hA}[0]{\hat{A}}
\newcommand{\hB}[0]{\hat{B}}
\newcommand{\hH}[0]{\hat{H}}

\newcommand{\tA}[0]{\tilde{A}}
\newcommand{\tB}[0]{\tilde{B}}
\newcommand{\tF}[0]{\tilde{F}}
\newcommand{\tE}[0]{\tilde{E}}
\newcommand{\tH}[0]{\tilde{H}}

% spinors definition
\newcommand{\barJ}[0]{\bar{J}}
\newcommand{\barF}[0]{\bar{F}}
\newcommand{\barP}[0]{\bar{P}}
\newcommand{\barW}[0]{\bar{W}}



\newcommand{\tnabla}[0]{\tilde{\nabla}}
\newcommand{\tphi}[0]{\tilde{\phi}}
\newcommand{\tpsi}[0]{\tilde{\psi}}

%
\newcommand{\wavep}[0]{\partial^+}
\newcommand{\wavem}[0]{\partial^-}

\newcommand{\wavepp}[0]{\tilde{\partial}^+}
\newcommand{\wavemp}[0]{\tilde{\partial}^-}

\newcommand{\wavepd}[0]{\bar{\partial}^+}
\newcommand{\wavemd}[0]{\bar{\partial}^-}

\newcommand{\pbd}[0]{\bar{\partial}_d}

% frequency

\newcommand{\helmp}[0]{{\underline{\partial}}^+}
\newcommand{\helmm}[0]{{\underline{\partial}}^-}

\newcommand{\helmpp}[0]{{\underline{\tilde{\partial}}}^+}
\newcommand{\helmmp}[0]{{\underline{\tilde{\partial}}}^-}

\newcommand{\helmpd}[0]{{\underline{\bar{\partial}}}^+}
\newcommand{\helmmd}[0]{{\underline{\bar{\partial}}}^-}

\newcommand{\pbfd}[0]{{\underline{\bar{\partial}}}_d}




\def \figname {Figure}
\def \emode {E }
\def \hmode {H }
\def \temode {TE }
\def \tmmode {TM }
\def \temoden {TE${}_n$ }
\def \tmmoden {TM${}_n$ }
\def \temodemn {TE${}_{mn}$ }
\def \tmmodemn {TM${}_{mn}$ }



\newcommand{\iGA}{{i}}
\newcommand{\conjg}[1] {\ensuremath{#1}^*}

\setbeamertemplate{bibliography item}{[\theenumiv]}


\title{Nabla operator with Pauli matrices}

\date{}

%\subtitle{Maximizing efficiency and power at a fixed frequency}
%\date{\today}
%\author{Alessandra Costanzo, Franco Mastri, Mauro Mongiardo*, Giuseppina Monti}
%\institute{*Department of Engineering,
%University of Perugia, Italy}

\author{ Mauro Mongiardo$^1$}

\institute{ $^1$ Department of Engineering, University of Perugia, Perugia, Italy.
}

%
\titlegraphic{\hfill\includegraphics[height=1.5cm]{logo}}


 \includecomment{comment} 
% \excludecomment{comment}

\begin{document}

\maketitle

\begin{frame}{Table of contents}
  \setbeamertemplate{section in toc}[sections numbered]
  \tableofcontents[hideallsubsections]
\end{frame}


%
\section{Vectors}
%
\begin{frame}[shrink=00]{Vectors}
Vector are generally represented in the various coordinate system as
\bea
\BA & = &  A_x \Bx_0 + A_y \By_0 + A_z \Bz_0 \nonumber \\
& = &  A_\rho \boldsymbol{\rho}_0 + A_\phi \boldsymbol{\phi}_0 + A_z \Bz_0  \nonumber \\
& = &  A_r \boldsymbol{r}_0 + A_\theta \boldsymbol{\theta}_0 + A_\phi \boldsymbol{\phi}_0 
\eea

Let us now represent them with the following form
\be
\BA = A_1 \Be_1 +A_2 \Be_2 +A_3 \Be_3
\ee
and let us impose that for $i=1,2,3$
\be
\left(\Be_i\right)^2 = 1
\label{Clii}
\ee
and that for $i \ne j$
\be
\Be_i \Be_j  = - \Be_j \Be_i 
\label{Clij}
\ee
These two conditions are sufficient to establish a Clifford algebra $Cl(3)$.

\end{frame}


%
\begin{frame}[shrink=00]{Disadvantages of the vector representation with versors}
\begin{itemize}
\item Vectors cannot be multiplied
\item it is not possible to find the inverse 
\item given two vectors $\Ba$ and $\Bb$ it is not possible to find the transformation which leads from $\Ba$ to $\Bb$ 
\item the cross product is misleading; instead, a bivector which is an oriented surface should be used.
\item no volume elements are present.
\item As we have seen the representation of a vector with Pauli matrices is the same in all coordinate systems, while with versors we only have the components.
\end{itemize}

\end{frame}
%
%

%
\begin{frame}[shrink=00]{Relation with Pauli matrices}
By using the  correspondence in Table \ref{sigmai_table2} the conditions (\ref{Clii}) and (\ref{Clij}) are realized.
\begin{table}[]
\centering
\caption{Basis vector in rectangular, cylindrical and spherical coordinate systems. The $\Be_i$ are a Clifford basis and correspond to the appropriate Pauli matrices.}
%\hspace*{2.0cm}
\label{sigmai_table2}
%\begin{tabular}{l c l c  l  l  l }
\begin{tabular}{| c | c | c | c | }
\hline
 & $\Be_1$ & $\Be_2$  & $\Be_3$  \\
\hline
rectangular & $\sigma_1$ & $\sigma_2$  &  $\sigma_3$ \\ \hline
cylindrical & $\sigma_\rho$ & $\sigma_\phi$  &  $\sigma_3$    \\ \hline
spherical & $\sigma_r$ & $\sigma_\theta$  &  $\sigma_\phi$   \\ \hline
\end{tabular}
\end{table}

\begin{table}[]
\centering
\caption{Vector $\BA$ equivalence for rectangular, cylindrical and spherical coordinate systems. }
%\hspace*{2.0cm}
\label{Avect_table}
%\begin{tabular}{l c l c  l  l  l }
\begin{tabular}{| c | c | c | c | }
\hline
 & $A_1$ & $A_2$  & $A_3$  \\
\hline
rectangular & $A_x$ & $A_y$  & $A_z$ \\ \hline
cylindrical &$A_\rho$ & $A_\phi$  & $A_z$    \\ \hline
spherical & $A_r$ & $A_\theta$  & $A_\phi$   \\ \hline
\end{tabular}
\end{table}
\end{frame}
%

\begin{frame}[shrink=00]{Pauli matrices in Rectangular  coordinates}

\bea
\sigma_1 & = &  \begin{pmatrix}0 & 1\cr 1 & 0\end{pmatrix} \nonumber \\
\sigma_2& = & \begin{pmatrix}0 & -i \cr i & 0\end{pmatrix} \nonumber \\
\sigma_3 & = & \begin{pmatrix}1 & 0\cr 0 & -1\end{pmatrix} \, .
\label{sigma_cyl}
\eea

\end{frame}
\begin{frame}[shrink=00]{Pauli matrices in Cylindrical coordinates}

\bea
\sigma_\rho & = &  \begin{pmatrix}0 & {e}^{-i\,\phi}\cr {e}^{i\,\phi} & 0\end{pmatrix} \nonumber \\
\sigma_\phi & = & \begin{pmatrix}0 & -i\,{e}^{-i\,\phi}\cr i\,{e}^{i\,\phi} & 0\end{pmatrix} \nonumber \\
\sigma_3 & = & \begin{pmatrix}1 & 0\cr 0 & -1\end{pmatrix} \, .
\label{sigma_cyl}
\eea

\alert{Attention! The Pauli matrices in Cylindrical coordinates depend on $\phi$!}
\end{frame}

\begin{frame}[shrink=0]{Pauli matrices in Spherical coordinates}

We have the following three matrices in spherical coordinates
\bea
\sigma_r & = &  \begin{pmatrix}\cos \theta  & {e}^{-i\,\phi}\, \sin \theta \cr {e}^{i\,\phi}  \, \sin \theta & -\cos \theta \end{pmatrix} \nonumber \\
\sigma_\theta & = &  \begin{pmatrix}-\sin\theta & {e}^{-i\,\phi}\, \cos \theta \cr {e}^{i\,\phi}  \, \cos \theta& \sin \theta \end{pmatrix} \nonumber \\
\sigma_\phi & = & \begin{pmatrix}0 & -i\,{e}^{-i\,\phi}\cr i\,{e}^{i\,\phi} & 0\end{pmatrix} \, .
\label{sigma_sph}
\eea

\alert{Attention! The Pauli matrices in spherical coordinates depend on $\theta, \phi$!}
\end{frame}

\begin{frame}[shrink=30]{Summary of Pauli matrices}
% Table summarizing the basis with Pauli matrices
\begin{table}[]
\centering
\caption{Basis vector in rectangular, cylindrical and spherical coordinate systems. 
%
The $\Be_i$ are a Clifford basis and correspond to the appropriate Pauli matrices.}
%\hspace*{2.0cm}
\label{sigmai_table2}
%\begin{tabular}{l c l c  l  l  l }
\begin{tabular}{| c | c | c | c | }
\hline
 & $\Be_1$ & $\Be_2$  & $\Be_3$  \\
\hline
 & & & \\
rectangular 
& $\sigma_1 = \begin{pmatrix}0 & 1\cr 1 & 0\end{pmatrix}$ 
& $\sigma_2 = \begin{pmatrix}0 & -i\cr i & 0\end{pmatrix}$  
& $\sigma_3 = \begin{pmatrix}1 & 0\cr 0 & -1\end{pmatrix} $ \\ 
 & & & \\
 \hline
 & & & \\
cylindrical 
&  $\sigma_\rho =  \begin{pmatrix}0 & {e}^{-i\,\phi}\cr {e}^{i\,\phi} & 0\end{pmatrix}$ 
&  $\sigma_\phi = \begin{pmatrix}0 & -i\,{e}^{-i\,\phi}\cr i\,{e}^{i\,\phi} & 0\end{pmatrix} $  
&  $\sigma_3 = \begin{pmatrix}1 & 0\cr 0 & -1\end{pmatrix}$    \\
 & & & \\
 \hline
 & & & \\
spherical 
&  $\sigma_r =  \begin{pmatrix}\cos \theta  & {e}^{-i\,\phi}\, \sin \theta \cr {e}^{i\,\phi}  \, \sin \theta & -\cos \theta \end{pmatrix}$ 
&  $\sigma_\theta = \begin{pmatrix}-\sin\theta & {e}^{-i\,\phi}\, \cos \theta \cr {e}^{i\,\phi}  \, \cos \theta& \sin \theta \end{pmatrix}$  
&  $\sigma_\phi  =  \begin{pmatrix}0 & -i\,{e}^{-i\,\phi}\cr i\,{e}^{i\,\phi} & 0\end{pmatrix}$  \\ 
 & & & \\
 \hline
 \end{tabular}
\end{table}
\end{frame}


\section{Gradient}
\begin{frame}[shrink=00]{Gradient}
Let us consider a scalar  function $\phi(\Br)$, which is single valued and continuous in a volume $V$. 
Physically, the function $\phi$ may represent an electric potential, a temperature, etc. 
\pause

At a point $P$ the function will take the value $\phi_P$. 
Now suppose to draw a sphere of radius $\Delta s$ centered in $P$. 
We can check the values of $\phi(\Br)$ on this sphere and, in general, there will be a point $Q$ on which the variation $\Delta \phi = \phi_P - \phi_Q$ is maximum. 
\pause

This defines also the direction from $P$ to $Q$. This direction is taken as the direction of a new vector which is called the \alert{gradient}. The magnitude of the gradient is defined as the value $\Delta \phi/\Delta s$ in this preferred direction. 

\pause
Thus the gradient of a scalar may be defined as 
\be \label{gradient}
grad \phi = \nabla \phi = \Ba_{max} \lim_{\Delta s \to 0} \left( \frac{\Delta \phi}{\Delta s}\right)_{max}
\ee
%
where $\Ba_{max}$ is a unit vector pointing in the direction of maximum $\frac{\Delta \phi}{\Delta s}$.

%It is noted that the definition (\ref{gradient}) is not dependent on a a particular coordinate system.


\end{frame}

\begin{frame}[shrink=00]{Gradient expression}
It is noted that the definition (\ref{gradient}) is not dependent on a a particular coordinate system.
\pause


In \alert{rectangular coordinates}, (\ref{gradient}) reduces to
%
\be \label{gradientrect}
 \nabla \phi = \Ba_{x}  \frac{\partial \phi}{\partial x} + \Ba_{y}  \frac{\partial \phi}{\partial y} + \Ba_{z}  \frac{\partial \phi}{\partial z} \, .
\ee
%
\pause


In \alert{circular--cylinder coordinates} , $\Delta s$ in the angular direction $\psi$ is not equal to $\Delta \psi$ but $\delta s = \rho \Delta \psi$. Thus we have
%
\be \label{gradientrect}
 \nabla \phi = \Ba_{\rho}  \frac{\partial \phi}{\partial \rho} + \frac{\Ba_{\psi}}{\rho}  \frac{\partial \phi}{\partial \psi} + \Ba_{z}  \frac{\partial \phi}{\partial z} \, .
\ee
%
\pause


Similarly, in \alert{spherical coordinates}, 
%
\be \label{gradientrect}
 \nabla \phi = \Ba_{r}  \frac{\partial \phi}{\partial r} + \frac{\Ba_{\psi}}{r \sin \theta}  \frac{\partial \phi}{\partial \psi} + \frac{\Ba_{\theta}}{r}  \frac{\partial \phi}{\partial \theta} \, .
\ee
%

\end{frame}

\section{Nabla definition}
\begin{frame}[shrink=00]{Nabla definition}

The \alert{nabla operator is defined} as
\be
\nabla = \sum_{i=1
%,2,3
}^3  c_i  \, \Be_i  \, \partial_i
\ee
%
where we have introduced appropriate \alert{scaling coefficients $c_i$}, a \alert{Clifford vector base composed by $\Be_i$}, and the \alert{partial derivatives $\partial_i$}.

Note that the $\Be_i$ are a \emph{Clifford basis} and they satisfy the properties of a Clifford basis, i.e. the $\Be_i \Be_i = 1$ and $\Be_i \Be_j = - \Be_j \Be_i$ for $i \ne j$.
Depending on the coordinate system we will identify the basis vector as reported in Table \ref{sigmai_table}.

\begin{table}[]
\centering
\caption{Basis vector in rectangular, cylindrical and spherical coordinate systems. The $\Be_i$ are a Clifford basis and correspond to the appropriate Pauli matrices.}
%\hspace*{2.0cm}
\label{sigmai_table}
%\begin{tabular}{l c l c  l  l  l }
\begin{tabular}{| c | c | c | c | }
\hline
 & $\Be_1$ & $\Be_2$  & $\Be_3$  \\
\hline
rectangular & $\sigma_1$ & $\sigma_2$  &  $\sigma_3$ \\ \hline
cylindrical & $\sigma_\rho$ & $\sigma_\phi$  &  $\sigma_3$    \\ \hline
spherical & $\sigma_r$ & $\sigma_\theta$  &  $\sigma_\phi$   \\ \hline
\end{tabular}
\end{table}




\end{frame}

\begin{frame}[shrink=00]{Nabla definition}

The scaling coefficients are reported in  Table \ref{Ci_table}. The partial derivatives symbols $\partial_i$ assume the meaning reported in Table \ref{partiali_table}

\begin{table}[]
\centering
\caption{Scaling coefficients in rectangular, cylindrical and spherical coordinate systems. }
%\hspace*{2.0cm}
\label{Ci_table}
%\begin{tabular}{l c l c  l  l  l }
\begin{tabular}{| c | c | c | c | }
\hline
 & $c_1$ & $c_2$  & $c_3$  \\
\hline
rectangular & 1 & 1  &  1 \\ \hline
cylindrical &1 & $1/\rho$  &  1    \\ \hline
spherical & 1 & $1/r$  &  $1/(r \sin(\theta))$   \\ \hline
\end{tabular}
\end{table}

\begin{table}[]
\centering
\caption{Partial derivatives for rectangular, cylindrical and spherical coordinate systems. }
%\hspace*{2.0cm}
\label{partiali_table}
%\begin{tabular}{l c l c  l  l  l }
\begin{tabular}{| c | c | c | c | }
\hline
 & $\partial_1$ & $\partial_2$  & $\partial_3$  \\
\hline
rectangular & $\partial_x$ & $\partial_y$  & $\partial_z$ \\ \hline
cylindrical &$\partial_\rho$ & $\partial_\phi$  & $\partial_z$    \\ \hline
spherical & $\partial_r$ & $\partial_\theta$  & $\partial_\phi$   \\ \hline
\end{tabular}
\end{table}


\end{frame}

\begin{frame}[shrink=00]{Fundamental identity (the geometric product)}

From the fundamental identity (the geometric product) we have:
%
\begin{eqnarray}
 \nabla \, {\bf F}  & = &   \nabla \cdot \BF + \nabla \wedge \BF   \, .
\end{eqnarray}
%
The term $\BF$ can be a multivector composed, in general, by a scalar part (grade 0), a vector part (grade 1), a bivector part (grade 2) and a trivector or pseudo scalar (grade 3). 
In Table \ref{nabla_scalar} it  is reported the application of the nabla operator to a scalar function.


\begin{table}[]
\centering
\caption{ For a scalar function $\phi$ application of the nabla operator gives a vector  (i.e. the gradient), here assumed equal to the external product of $\nabla$ and $\phi$. Note that in the geometric product expansion of $\phi$ also a term of the type $\nabla \cdot \phi$ is present, but this term is equal to zero.
Further application of the nabla operator provides the Laplacian (which is of 0 grade) and the term $\nabla \wedge \nabla \wedge \phi =0$.}
%\hspace*{2.0cm}
\label{nabla_scalar}
%\begin{tabular}{l c l c  l  l  l }
\begin{tabular}{| c | c | c | c | c |}
%
\hline
Grade & $0$ & $1$  & $2$ & $3$ \\
\hline
quantity & $\phi$ &   &  &  \\ \hline
$\nabla \phi$ & 0 &  $\nabla \wedge \phi = \nabla \phi$ &  &  \\ \hline
$\nabla \nabla \phi= \nabla^2 \phi$ & $\nabla \cdot \nabla \phi$ &   & $\nabla \wedge \nabla \wedge \phi =0$ &  \\ \hline
\end{tabular}
\end{table}




\end{frame}

\begin{frame}[shrink=10]{Grade structure}

In Table \ref{A_vector}  the nabla operator has been applied to a vector function.
\begin{table}[]
\centering
\caption{ For a vector function $\BA$ application of the nabla operator gives a scalar  (i.e. the divergence), and a bivector. 
From $\nabla \cdot \BA$ a further application of the nabla operator gives the vector term $\nabla \nabla \cdot \BA$, while the nabla operator applied to
$\nabla \wedge \BA$ gives the term $\nabla \cdot \nabla \wedge \BA $ and the term $\nabla \wedge \nabla \wedge \BA $ which is equal to zero.
}
%\hspace*{2.0cm}
\label{A_vector}
%\begin{tabular}{l c l c  l  l  l }
\begin{tabular}{| c | c | c | c | c |}
%
\hline
Grade & $0$ & $1$  & $2$ & $3$ \\
\hline
quantity &  & $\BA$   &  &  \\ \hline
$\nabla \BA$ & $\nabla \cdot \BA$ &    & $\nabla \wedge \BA$ &  \\ \hline
$\nabla \nabla \BA= \nabla^2 \BA$ & & $\nabla \nabla \cdot \BA + \nabla \cdot \nabla \wedge \BA $  &  & $\nabla \wedge \nabla \wedge \BA =0$  \\ \hline
\end{tabular}
\end{table}

It is noted that by substituting the vector $\BA$ with $i \BB$ we have the corresponding table for nabla operating on a bivector. Similarly, by substituting $\phi$ with $i \psi$ we have nabla operating on a pseudoscalar.

\end{frame}

%\begin{frame}[shrink=00]{}
%\end{frame}



\section{Nabla operator in rectangular coordinates}
\begin{frame}[shrink=00]{Nabla operator in rectangular coordinates}


By using Pauli matrices a field vector $\BF$ may be written as
%
\begin{eqnarray}
 \tF  & = &  \sigma_1 F_x + \sigma_2 F_y  + \sigma_3 F_z \nonumber \\
       & = &  \begin{pmatrix}{F}_{z} & {F}_{x}-i\,{F}_{y}\cr i\,{F}_{y}+{F}_{x} & -{F}_{z}\end{pmatrix}  
% H  & = &    \begin{pmatrix}{H}_{z} & {H}_{x}-j\,{H}_{y}\cr j\,{H}_{y}+{H}_{x} & -{H}_{z}\end{pmatrix} 
\end{eqnarray}

%
\pause
Similarly, the Pauli matrix representation of the $\nabla$ operator takes the form
%
\begin{eqnarray}
\tnabla  & = &  
 \sigma_1 \partial_x + \sigma_2 \partial_y  + \sigma_3 \partial_z \nonumber \\
  & = &  
 \begin{pmatrix}{\partial}_{z} & {\partial}_{x}-i\,{\partial}_{y}\cr i\,{\partial}_{y}+{\partial}_{x} & -{\partial}_{z}\end{pmatrix} \, .
  %
\end{eqnarray}
%
From the fundamental identity (the geometric product) we have:
%
\begin{eqnarray}
 \nabla \, {\bf F}  & = &   \nabla \cdot \BF + \nabla \wedge \BF   \, .
\end{eqnarray}
%


\end{frame}

\begin{frame}[shrink=20]{Matrix product evaluation}
When evaluating $\nabla \, {\bf F}$ via Pauli matrices we simply need to perform the following matrix product:
\small
\bea
\tnabla \, \tF & = & 
\begin{pmatrix}{\partial}_{z} & {\partial}_{x}-i\,{\partial}_{y}\cr i\,{\partial}_{y}+{\partial}_{x} & -{\partial}_{z}\end{pmatrix} \, 
\begin{pmatrix}{F}_{z} & {F}_{x}-i\,{F}_{y}\cr i\,{F}_{y}+{F}_{x} & -{F}_{z}\end{pmatrix} \nonumber  \\
% & = & \begin{pmatrix}
%\partial_z\,{F}_{z}+\partial_y\,{F}_{y}+i\,\left(\partial_x\,{F}_{y}\right) -i\,\left(\partial_y\,{F}_{x}\right) +\partial_x\,{F}_{x} & i\,\left(\partial_y\,{F}_{z}\right) -\partial_x\,{F}_{z}-i\,\left(\partial_z\,{F}_{y}\right) +\partial_z\,{F}_{x}\cr i\,\left(\partial_y\,{F}_{z}\right) +\partial_x\,{F}_{z}-i\,\left(\partial_z\,{F}_{y}\right) -\partial_z\,{F}_{x} &\partial_z\,{F}_{z}+\partial_y\,{F}_{y}-i\,\left(\partial_x\,{F}_{y}\right) +i\,\left(\partial_y\,{F}_{x}\right) +\partial_x\,{F}_{x}\end{pmatrix} \nonumber \\
& = & \begin{pmatrix}\partial_z\,{F}_{z}+\partial_y\,{F}_{y}+\partial_x\,{F}_{x} & 0\cr 0 & \partial_z\,{F}_{z}+\partial_y\,{F}_{y}+\partial_x\,{F}_{x}\end{pmatrix} +\nonumber \\
& + & 
\begin{pmatrix}i\,\left( \partial_x\,{F}_{y}-\partial_y\,{F}_{x}\right)  & i\,\left( \partial_y\,{F}_{z}\right) -\partial_x\,{F}_{z}-i\,\left( \partial_z\,{F}_{y}\right) +\partial_z\,{F}_{x}\cr i\,\left( \partial_y\,{F}_{z}\right) +\partial_x\,{F}_{z}-i\,\left( \partial_z\,{F}_{y}\right) -\partial_z\,{F}_{x} & -i\,\left( \partial_x\,{F}_{y}-\partial_y\,{F}_{x}\right) \end{pmatrix} \,. \label{nablaF}
%
\eea
\normalsize
In (\ref{nablaF}) we have separated the scalar part corresponding to $\nabla \cdot \BF$ (diagonal matrix) from the external product $ \nabla \wedge \BF$.

In the following code it is shown how to compute, for a vector $\BE$, the two geometric products $\nabla \BE$ and $\nabla \nabla \BE$.

\end{frame}

\begin{frame}[shrink=00]{}

\small
\lstinputlisting{wxMaxima/1Pauli_nabla3rect.wxm}
\normalsize

\end{frame}

\begin{frame}[shrink=00]{Code}

Instead of programming several times the nabla operation it is more convenient to use the block function.

We have one block called Nablarect that performs the nabla operator on a given Pauli matrix.

Then we have another block, called Gradesep, which for a given Pauli matrix separate the different grades.
\end{frame}

%\begin{frame}[shrink=00]{}
%
%\small
%\lstinputlisting{Pauli_nabla2rect.wxm}
%\normalsize
%
%\end{frame}

\begin{frame}[shrink=20]{Second order expressions}
In several instances it is necessary to form the second order expressions, e.g. $\nabla \nabla \, {\bf F}$. Computation of this quantity via Pauli matrices is embarrassing simple, since only matrix multiplication is  required
%
\small
\bea
\tnabla \, \tnabla \,\tF & = & 
\begin{pmatrix}{\partial}_{z} & {\partial}_{x}-i\,{\partial}_{y}\cr i\,{\partial}_{y}+{\partial}_{x} & -{\partial}_{z}\end{pmatrix} \, 
\begin{pmatrix}{\partial}_{z} & {\partial}_{x}-i\,{\partial}_{y}\cr i\,{\partial}_{y}+{\partial}_{x} & -{\partial}_{z}\end{pmatrix} \, 
\begin{pmatrix}{F}_{z} & {F}_{x}-i\,{F}_{y}\cr i\,{F}_{y}+{F}_{x} & -{F}_{z}\end{pmatrix} \nonumber  \\
& = & 
\begin{pmatrix}{\partial}_{z}^{2}+\left( {\partial}_{x}-i\,{\partial}_{y}\right) \,\left( i\,{\partial}_{y}+{\partial}_{x}\right)  & 0\cr 0 & {\partial}_{z}^{2}+\left( {\partial}_{x}-i\,{\partial}_{y}\right) \,\left( i\,{\partial}_{y}+{\partial}_{x}\right) \end{pmatrix}
\begin{pmatrix}{F}_{z} & {F}_{x}-i\,{F}_{y}\cr i\,{F}_{y}+{F}_{x} & -{F}_{z}\end{pmatrix} \nonumber  \\
% & = & 
% \begin{pmatrix}
% \left( {\partial}_{z}^{2}+{\partial}_{y}^{2}+{\partial}_{x}^{2}\right) \,{F}_{z} 
% & {\partial}_{z}^{2} \, \left( {F}_{x}-i\,{F}_{y}\right) +\left( -i\,{\partial}_{y}^{2}-i\,{\partial}_{x}^{2}\right) \,{F}_{y}+ {\partial}_{y}^{2} \, {F}_{x}+{\partial}_{x}^{2}\,{F}_{x}\cr 
% \,{\partial}_{z}^{2} \, \left( i\,{F}_{y}+{F}_{x}\right) + \left( i\,{\partial}_{y}^{2}+i\,{\partial}_{x}^{2}\right) \,{F}_{y}+ {\partial}_{y}^{2} \, {F}_{x}\,+{\partial}_{x}^{2}\,{F}_{x} & \left( -{\partial}_{z}^{2}-{\partial}_{y}^{2}-{\partial}_{x}^{2}\right) \,{F}_{z}\end{pmatrix} \nonumber \\
 & = & 
 \begin{pmatrix}
 \Delta \,{F}_{z} 
 &  \Delta \, {F}_{x} - i \, \Delta \, {F}_{y} \cr 
  i \, \Delta \, {F}_{y} + \Delta \, {F}_{x}
 & \Delta \,{F}_{z}\end{pmatrix} \nonumber  \\
 & = & \Delta \,\tF
 \label{nablanablaF}
%
\eea
\normalsize
where we have introduced the laplacian $\Delta$ defined as
\be
\Delta = {\partial}_{z}^{2}+{\partial}_{y}^{2}+{\partial}_{x}^{2}
\ee

\end{frame}




%%%%%%%%%%%%%%%%%%%%%%%%%%%%%%%%%%%%%%%%%%%%%%%%
\section{Nabla operator in circular cylindrical coordinates using Pauli matrices}

\begin{frame}[shrink=00]{Nabla operator in circular cylindrical coordinates}
The gradient operator of a scalar function $w$ in circular cylindrical coordinates has the following form:
%
\be \label{gradcyl}
\nabla w = \Bu_\rho \, \frac{\partial\,w}{\partial\,\rho} 
+ \Bu_\phi \,\frac{ 1 }{\rho} \frac{\partial\,w}{\partial\,\phi}  + \Bu_z\frac{\partial\,w}{\partial\,z} \, ,
\ee
%
from which we can infer the Pauli matrix representation of the nabla operator $\tilde{\nabla}$ in cylindrical coordinates.
\pause

By substituting the unit versors with the matrices in Table \ref{sigmai_table}, and using the abbreviated  notation e.g. $\partial_\rho = \partial / \partial \rho$, we obtain
%
\bea
\tilde{\nabla} & = & \sigma_\rho \, \partial_\rho + \frac{ 1 }{\rho}\sigma_\phi \, \partial_\phi + \sigma_z \, \partial_z \nonumber \\
& = & \begin{pmatrix} \partial_z & 
\frac{{e}^{-i\,\phi}}{\rho}\,\left(\rho\,{\partial}_{\rho} - i\,{\partial}_{\phi}\right) \cr 
\frac{{e}^{i\,\phi}}{\rho}\,\left(\rho\,{\partial}_{\rho} + i\,{\partial}_{\phi}\right)  & -\partial_z \end{pmatrix} \nonumber \\
& = & 
 \begin{pmatrix}
 \partial_z & \nabla_t^* \cr
\nabla_t & -\partial_z
 \end{pmatrix} \label{nablaPcyl}
\eea

\end{frame}

\begin{frame}[shrink=00]{Nabla operator in circular cylindrical }
where we have introduced the operator $\nabla_t$ and its complex conjugate $\nabla_t^*$ defined as 
%
\bea
\nabla_t &=& \frac{{e}^{i\,\phi}}{\rho}\, \left(\rho\,{\partial}_{\rho} + i\,{\partial}_{\phi}\right) \nonumber \\
\nabla_t^* &=& \frac{{e}^{-i\,\phi}}{\rho}\, \left(\rho\,{\partial}_{\rho} - i\,{\partial}_{\phi}\right) \, .
\eea
%
\pause

Let us now consider a vector $\BA$ expressed in terms of cylindrical Pauli matrices as:
\be \label{Acyl}
\tilde{A} = \begin{pmatrix}A_z & {e}^{-i\,\phi} \, \left( A_\rho-i\,A_\phi\right) \cr {e}^{i\,\phi} \, \left( A_\rho+i\,A_\phi\right)  & -A_z\end{pmatrix}
\ee
and let us recall that, due to the fundamental identity of geometric algebra, we also have:
\be
\nabla \BA = \nabla \cdot \BA + \nabla \wedge \BA \, .
\ee
%


\end{frame}

\begin{frame}[shrink=20]{vector $\BA$ expressed in terms of cylindrical Pauli matrices}
%
By performing the matrix multiplication of (\ref{nablaPcyl}) with (\ref{Acyl}), and by denoting with $\left(\tilde{\nabla}\,\tilde{A}\right)_{ij} $ the $ij$ element of the matrix, we obtain:
\bea
\left(\tilde{\nabla}\,\tilde{A}\right)_{11} & = & \nabla \cdot \BA +  i \left[ \frac{A_\phi}{\rho} + \frac{\partial \, A_\phi}{\partial\,\rho} -\frac{1}{\rho} \frac{\partial \, A_\rho}{\partial\,\phi} \right]
\nonumber \\
%
\left(\tilde{\nabla}\,\tilde{A}\right)_{12} & = & \left[ - \left(
\frac{\partial \,A_z}{\partial\,\rho}-\frac{\partial \,A_\rho}{\partial \,z} \right)
+ i \left(  \frac{1}{\rho}\frac{\partial \, A_z}{\partial\,\phi}       - \frac{\partial \, A_\phi}{\partial\,z}\, \right) \right]\, {e}^{-i\,\phi}
\nonumber \\
%
\left(\tilde{\nabla}\,\tilde{A}\right)_{21} & = & \left[ \left(
\frac{\partial \,A_z}{\partial\,\rho}-\frac{\partial \,A_\rho}{\partial \,z} \right)
+ i \left(  \frac{1}{\rho}\frac{\partial \, A_z}{\partial\,\phi}       - \frac{\partial \, A_\phi}{\partial\,z}\, \right) \right]\, {e}^{i\,\phi}
\nonumber \\
%
\left(\tilde{\nabla}\,\tilde{A}\right)_{22} & = & \nabla \cdot \BA -  i \left[ \frac{A_\phi}{\rho} + \frac{\partial \, A_\phi}{\partial\,\rho} -\frac{1}{\rho} \frac{\partial \, A_\rho}{\partial\,\phi} \right] \, .
\label{nablaAcylcomp}
\eea
%
In (\ref{nablaAcylcomp}), the divergence term
\be
\nabla \cdot \BA =  \frac{\partial \, A_\rho}{\partial\,\rho} +  \frac{ A_\rho}{\rho} + \frac{1}{\rho}\frac{\partial \, A_\phi}{\partial\,\phi} + \frac{\partial \, A_z}{\partial\,z}
\ee
has been singled out.


\end{frame}





\begin{frame}[shrink=00]{The matrix $\tilde{\nabla}\,\tilde{A}$}
The matrix $\tilde{\nabla}\,\tilde{A}$ contains therefore the divergence term, which can be obtained by dividing by two the matrix trace, and contains also all the terms relative to the external product o$\tilde{\nabla}$ and $\tilde{A}$. 

The specific component of the part corresponding to the external product, i.e. to the bivector, is obtained in the following manner.

First the matrix $\tilde{\nabla} \wedge \tilde{A}$ is obtained as
\be
\tilde{\nabla} \wedge \tilde{A} = \tilde{\nabla}\,\tilde{A} - \left( \nabla \cdot \BA \right)\, \sigma_0 \,.
\ee
Then the component are retrieved by dot multiplication for the appropriate Pauli matrix. 
The dot multiplication is obtained from the multiplication of the matrices and then by taking half of the trace. 

\end{frame}

\begin{frame}[shrink=00]{Curl cylindrical coordinates}
As an example for the $\rho$ component we have
\bea
\left(\nabla \times \BA\right)_\rho & = & \frac{1}{2\, i} trace\left\{\left( \tilde{\nabla} \wedge \tilde{A} \right) \sigma_\rho\right\} 
\nonumber \\
& = & \frac{1}{\rho} \frac{\partial \, A_z}{\partial\,\phi} - \frac{\partial \, A_\phi}{\partial\,z} \label{retrieve_rho}
\eea
while for the $\phi$ component one has
\bea
\left(\nabla \times \BA\right)_\phi & = & \frac{1}{2\, i} trace\left\{\left( \tilde{\nabla} \wedge \tilde{A} \right) \sigma_\phi\right\} 
\nonumber \\
& = &  \frac{\partial \, A_\rho}{\partial\,z} - \frac{\partial \, A_z}{\partial\,\rho} \label{retrieve_phi}
\eea
and, finally, for the $z$ component we obtain
\bea
\left(\nabla \times \BA\right)_z & = & \frac{1}{2\, i} trace\left\{\left( \tilde{\nabla} \wedge \tilde{A} \right) \sigma_z\right\} 
\nonumber \\
& = & \frac{A_{\phi}}{\rho} + \frac{\partial \, A_\phi}{\partial\,\rho} + \frac{1}{\rho} \frac{\partial \, A_\rho}{\partial\,\phi} \, .
 \label{retrieve_z}
\eea

\end{frame}

\begin{frame}[shrink=00]{Second order operators}

%\subsubsection{Second order operators}
Further application of the nabla operator allows to obtain
\be 
\tilde{\nabla}\, \tilde{\nabla}\,\tilde{A} = \tilde{\nabla}^2\,\tilde{A} 
\ee
which, in terms of Pauli matrices, can be be computed by matrix multiplication. 
\pause
In vector terms the second order operator gives the vector Laplacian 
$\nabla^2 \BA$:
\be 
\nabla\left( \nabla \BA\right) =  \nabla^2 \BA = \nabla \cdot \nabla \wedge \BA + \nabla \wedge \nabla \cdot \BA
\label{vectorLaplacian}
\ee
which shows that the final result is a vector.

It is convenient to introduce the scalar Laplacian operator $\nabla_s^2 $ defined as:
\be \label{laplacianscyl}
\nabla_s^2 = \frac{\partial^2 \, }{\partial\,\rho^2} + \frac{1}{\rho^2} \frac{\partial^2 \, }{\partial\,\phi^2} + \frac{\partial^2 \, }{\partial\,z^2} +
\frac{1}{\rho} \frac{\partial \, }{\partial\,\rho}
\ee
%


\end{frame}

\begin{frame}[shrink=00]{Differences between the scalar Laplacian (\ref{laplacianscyl}) and the vector Laplacian (\ref{vectorLaplacian})}
As a result of the matrix multiplication, and by identifying the various components, the following results are obtained:
%
\bea 
%
\left( \tilde{\nabla}^2\,\tilde{A} \right)_\rho  & = &  \nabla_s^2 A_\rho  - \frac{A_\rho} {\rho^2}  - \frac{2}{\rho^2} \frac{\partial \, A_\phi}{\partial \, \phi}
\nonumber \\
%
\left( \tilde{\nabla}^2\,\tilde{A} \right)_\phi  & = &  \nabla_s^2 A_\phi  - \frac{A_\phi} {\rho^2}  + \frac{2}{\rho^2} \frac{\partial \, A_\rho}{\partial \, \phi}
\nonumber \\
%
\left( \tilde{\nabla}^2\,\tilde{A} \right)_\rho  & = &  \nabla_s^2 A_z \, .
\label{nabla2:cyl}
%
\eea
%
\pause
\alert{It is important to note  the differences between the scalar Laplacian (\ref{laplacianscyl}) and the vector Laplacian (\ref{vectorLaplacian})}.

In conventional vector algebra the vector Laplacian is expressed as
\be 
  \nabla^2 \BA = \nabla \nabla \cdot  \BA - \nabla \times \nabla \times \BA \, .
\label{vectorLaplacian:conv}
\ee
\end{frame}

\begin{frame}[shrink=00]{Code for Nabla operations and grade retrieval in cylindrical coordinates}
A  code illustrating the nabla operation and  the grade extraction is shown next.

The block Nablacyl performs the nabla operation on a Pauli matrix.

The block Gradecyl perform the grade extraction on a Pauli matrix in cylindrical coordinates.

Optionally these operations are used to find the gradient of a line charge distribution and the magnetic field on an infinite current line.
\end{frame}


\begin{frame}[shrink=00]{}
\clearpage
\small
\lstinputlisting{wxMaxima/2Nabla_cyl_4_blocks.wxm}
\normalsize
\end{frame}



\begin{frame}[shrink=00]{Code}
In the following code the relevant quantities are computed first in terms of conventional operators. 

Then they are computed in terms of Pauli matrices.

When feasible a test has been introduced showing that the results are coincident.

%\small
%\begin{verbatim}
%wxMaxima/Pnabla_cyl.wxm
%\end{verbatim}
%\normalsize
%
%
It is noted that so far we have considered all the components with all the spatial dependencies. 

In several cases noticeable simplifications can occur. 

Consider e.g. the case of infinite filament of constant current, placed in the $z$ direction. 

The potential $A$ will be of the form $A_z(\rho)$ and therefore the problem is considerably simpler.

%\begin{comment}

%\end{comment}

\end{frame}
\begin{frame}[shrink=00]{}
\clearpage
\small
\lstinputlisting{wxMaxima/2Pnabla_cyl.wxm}
\normalsize
\end{frame}


%%%%%%%%%%%%%%%%%%%%%%%%%%%%%%%%%%%%%%%%%%%%%%%%
\section{Nabla operator in spherical coordinates using Pauli matrices}
\begin{frame}[shrink=20]{Nabla operator in spherical coordinates using Pauli matrices}

The gradient operator of a scalar function $w$ in spherical coordinates has the following form:
%
\be \label{gradcyl}
\nabla w = \Bu_r \, \frac{\partial\,w}{\partial\,r} 
+ \Bu_\theta \,\frac{ 1 }{r} \frac{\partial\,w}{\partial\,\theta}  + \frac{1}{r \, \sin{\theta}}\Bu_\phi\frac{\partial\,w}{\partial\,\phi} \, ,
\ee
%
from which we can infer the Pauli matrix representation of the nabla operator $\tilde{\nabla}$ in spherical coordinates.


\pause
By substituting the unit versors with the matrices in Table \ref{sigmai_table}, and using the abbreviated  notation e.g. $\partial_r = \partial / \partial r$, we obtain
%
\bea
\tilde{\nabla} & = & \sigma_r \, \partial_r + \frac{ 1 }{r}\sigma_\theta \, \partial_\theta + \frac{1}{r \, \sin{\theta}} \sigma_\phi \, \partial_\phi \nonumber \\
& = & 
\begin{pmatrix}  \cos \theta \, {\partial}_{r} -\sin \theta \, {\partial}_{\theta}\,  & 
{e}^{-i\,\phi}\,\left( \sin \theta\, {\partial}_{r} \,+\cos \theta \,{\partial}_{\theta}\,  -i \,{\partial}_{\phi}\right) \cr 
{e}^{ i\,\phi}\,\left( \sin \theta\, {\partial}_{r} \,+\cos \theta \,{\partial}_{\theta}\,  +i \,{\partial}_{\phi}\right)& 
-\left(\cos \theta \, {\partial}_{r} -\sin \theta \, {\partial}_{\theta}\,\right)
\end{pmatrix} 
\label{nablaPsph}
\eea

\end{frame}

\begin{frame}[shrink=20]{vector $\BA$}
Let us now consider a vector $\BA$ expressed in terms of spherical Pauli matrices as:
\be \label{Asph}
\tilde{A} = 
\begin{pmatrix} {A}_{r}\, \cos \theta-{A}_{\theta}\,\sin \theta  & 
{e}^{-i\,\phi}\,\left({A}_{r}\, \sin \theta \,-i \,{A}_{\phi}+{A}_{\theta}\,\cos \theta\right) \cr 
{e}^{i\,\phi}\,\left({A}_{r}\, \sin \theta \,+i \,{A}_{\phi}+{A}_{\theta}\,\cos \theta\right) & 
-{A}_{r}\, \cos \theta+{A}_{\theta}\,\sin \theta
\end{pmatrix} 
\ee

and let us recall that, due to the fundamental identity of geometric algebra, we also have:

%
\be
\nabla \BA = \nabla \cdot \BA + \nabla \wedge \BA \, .
\ee
%

By performing the matrix multiplication of (\ref{nablaPsph}) with (\ref{Asph}), we obtain
%
\be
\tilde{\nabla} \, \tilde{A}  = \nabla \cdot \BA \, \sigma_0 + n_r  \,\sigma_r + n_\theta \, \sigma_\theta + n_\phi \,  \sigma_\phi \, ,
\ee
%
where the following symbols have been used:
\end{frame}

\begin{frame}[shrink=20]{$\tilde{\nabla} \, \tilde{A} $}

\bea
\nabla \cdot \BA & = & \partial_r \, A_r + \frac{2 \, A_r}{r} + \frac{\partial_\theta A_\theta}{r} + \frac{\cos \theta \, A_\theta}{r \, \sin \theta} + \frac{\partial_\phi A_\phi}{r\, \sin \theta}
\nonumber \\
%
n_r & = &  \frac{i}{r\, \sin \theta} \left(  - \partial_\phi A_\theta + \sin \theta \,  \partial_\theta A_\phi  + \cos \theta \, A_\phi \right)
 \nonumber \\
%
n_\theta & = & i \left(   \frac{\partial_\phi \, A_r}{r\, \sin \theta} -  \frac{A_\phi}{r} - \partial_r \, A_\phi \right)
\nonumber \\
%
n_\phi & = &  i \left( \frac{\partial_\theta A_r}{r} - \frac{ A_\theta}{r}  + \partial_r A_\theta     \right)  \, .
\label{nablaAsphcomp}
\eea
%
In (\ref{nablaAsphcomp}) the term $\nabla \cdot \BA$ is the conventional divergence. 

Since $\nabla \wedge \BA = i \nabla \times \BA$ the other terms $n_r, n_\theta, n_\phi$ are simply the components along $r, \theta$ and $\phi$ of the curl operator multiplied by $i$.

\alert{It is noted that while in conventional algebra there is no single operator providing both the divergence and the curl, by using Pauli matrices a single operator (\ref{nablaPsph}) exists for the nabla representation.}

\end{frame}

\begin{frame}[shrink=00]{Second order operators}

%\subsubsection{Second order operators}
Further application of the nabla operator allows to obtain
\be 
\tilde{\nabla}\, \tilde{\nabla}\,\tilde{A} = \tilde{\nabla}^2\,\tilde{A} 
\ee
which, in terms of Pauli matrices, can be be computed by matrix multiplication. 

In vector terms the second order operator gives the vector Laplacian 
$\nabla^2 \BA$:
\be 
\nabla\left( \nabla \BA\right) =  \nabla^2 \BA = \nabla \cdot \nabla \wedge \BA + \nabla \wedge \nabla \cdot \BA
\label{vectorLaplacian2}
\ee
which shows that the final result is a vector.

\pause
It is convenient to introduce the scalar Laplacian operator for spherical coordinates $\nabla_s^2 $ defined as:
%
\be \label{laplaciansph}
 \nabla_s^2 w  = \frac{1}{r^2}\frac{\partial}{\partial r}\left(r^2  \frac{\partial w}{\partial r}\right)  +   
 \frac{1}{r^2 \sin \theta}   \frac{\partial}{\partial \theta}  \left(  \sin \theta\frac{\partial w}{\partial \theta} \right) 
+  \frac{1}{r^2 \sin^2 \theta}\frac{\partial^2 w}{\partial \phi^2}  
\ee
%



\end{frame}

\begin{frame}[shrink=00]{ Vector Laplacian}

and the coefficients:

\bea
%
N_r & = & \nabla_s^2 A_r - \frac{2}{r^2 \, \sin \theta}
\left( A_r \, \sin \theta + A_\theta \, \cos \theta + \sin \theta \partial_\theta A_\theta + \partial_\phi A_\phi   \right)
 \nonumber \\
%
N_\theta & = & \nabla_s^2 A_\theta + \frac{1}{r^2 \, \sin^2 \theta}
 \left( {2\, \sin^2 \theta \, \partial_\theta \,A_r - A_\theta - 2\, \cos \theta \partial_\phi \, A_\phi}   \right)
\nonumber \\
%
N_\phi & = &  \nabla_s^2 A_\theta + \frac{1}{r^2 \, \sin^2 \theta}
 \left( {2\, \partial_\phi \,A_r + 2\, \cos \theta \partial_\phi \, A_\theta - A_\phi }   \right)  \, .
\label{nablanablaAsphcomp}
\eea
%
The second order nabla operator can finally be expressed as:
%
\be
\tilde{\nabla}^2 \, \tilde{A}  =  N_r  \,\sigma_r + N_\theta \, \sigma_\theta + N_\phi \,  \sigma_\phi \, ,
\ee
%
\end{frame}





\begin{frame}[shrink=00]{Differences between the scalar Laplacian (\ref{laplacianscyl}) and the vector Laplacian (\ref{vectorLaplacian2})}

It is important to note  the differences between the scalar Laplacian (\ref{laplacianscyl}) and the vector Laplacian (\ref{vectorLaplacian2}).
In conventional vector algebra the vector Laplacian is expressed as
\be 
  \nabla^2 \BA = \nabla \nabla \cdot  \BA - \nabla \times \nabla \times \BA \, .
\label{vectorLaplacian:conv}
\ee

In the following code the relevant quantities are computed first in terms of conventional operators. Then they are computed in terms of Pauli matrices.
When feasible a test has been introduced showing that the results are coincident.


\end{frame}

\begin{frame}[shrink=00]{}
%\begin{comment}
\clearpage
\small
\lstinputlisting{wxMaxima/3Pnabla_sph2.wxm}
\normalsize
\newpage
%\end{comment}

\end{frame}

%%%%%%%%%%%%%%%%%%%%%%%%%%%%%%%%%%%%%%%%%
\section{Relating vector algebra to conventional vector analysis}
\begin{frame}[shrink=00]{Triple products}

We have previously shown that the following identities between vectors hold:
%
\bea 
\Ba \wedge \Bb & = & i \, \Ba \times \Bb \label{eqextcurl} \\
\Ba \cdot  \Bb \wedge \Bc & = & - \, \Ba \times \Bb  \times \Bc \label{eqdotext} \\
\Ba \wedge \Bb \wedge \Bc& = & i \, \Ba \cdot ( \Bb \times \Bc) \label{eqextext} \, .
\eea
%
\pause
These equations allows to translate into conventional vector analysis (VA) the identities obtained in geometric algebra (GA) and viceversa.
As an example, consider the triple geometric product $\Ba \, \Bb \, \Bc$ expressed as:
\bea
\Ba \, \Bb \, \Bc & = & \Ba \left( \Bb \cdot \Bc + \Bb \wedge \Bc      \right) \nonumber \\
& = & \Ba \cdot \left( \Bb \cdot \Bc + \Bb \wedge \Bc  \right)    + \Ba \wedge \left( \Bb \cdot \Bc + \Bb \wedge \Bc      \right) \nonumber \\
& = & \Ba \cdot  \Bb \cdot \Bc +  \Ba \cdot (\Bb \wedge \Bc)      + \Ba \wedge  (\Bb \cdot \Bc)   + \Ba \wedge \Bb \wedge \Bc       \nonumber \\
& = & - \, \Ba \times \Bb  \times \Bc + \Ba (\Bb \cdot \Bc) + i \, \Ba \cdot ( \Bb \times \Bc) \label{tripleabc}
\eea
where the term $\Ba \cdot  \Bb \cdot \Bc $ evidently is equal to zero.

\end{frame}

\begin{frame}[shrink=00]{Triple products: grade interpretation}

Computationally, by using the Pauli matrices ($\ta, \tb, \tc$) the triple product is trivial and is simply a matrix product. 

However, from direct interpretation of the result in (\ref{tripleabc}) one may also recognize:
\begin{itemize}
\item the term $\Ba \cdot (\Bb \wedge \Bc) = - \, \Ba \times \Bb  \times \Bc $ is the dot product of a vector and a bivector giving as a result a vector;
\item the term $\Ba \wedge  (\Bb \cdot \Bc)= \Ba (\Bb \cdot \Bc)$ corresponds also to a vector;
\item the last term $\Ba \wedge \Bb \wedge \Bc  = i \, \Ba \cdot ( \Bb \times \Bc)$ is a trivector (or pseudoscalar).
\end{itemize}

The above identities are also useful in order to find some relationships concerning the nabla operator.

\end{frame}

\begin{frame}[shrink=10]{The nabla operator in geometric algebra}

%\subsubsection{The nabla operator in geometric algebra}
It is instructive to start from (\ref{eqextcurl})--(\ref{eqextext}) and to formally substitute $\Ba$ with the nabla operator $\nabla$, thus obtaining:
%
\bea 
\nabla \wedge \Bb & = & i \, \nabla \times \Bb \label{neqextcurl} \\
\nabla \cdot  (\Bb \wedge \Bc) & = & - \, \nabla \times (\Bb  \times \Bc) \label{neqdotext} \\
\nabla \wedge (\Bb \wedge \Bc)& = & i \, \nabla \cdot ( \Bb \times \Bc) \label{neqextext} \, .
\eea
%
Equation (\ref{neqextcurl}) simply relates  the external product of $\nabla$ with a vector $\Bb$ with the curl operation. 
\pause

The other two equations (\ref{neqdotext}), (\ref{neqextext}) are interesting, since they refer to the divergence and the external product of a \emph{bivector}.
\pause

Equation (\ref{neqdotext}) states that the divergence of a bivector $\Bb\wedge\Bc$ is equal to minus the curl of $\Bb \times \Bc$., i.e. to a vector.

\pause
Equation (\ref{neqextext}) tells us that the external product of $\nabla$ with a bivector $\Bb\wedge\Bc$ is equal to the divergence of $\Bb \times \Bc$ multiplied by $i$.


\end{frame}

\begin{frame}[shrink=10]{Proof of $\nabla \cdot  (\Bb \wedge \Bc)  =  - \, \nabla \times (\Bb  \times \Bc)$}

An alternative way to prove (\ref{neqdotext}) is the following.
Let us assume
\be
\nabla = \Be_1 \partial_1 + \Be_2 \partial_2 + \Be_3 \partial_3
\ee
and consider a bivector ${\hat \BB}$ given by
\be
{\hat \BB} = i \BB = i\left( \Be_1 B_1 +  \Be_2 B_2+  \Be_3 B_3 \right) = \Be_{23} B_1 + \Be_{31} B_2 + \Be_{12} B_3 \, .
\ee
%
Dot multiplication provides:
%
\bea
\nabla \cdot  {\hat \BB} & = & \left( \Be_1 \partial_1 + \Be_2 \partial_2 + \Be_3 \partial_3 \right) \cdot \left( \Be_{23} B_1 + \Be_{31} B_2 + \Be_{12} B_3 \right) \nonumber \\
& = & - \Be_3 \partial_1 B_2 + \Be_2 \partial_1 B_3
+  \Be_3 \partial_2 B_1 - \Be_1 \partial_2 B_3
-  \Be_2 \partial_3 B_1 + \Be_1 \partial_3 B_2 
\nonumber \\
& = & - \left[ 
\Be_1 \left( \partial_2 B_3 -  \partial_3 B_2 \right)  + 
\Be_2 \left( \partial_3 B_1 -  \partial_1 B_3 \right)  + 
\Be_3 \left( \partial_1 B_2 -  \partial_2 B_1 \right)  
\right] \nonumber \\
& = & - \nabla \times \BB
\eea
which proves the relation (\ref{neqextcurl}).

We also have
\alert{$\nabla \cdot  (\nabla \wedge \Bc)  =  - \, \nabla \times (\nabla  \times \Bc)$}

\end{frame}

\begin{frame}[shrink=10]{Proof of $\nabla \wedge (\Bb \wedge \Bc) =  i \, \nabla \cdot ( \Bb \times \Bc)$}

Equation (\ref{neqextext}) can be verified by considering:
\be
\nabla = \Be_1 \partial_1 + \Be_2 \partial_2 + \Be_3 \partial_3
\ee

\be
{\hat \BB} = i \BB = i\left( \Be_1 B_1 +  \Be_2 B_2+  \Be_3 B_3 \right) = \Be_{23} B_1 + \Be_{31} B_2 + \Be_{12} B_3 \, .
\ee

\bea
\nabla \wedge {\hat \BB} & = & \left( \Be_1 \partial_1 + \Be_2 \partial_2 + \Be_3 \partial_3 \right) \wedge \left( \Be_{23} B_1 + \Be_{31} B_2 + \Be_{12} B_3 \right) \nonumber \\
& = &  
 \Be_{123}  \partial_1 B_1 +
\Be_{231}  \partial_2 B_2 +
\Be_{312}  \partial_3 B_3 
\nonumber \\
& = & i \nabla \cdot \BB
\eea
since
\be
i =  \Be_{123} =\Be_{231} = \Be_{312} \,.
\ee

We also have $\nabla \wedge (\nabla \wedge \Bc) =  i \, \nabla \cdot ( \nabla \times \Bc)=0$
\end{frame}



%\begin{frame}[shrink=00]{}
%\end{frame}
%\begin{frame}[shrink=00]{}
%\end{frame}

\begin{frame}[shrink=00]{Geometric product of $\nabla(\Bb\,\Bc)$}

%\subsubsection{Geometric product of $\nabla(\Bb\,\Bc)$}
By formally substituting $\Ba$ with $\nabla$ and considering the latter acting on $\Bb\, \Bc$ one derives from (\ref{tripleabc}):
\bea
\nabla \, (\Bb\, \Bc) & = & \nabla \cdot (\Bb \wedge \Bc)      + \nabla \wedge  (\Bb \cdot \Bc)   + \nabla \wedge (\Bb \wedge \Bc)       \nonumber \\
& = & - \, \nabla \times (\Bb  \times \Bc) + \nabla (\Bb \cdot \Bc) + i \, \nabla \cdot ( \Bb \times \Bc) \, . \label{nablabc}
\eea
Naturally $\Bb \, \Bc$ is a product of two vectors and, as such, is a scalar plus a bivector. 
\pause

The operation $\nabla \wedge (\Bb \wedge \Bc)$  provides a trivector, while the other two operations, $\nabla \cdot (\Bb \wedge \Bc)$ and $\nabla \wedge  (\Bb \cdot \Bc)$, return a vector. 
\pause

Note that we have made use of the fact (cite Jancewicz) that, for a scalar function $\phi$, we may interpret $\nabla \phi = \nabla \wedge \phi$ i.e. the gradient is obtained as an external product of a vector with a scalar.


\end{frame}

\begin{frame}[shrink=00]{Second order derivatives}

Always referring to (\ref{tripleabc}) it is also possible to make the additional substitution of $\Bb$ with $ \nabla$. It is noted that in GA we have (the external part of a vector to itself is null)
\be \label{ww0}
\nabla \wedge \nabla \wedge \Bc = 0 \, ,
\ee
\pause
and therefore we get the identity:
\bea
\nabla \, \nabla\, \Bc & = & \nabla \cdot (\nabla \wedge \Bc)      + \nabla \wedge  (\nabla \cdot \Bc)        \\
& = & \nabla^2 \Bc = \Delta \Bc
\eea
where $\Delta $ is the \emph{Laplacian}. 

It is straightforward to prove that the above identity also holds for multivectors. 

\end{frame}

\begin{frame}[shrink=00]{Noticeable cases}
%\subsubsection{Noticeable cases}
From (\ref{ww0}) applied to a scalar function function $\phi$ and taking into account (\ref{eqextcurl}) one gets
\be \label{wwphi0}
\nabla \wedge \nabla \wedge \phi = 0 = i \, \left[ \nabla \times (\nabla \phi)\right] \, ,
\ee
i.e. the classical curl(grad $\phi = 0$). 

\pause
Moreover, from (\ref{ww0}) and (\ref{eqextext}) we get:
\be \label{wwc0}
\nabla \wedge \nabla \wedge \Bc = 0 = i \, \left[ \nabla \cdot (\nabla \times \Bc)\right] \, ,
\ee
i.e. div (curl $\Bc = 0$).



\end{frame}

\begin{frame}[shrink=00]{Consequences}
The above identities suggest that if we have e.g. a vector field $\BE$, which has to satisfy $\nabla \wedge \BE = 0$, such a field can be expressed as $\BE = \nabla \phi$ and (\ref{wwphi0}) will be automatically satisfied. 
\pause

Similarly, if we have a bivector field $\hat{\BH}=i \, \BH$, which has to satisfy 

$\nabla \wedge \hat{\BH} =   \nabla \cdot ( i \, \BH)=0$, 

the bivector field $\hat{\BH}$ can be expressed as $\hat{\BH} = \nabla \wedge \BA$, with (\ref{wwc0}) satisfied. 

\pause
Note that in (\ref{wwc0}) we can also give another interpretation by looking at the r.h.s.. In fact, we can say that if a vector $\BH$ has to have zero divergence than it can be expressed as the curl of a vector $\BA$. 

\end{frame}


\begin{frame}[shrink=30]{Summary of Pauli matrices}
% Table summarizing the basis with Pauli matrices
\begin{table}[]
\centering
\caption{Basis vector in rectangular, cylindrical and spherical coordinate systems. 
%
The $\Be_i$ are a Clifford basis and correspond to the appropriate Pauli matrices.}
%\hspace*{2.0cm}
\label{sigmai_table2}
%\begin{tabular}{l c l c  l  l  l }
\begin{tabular}{| c | c | c | c | }
\hline
 & $\Be_1$ & $\Be_2$  & $\Be_3$  \\
\hline
 & & & \\
rectangular 
& $\sigma_1 = \begin{pmatrix}0 & 1\cr 1 & 0\end{pmatrix}$ 
& $\sigma_2 = \begin{pmatrix}0 & -i\cr i & 0\end{pmatrix}$  
& $\sigma_3 = \begin{pmatrix}1 & 0\cr 0 & -1\end{pmatrix} $ \\ 
 & & & \\
 \hline
 & & & \\
cylindrical 
&  $\sigma_\rho =  \begin{pmatrix}0 & {e}^{-i\,\phi}\cr {e}^{i\,\phi} & 0\end{pmatrix}$ 
&  $\sigma_\phi = \begin{pmatrix}0 & -i\,{e}^{-i\,\phi}\cr i\,{e}^{i\,\phi} & 0\end{pmatrix} $  
&  $\sigma_3 = \begin{pmatrix}1 & 0\cr 0 & -1\end{pmatrix}$    \\
 & & & \\
 \hline
 & & & \\
spherical 
&  $\sigma_r =  \begin{pmatrix}\cos \theta  & {e}^{-i\,\phi}\, \sin \theta \cr {e}^{i\,\phi}  \, \sin \theta & -\cos \theta \end{pmatrix}$ 
&  $\sigma_\theta = \begin{pmatrix}-\sin\theta & {e}^{-i\,\phi}\, \cos \theta \cr {e}^{i\,\phi}  \, \cos \theta& \sin \theta \end{pmatrix}$  
&  $\sigma_\phi  =  \begin{pmatrix}0 & -i\,{e}^{-i\,\phi}\cr i\,{e}^{i\,\phi} & 0\end{pmatrix}$  \\ 
 & & & \\
 \hline
 \end{tabular}
\end{table}
\end{frame}



%
\begin{frame}[shrink=10]{Summary of vector expressions}
\begin{eqnarray}
 \tA  & = &  \sigma_1 A_x + \sigma_2 A_y  + \sigma_3 A_z \nonumber \\
       & = &  \begin{pmatrix}{A}_{z} & {A}_{x}-i\,{A}_{y}\cr i\,{A}_{y}+{A}_{x} & -{A}_{z}\end{pmatrix}  \nonumber
% H  & = &    \begin{pmatrix}{H}_{z} & {H}_{x}-j\,{H}_{y}\cr j\,{H}_{y}+{H}_{x} & -{H}_{z}\end{pmatrix} 
\end{eqnarray}

\bea 
\tilde{A} & = &  \sigma_\rho A_\rho + \sigma_\phi A_\phi  + \sigma_3 A_z \nonumber \\
& = & \begin{pmatrix}A_z & {e}^{-i\,\phi} \, \left( A_\rho-i\,A_\phi\right) \cr {e}^{i\,\phi} \, \left( A_\rho+i\,A_\phi\right)  & -A_z\end{pmatrix}
\nonumber
\eea

\bea 
\tilde{A} 
& = &   \sigma_r A_r  + \sigma_\theta A_\theta +\sigma_\phi A_\phi   \nonumber \\
& = &\begin{pmatrix} {A}_{r}\, \cos \theta-{A}_{\theta}\,\sin \theta  & 
{e}^{-i\,\phi}\,\left({A}_{r}\, \sin \theta \,-i \,{A}_{\phi}+{A}_{\theta}\,\cos \theta\right) \cr 
{e}^{i\,\phi}\,\left({A}_{r}\, \sin \theta \,+i \,{A}_{\phi}+{A}_{\theta}\,\cos \theta\right) & 
-{A}_{r}\, \cos \theta+{A}_{\theta}\,\sin \theta
\end{pmatrix} \nonumber
\eea%
\end{frame}
%

%
\begin{frame}[shrink=10]{Summary of Nabla expressions}
%
%Fundamental identity (the geometric product):
%%
%\begin{eqnarray}
% \nabla \, {\bf F}  & = &   \nabla \cdot \BF + \nabla \wedge \BF   \, .
%\end{eqnarray}
%
Rectangular:
\begin{eqnarray}
\tnabla  & = &  
 \sigma_1 \partial_x + \sigma_2 \partial_y  + \sigma_3 \partial_z \nonumber \\
  & = &  
 \begin{pmatrix}{\partial}_{z} & {\partial}_{x}-i\,{\partial}_{y}\cr i\,{\partial}_{y}+{\partial}_{x} & -{\partial}_{z}\end{pmatrix} \, \nonumber.
  %
\end{eqnarray}

Cylindrical:
%
\bea
\tilde{\nabla} & = & \sigma_\rho \, \partial_\rho + \frac{ 1 }{\rho}\sigma_\phi \, \partial_\phi + \sigma_z \, \partial_z \nonumber \\
& = & \begin{pmatrix} \partial_z & 
\frac{{e}^{-i\,\phi}}{\rho}\,\left(\rho\,{\partial}_{\rho} - i\,{\partial}_{\phi}\right) \cr 
\frac{{e}^{i\,\phi}}{\rho}\,\left(\rho\,{\partial}_{\rho} + i\,{\partial}_{\phi}\right)  & -\partial_z \end{pmatrix} \nonumber 
 \label{nablaPcyl2}
\eea

Spherical:
\bea
\tilde{\nabla} & = & \sigma_r \, \partial_r + \frac{ 1 }{r}\sigma_\theta \, \partial_\theta + \frac{1}{r \, \sin{\theta}} \sigma_\phi \, \partial_\phi \nonumber \\
& = & 
\begin{pmatrix}  \cos \theta \, {\partial}_{r} -\sin \theta \, {\partial}_{\theta}\,  & 
{e}^{-i\,\phi}\,\left( \sin \theta\, {\partial}_{r} \,+\cos \theta \,{\partial}_{\theta}\,  -i \,{\partial}_{\phi}\right) \cr 
{e}^{ i\,\phi}\,\left( \sin \theta\, {\partial}_{r} \,+\cos \theta \,{\partial}_{\theta}\,  +i \,{\partial}_{\phi}\right)& 
-\left(\cos \theta \, {\partial}_{r} -\sin \theta \, {\partial}_{\theta}\,\right) \nonumber
\end{pmatrix} 
\label{nablaPsph2}
\eea

\end{frame}

%\begin{frame}[shrink=00]{}
%\end{frame}
%
%\begin{frame}[shrink=00]{}
%\end{frame}
%
%

%\begin{frame}[shrink=20]{Coordinate systems}
%\end{frame}
%\begin{frame}[shrink=20]{Coordinate systems}
%\end{frame}
%
%\begin{frame}[shrink=20]{Coordinate systems}
%\end{frame}
%
%\begin{frame}[shrink=20]{Coordinate systems}
%\end{frame}
%
%\begin{frame}[shrink=20]{Coordinate systems}
%\end{frame}
%
%\begin{frame}[shrink=20]{Coordinate systems}
%\end{frame}
%
%\begin{frame}[shrink=20]{Coordinate systems}
%\end{frame}
%
%\begin{frame}[shrink=20]{Coordinate systems}
%\end{frame}
%
%\begin{frame}[shrink=20]{Coordinate systems}
%\end{frame}
%
%\begin{frame}[shrink=20]{Coordinate systems}
%\end{frame}
%
%\begin{frame}[shrink=20]{Coordinate systems}
%\end{frame}
%
%\begin{frame}[shrink=20]{Coordinate systems}
%\end{frame}



\end{document}
