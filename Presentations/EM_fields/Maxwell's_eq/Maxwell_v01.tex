%\documentclass[10pt]{beamer}
\documentclass[handout,10pt]{beamer}

\usetheme[progressbar=frametitle]{metropolis}

\usepackage{booktabs}
\usepackage[scale=2]{ccicons}


\usepackage{amsmath}
\usepackage{pgfplots}
\usepgfplotslibrary{dateplot}

\usepackage{xspace}
\newcommand{\themename}{\textbf{\textsc{metropolis}}\xspace}

%\usepackage{placeins} %%%
\usepackage{subfig}
\usepackage{physics}
\usepackage{amssymb}


\usepackage{tikz}
\usepackage{circuitikz}
\usepackage{siunitx}


\usepackage{latexsym}
\usepackage{mathtools}
\usepackage{slashed} % for the Feynman slash notation

\usepackage{listings}

\usepackage{balance}


% edited by Mauro 28-12-16
%
%% <local definitions>
\newcommand{\R}{\mathbb{R}}	
\newcommand{\C}{\mathbb{C}}
\newcommand{\HQ}{\mathbb{H}}
\newcommand{\N}{\mathbb{N}}
\newcommand{\be}{\begin{equation}}
\newcommand{\ee}{\end{equation}}	
\newcommand{\bea}{\begin{eqnarray}}
\newcommand{\eea}{\end{eqnarray}}	
\newcommand{\Pin}{\mathrm{Pin}}	
\newcommand{\Spin}{\mathrm{Spin}}
\renewcommand{\O}{\mathrm{O}}
\newcommand{\SO}{\mathrm{SO}}
\renewcommand{\eqref}[1]{(\ref{#1})}
\newcommand{\cl}[1]{\ensuremath{Cl(#1)}} % #1 stands for the values p,q. $\cl{p,q}$ produces 'Cl(p,q)'.
\newcommand{\gvec}[1]{\ensuremath{\mbox{\textbf{\textit{#1}}}}}
\newcommand{\vect}[1]{\ensuremath{\mbox{\textbf{\textit{#1}}}}}
%% </local definitions>

\newcommand{\Ba}[0]{\mathbf{a}}
\newcommand{\Bb}[0]{\mathbf{b}}
\newcommand{\Bc}[0]{\mathbf{c}}
\newcommand{\Bd}[0]{\mathbf{d}}
\newcommand{\Be}[0]{\mathbf{e}}
\newcommand{\Bf}[0]{\mathbf{f}}
\newcommand{\Bg}[0]{\mathbf{g}}
\newcommand{\Bh}[0]{\mathbf{h}}
\newcommand{\Bi}[0]{\mathbf{i}}
\newcommand{\Bj}[0]{\mathbf{j}}
\newcommand{\Bk}[0]{\mathbf{k}}
\newcommand{\Bl}[0]{\mathbf{l}}
\newcommand{\Bm}[0]{\mathbf{m}}
\newcommand{\Bn}[0]{\mathbf{n}}
\newcommand{\Bo}[0]{\mathbf{o}}
\newcommand{\Bp}[0]{\mathbf{p}}
\newcommand{\Bq}[0]{\mathbf{q}}
\newcommand{\Br}[0]{\mathbf{r}}
\newcommand{\Bs}[0]{\mathbf{s}}
\newcommand{\Bt}[0]{\mathbf{t}}
\newcommand{\Bu}[0]{\mathbf{u}}
\newcommand{\Bv}[0]{\mathbf{v}}
\newcommand{\Bw}[0]{\mathbf{w}}
\newcommand{\Bx}[0]{\mathbf{x}}
\newcommand{\By}[0]{\mathbf{y}}
\newcommand{\Bz}[0]{\mathbf{z}}
\newcommand{\BA}[0]{\mathbf{A}}
\newcommand{\BB}[0]{\mathbf{B}}
\newcommand{\BC}[0]{\mathbf{C}}
\newcommand{\BD}[0]{\mathbf{D}}
\newcommand{\BE}[0]{\mathbf{E}}
\newcommand{\BF}[0]{\mathbf{F}}
\newcommand{\BG}[0]{\mathbf{G}}
\newcommand{\BH}[0]{\mathbf{H}}
\newcommand{\BI}[0]{\mathbf{I}}
\newcommand{\BJ}[0]{\mathbf{J}}
\newcommand{\BK}[0]{\mathbf{K}}
\newcommand{\BL}[0]{\mathbf{L}}
\newcommand{\BM}[0]{\mathbf{M}}
\newcommand{\BN}[0]{\mathbf{N}}
\newcommand{\BO}[0]{\mathbf{O}}
\newcommand{\BP}[0]{\mathbf{P}}
\newcommand{\BQ}[0]{\mathbf{Q}}
\newcommand{\BR}[0]{\mathbf{R}}
\newcommand{\BS}[0]{\mathbf{S}}
\newcommand{\BT}[0]{\mathbf{T}}
\newcommand{\BU}[0]{\mathbf{U}}
\newcommand{\BV}[0]{\mathbf{V}}
\newcommand{\BW}[0]{\mathbf{W}}
\newcommand{\BX}[0]{\mathbf{X}}
\newcommand{\BY}[0]{\mathbf{Y}}
\newcommand{\BZ}[0]{\mathbf{Z}}

\newcommand{\ta}[0]{\tilde{a}}
\newcommand{\tb}[0]{\tilde{b}}
\newcommand{\tc}[0]{\tilde{c}}
\newcommand{\td}[0]{\tilde{d}}

\newcommand{\hA}[0]{\hat{A}}
\newcommand{\hB}[0]{\hat{B}}
\newcommand{\hH}[0]{\hat{H}}

\newcommand{\tA}[0]{\tilde{A}}
\newcommand{\tB}[0]{\tilde{B}}
\newcommand{\tF}[0]{\tilde{F}}
\newcommand{\tE}[0]{\tilde{E}}
\newcommand{\tH}[0]{\tilde{H}}

% spinors definition
\newcommand{\barJ}[0]{\bar{J}}
\newcommand{\barF}[0]{\bar{F}}
\newcommand{\barP}[0]{\bar{P}}
\newcommand{\barW}[0]{\bar{W}}



\newcommand{\tnabla}[0]{\tilde{\nabla}}
\newcommand{\tphi}[0]{\tilde{\phi}}
\newcommand{\tpsi}[0]{\tilde{\psi}}

%
\newcommand{\wavep}[0]{\partial^+}
\newcommand{\wavem}[0]{\partial^-}

\newcommand{\wavepp}[0]{\tilde{\partial}^+}
\newcommand{\wavemp}[0]{\tilde{\partial}^-}

\newcommand{\wavepd}[0]{\bar{\partial}^+}
\newcommand{\wavemd}[0]{\bar{\partial}^-}

\newcommand{\pbd}[0]{\bar{\partial}_d}

% frequency

\newcommand{\helmp}[0]{{\underline{\partial}}^+}
\newcommand{\helmm}[0]{{\underline{\partial}}^-}

\newcommand{\helmpp}[0]{{\underline{\tilde{\partial}}}^+}
\newcommand{\helmmp}[0]{{\underline{\tilde{\partial}}}^-}

\newcommand{\helmpd}[0]{{\underline{\bar{\partial}}}^+}
\newcommand{\helmmd}[0]{{\underline{\bar{\partial}}}^-}

\newcommand{\pbfd}[0]{{\underline{\bar{\partial}}}_d}




\def \figname {Figure}
\def \emode {E }
\def \hmode {H }
\def \temode {TE }
\def \tmmode {TM }
\def \temoden {TE${}_n$ }
\def \tmmoden {TM${}_n$ }
\def \temodemn {TE${}_{mn}$ }
\def \tmmodemn {TM${}_{mn}$ }



\newcommand{\iGA}{{i}}
\newcommand{\conjg}[1] {\ensuremath{#1}^*}

\setbeamertemplate{bibliography item}{[\theenumiv]}


\title{Maxwell's equations}

\date{}

%\subtitle{Maximizing efficiency and power at a fixed frequency}
%\date{\today}
%\author{Alessandra Costanzo, Franco Mastri, Mauro Mongiardo*, Giuseppina Monti}
%\institute{*Department of Engineering,
%University of Perugia, Italy}

\author{ Mauro Mongiardo$^1$}

\institute{
 $^1$ Department of Engineering, University of Perugia, Perugia, Italy.
}

%
\titlegraphic{\hfill\includegraphics[height=1.5cm]{logo}}


\begin{document}

\maketitle

\begin{frame}{Table of contents}
  \setbeamertemplate{section in toc}[sections numbered]
  \tableofcontents[hideallsubsections]
\end{frame}

%=========================================================================
\section{Maxwell's equations}
%=========================================================================

%=========================================================================
\begin{frame}[fragile]{Maxwell's equations}
Equations linking electromagnetic field quantities have been
introduced by \alert{James Clerk Maxwell} \index{Maxwell, J.C.} in an
elegant treatise first published in \alert{1873}.

\pause
 We assume that a student reader is familiar with
these equations. In what
follows, we summarize Maxwell's equations in \alert{time and frequency} domains.

%\subsection{Maxwell's Equations in Time--Dependent Form}
\pause

It is customary to write \alert{Maxwell's equations} \index{Maxwell's
equations}in either \alert{local or in global form}; we shall first consider
their local form. 

\pause
We also note that, unfortunately, it is customary
to designate the local form as differential form and this generates
some confusion with the general meaning that differential forms
have. 

\end{frame}



%=========================================================================
\begin{frame}[shrink=20]{Local Form of Maxwell's Equations}
%\subsubsection{Local Form of Maxwell's Equations}
In three--dimensional vector notation, with vector $\Br$ a
indicating a position in space and $t$ the time variable, Maxwell's
equations are \index{Maxwell's equations!differential
form}\index{Faraday, M.} \index{Amp\`ere, A.M.} \index{Gauss, K.F.}
%
\begin{subequations}
\begin{align}
  \nabla \times {\BE}({\Br},t) & =
  - {\partial {\BB}({\Br},t) \over
  \partial t} \, , &&\text{Faraday's law} \label{Faraday} \\
  \nabla \times {\BH}({\Br},t) & =
  {\partial {\BD}({\Br},t) \over
  \partial t} + {\BJ}({\Br},t) \, ,
  &&\text{Amp\`ere's law}  \label{Ampere} \\
  \nabla \cdot {\BD}({\Br},t)  & =
  \rho_e({\Br},t) \, , &&\text{Gauss' law}
  \label{Gausse} \\[1mm]
  \nabla \cdot {\BB}({\Br},t) & =
  0 \, , &&\text{Magnetic flux continuity}
  \label{Gaussm}
\end{align}
\end{subequations}
%
%
where bold face symbols denote vector quantities. The quantities are
defined as
%
\begin{align}
{\BE}({\Br},t)  \quad &
  \mbox{electric field strength}
 \nonumber \\
{\BD}({\Br},t) \quad &
  \mbox{electric displacement}
 \nonumber \\
{\BB}({\Br},t) \quad &
  \mbox{magnetic flux density}
 \nonumber \\
{\BH}({\Br},t) \quad &
  \mbox{magnetic field strength}
 \nonumber \\
{\BJ}({\Br},t)  \quad &
  \mbox{electric current density }
 \nonumber \\
\rho_e({\Br},t) \quad &
  \mbox{electric charge density }
 \nonumber
\end{align}
 %


\end{frame}

%=========================================================================
\begin{frame}[fragile]{Equations dependence}
\alert{Equations (\ref{Faraday})--(\ref{Gaussm}) are not independent}
since, for example, we may derive (\ref{Gaussm}) by taking the
divergence of (\ref{Faraday}). 

\pause

Another fundamental relationship
can be derived by introducing (\ref{Gausse}) into the divergence
of (\ref{Ampere})
%
\begin{eqnarray}
 \nabla \cdot {\BJ}({\Br},t) & = &
 - {\partial \rho_e ({\Br},t) \over\partial t}
\label{currentlaw}
\end{eqnarray}
%
which provides the \alert{conservation law} \index{Conservation law} for
electric charge and current densities.

\pause
Actually, the set of three equations (\ref{Faraday}), (\ref{Ampere})
and (\ref{currentlaw}) may be considered as the independent
equations describing macroscopic electromagnetic fields, since the
two Gauss equations (\ref{Gausse}) and (\ref{Gaussm}) can be derived
from this set. \index{Gauss, K.F.}

\end{frame}


%=========================================================================
\begin{frame}[fragile]{Static case}
%\paragraph{Static case}
Note that in the static case ${\partial  \over
\partial t}=0$ the electric and magnetic fields are not any more
interdependent and the equations  (\ref{Faraday}) -- (\ref{Gaussm})
become
%
\begin{subequations}
\begin{align}
\nabla \times {\BE}({\Br})  & =   0\,,
\\[1mm]
  \nabla \times {\BH}({\Br} )  & =  {\BJ}({\Br})\,,
\\[1mm]
  \nabla \cdot {\BD}({\Br})  & =   \rho_e({\Br})\,,
\\[1mm]
  \nabla \cdot {\BB}({\Br})  & =   0\,.
%\label{Maxwelleq}
\end{align}
\end{subequations}
%

\pause
Note that, if we assign the electric current density
${\BJ}({\Br})$ and the electric charge density
$\rho_e({\Br})$, we have, from (\ref{Faraday}) and
(\ref{Ampere}), two vector equations (i.e.\ six scalar equations)
while we have four unknown vectors (i.e.\ twelve scalar quantities).

\pause
To complete the number of equations we have to account for the media
properties expressed by the \alert{constitutive relations}.


\end{frame}


%=========================================================================
\section{Integral  Form of Maxwell's Equations}
%=========================================================================
\begin{frame}[fragile]{Maxwell's equations!integral form}

The properties of an electromagnetic field  may also be expressed
globally by an equivalent system of integral relations through use
of the two fundamental theorems of vector analysis: the divergence
theorem and Stokes' theorem

\pause
\alert{Divergence or Gauss' Theorem}

%
Let ${\BU}({\Br})$ be any vector function of position,
continuous together with its first derivative throughout a volume
$V$ bounded by a surface $S$. The divergence theorem states that
%
\begin{equation}
\oint_{S} {\BU}({\Br}) \cdot {\Bn} \; dS = \int_{V}
\nabla \cdot {\BU}({\Br})  \; dV\,, \label{divergenceq}
\end{equation}
%
where ${\bf n }$ is the outward unit vector normal to $S$. 

\pause

\alert{Gauss's theorem may also be used to \emph{define} the divergence}.


\end{frame}


%=========================================================================

%=========================================================================
\begin{frame}[fragile]{Stokes' Theorem}

\alert{Stokes' theorem}
%
Let ${\BU}({\Br})$ be any vector function of position,
continuous together with its first derivatives throughout an
arbitrary surface $S$ bounded by a contour $C$, and assumed to be
resolvable into a finite number of regular arcs. 

\pause
Stokes' theorem
(also called curl theorem) states that
%
\begin{equation}
%\oint_{C} {\BU}({\Br}) \cdot d{\bf \ell}  = \int_{S} \left[
\oint_{C} {\BU}({\Br}) \cdot d\, \Bl  = \int_{S} \left[
\nabla \times {\BU}({\Br})\right] \cdot {\Bn} \; dS \,,
\label{curleq}
\end{equation}
%
where $d\,\Bl$ is an element of length along $C$, and
${\Bn}$ is a unit vector normal to the positive side of the
element area $dS$ as defined by the right--hand thumb rule. 

\pause
\alert{This
relationship may also be considered as an equation defining the
\emph{curl} or \emph{circulation}}.\index{Circulation}


\end{frame}


%=========================================================================

%=========================================================================
\begin{frame}[fragile]{Integral form of Maxwell's equations}
By applying the curl theorem to (\ref{Faraday}) and
(\ref{Ampere}), and the divergence theorem to (\ref{Gausse}) and
(\ref{Gaussm}), we get

%
\begin{subequations}
\begin{align}
% \oint_{C} {\BE}({\Br},t) \cdot d{\bf {\ell}}   & =
  \oint_{C} {\BE}({\Br},t) \cdot d\,{\Bl}   & =
- \int_{S}{\partial {\BB}({\Br},t) \over \partial t} \cdot
{\Bn} \; dS\,,
  \label{intinduction}
\\[1mm]
% \oint_{C} {\BH}({\Br},t) \cdot d{\bf \ell}   & =
  \oint_{C} {\BH}({\Br},t) \cdot d\, \Bl   & =
\int_{S}{\partial {\BD}({\Br},t) \over \partial t} \cdot
{\Bn} \; dS + \int_{S}{{\BJ}({\Br},t)} \cdot {\Bn}
\; dS \,,
  \label{intAmpere}
\\[1mm]
 \int_{V} \nabla \cdot {\BD}({\Br},t) dv   & =
\oint_{S}{\BD}({\Br},t) \cdot {\Bn} \; dS = \int_{V}
\rho_e ({\Br},t) \; dv \,,
  \label{intGausse}
\\[1mm]
 \int_{V} \nabla \cdot {\BB}({\Br},t) dv   & =
\oint_{S}{\BB}({\Br},t) \cdot {\Bn} \; dS =0 \,.
  \label{intGaussm}
\end{align}
\end{subequations}

\end{frame}


%=========================================================================

%=========================================================================
\section{Maxwell's Equations in the Frequency Domain}

%=========================================================================
\begin{frame}[fragile]{Maxwell's Equations in the Frequency Domain}

%
Electromagnetic fields operating at a particular frequency are known
as \alert{time--harmonic steady--state or monochromatic fields}. 

\pause
By adopting
the time dependence $e^{j\omega t}$ to denote a time--harmonic field
with angular frequency $\omega$, we write
%
\begin{equation}
{\BE}({\Br}, t) = \Re \left\{ {\BE} ({\Br})
e^{j\omega t} \right\}\,, \label{phasor}
\end{equation}
%
where $\Re$ denotes the mathematical operator which selects the real
part of a complex quantity. 

\pause
The complex quantity is
$\BE(\Br)$ is called a \emph{vector phasor}. In (\ref{phasor}) we have used the same symbol to denote
both the real quantity in the time domain, ${\BE}({\Br},
t)$, and the complex quantity, ${\BE} ({\Br})$, in the
frequency domain. 

%\pause
%In what follows we shall generally refer to
%complex quantities unless otherwise explicitly stated.

%\pause


\end{frame}


%=========================================================================

%=========================================================================
\begin{frame}[fragile]{Example}

By applying (\ref{phasor}) to the field quantities appearing in
(\ref{Faraday}), (\ref{Ampere}), (\ref{Gausse}) and (\ref{Gaussm})
we obtain Maxwell's equations in the frequency domain. 
\pause

As an
example, let us consider (\ref{Faraday}) for which we have
%
\begin{equation}
\Re  \left\{ \left[ \nabla \times {\BE}({\Br}) +
  j \omega  {\BB}({\Br})  \right] e^{j\omega t} \right\}  =0
  \,.
\end{equation}
%

Since this equation is valid for {\em all times} $t$, we may make
use of the above lemma and state that the quantity inside the square
bracket must be equal to zero.

\end{frame}


%=========================================================================

%=========================================================================
\begin{frame}[fragile]{Maxwell's equation in frequency domain}

%
\begin{subequations}
\begin{align}
 \nabla \times {\BE}({\Br})  & =    - j\omega
 {\BB}({\Br})\,,
 \\[1mm]
 \nabla \times {\BH}({\Br})  & =     j\omega {\BD}({\Br})
+ {\BJ}({\Br})\,,  \\[1mm]
  \nabla \cdot {\BD}({\Br})  & =   \rho_e({\Br})\,,
  \\[1mm]
  \nabla \cdot {\BB}({\Br})  & =   0\,.
\end{align}
\end{subequations}
%
\end{frame}





%=========================================================================
\begin{frame}[fragile]{Some identities for expressing Maxwell's Equations in GA}
%\subsubsection{Some identities for expressing Maxwell's Equations in GA}
Let us consider a generic vector $\Ba$  expressed in terms of the Pauli representation $\ta$ as
\be
 \ta = \sigma_1 \, a_x + \sigma_2 \, a_y + \sigma_3 \, a_z
\label{asigma}
\ee
where the elements $\sigma_i$ constitutes the basis elements.

%\pause
%%(they can also be interpreted as Pauli matrices, but in the following we will simply make use of their properties as a GA basis). 
%The  bivector $\hB=i \sigma_0\, \tB$ is expressed, using the identity $i \sigma_0= \sigma_1 \, \sigma_2 \,\sigma_3$, as
%%
%\bea
%\hB &=&  i  \sigma_0\, \left(\sigma_1 \, B_x + \sigma_2 \, B_y + \sigma_3 \, B_z \right) \nonumber \\
%&=&  \sigma_2 \,\sigma_3 \, B_x + \sigma_3 \, \sigma_1 \, B_y + \sigma_1 \, \sigma_2 \, B_z
%\label{BBivsigma}
%\eea

\pause
The $\nabla$ operator, in cartesian coordinates, is given by:
\be
\tnabla = \sigma_1 \, \partial_x + \sigma_2 \, \partial_y + \sigma_3 \, \partial_z \,.
\label{nablasigma2}
\ee
Although we are referring to cartesian coordinates, the results derived next are valid in general.



\end{frame}

%=========================================================================
\begin{frame}[fragile]{External product of nabla with the vector $\Ba$}

The divergence is readily expressed as usual as:
\bea
\tnabla \cdot \ta & = & 
\left(  \sigma_1 \, \partial_x + \sigma_2 \, \partial_y + \sigma_3 \, \partial_z \right) 
\cdot \left(  \sigma_1  a_x +  \sigma_2 \, a_y + \sigma_3 \, a_z \right) 
\nonumber \\
& = &  \left( \partial_x \, a_x + \partial_y \, a_y + \partial_z \, a_z\right) \sigma_0 \,.
\eea

\pause
%
The external product of nabla with the vector $\ta$ is (by using 1,2,3 instead of x,y,z)
\bea
\tnabla \wedge \ta & = & \left(  \sigma_1 \, \partial_1 + \sigma_2 \, \partial_2 + \sigma_3 \, \partial_3 \right) 
\wedge  
\left(  \sigma_1  a_1 +  \sigma_2 \, a_2 + \sigma_3 \, a_3 \right)  \nonumber \\
& = & 
\sigma_1 \sigma_2 \left( \partial_1 a_2 - \partial_2 a_1\right) +
\sigma_1 \sigma_3 \left( \partial_1 a_3 - \partial_3 a_1\right) +
\sigma_2 \sigma_3 \left( \partial_2 a_3 - \partial_3 a_2\right) 
\nonumber \\
& = & i \left[
\sigma_1  \left( \partial_2 a_3 - \partial_3 a_2\right) -
\sigma_2  \left( \partial_1 a_3 - \partial_3 a_1\right)+
\sigma_3  \left( \partial_1 a_2 - \partial_2 a_1\right)
\right]
\nonumber \\
%& = &  \sigma_1 \, \sigma_2 \,\sigma_3\left( \partial_x \, B_x + \partial_y \, B_y\ + \partial_z \, B_z \right)  \nonumber \\
& = & i \nabla \times \Ba \, .
\eea

\end{frame}


%=========================================================================
\begin{frame}[fragile]{Some identities for expressing Maxwell's Equations in GA: Bivectors}
%\subsubsection{Some identities for expressing Maxwell's Equations in GA}
Let us consider a generic vector $\BB$  expressed in terms of the Pauli representation $\tB$ as
\be
 \tB = \sigma_1 \, B_x + \sigma_2 \, B_y + \sigma_3 \, B_z
\label{Bsigma}
\ee
where the elements $\sigma_i$ constitutes the basis elements.

\pause
%(they can also be interpreted as Pauli matrices, but in the following we will simply make use of their properties as a GA basis). 
The  bivector $\hB=i \sigma_0\, \tB$ is expressed, using the identity $i \sigma_0= \sigma_1 \, \sigma_2 \,\sigma_3$, as
%
\bea
\hB &=&  i  \sigma_0\, \left(\sigma_1 \, B_x + \sigma_2 \, B_y + \sigma_3 \, B_z \right) \nonumber \\
&=&  \sigma_2 \,\sigma_3 \, B_x + \sigma_3 \, \sigma_1 \, B_y + \sigma_1 \, \sigma_2 \, B_z
\label{BBivsigma}
\eea

\pause
The $\nabla$ operator, in cartesian coordinates, is given by:
\be
\nabla = \sigma_1 \, \partial_x + \sigma_2 \, \partial_y + \sigma_3 \, \partial_z \,.
\label{nablasigma}
\ee
Although we are referring to cartesian coordinates, the results derived next are valid in general.



\end{frame}

%=========================================================================
\begin{frame}[fragile]{External product of nabla with the bivector $\hB$}

The divergence is readily expressed as usual as:
\be
\nabla \cdot \BB = \partial_x \, B_x + \partial_y \, B_y\ + \partial_z \, B_z \,.
\ee

\pause
%
The external product of nabla with the bivector $\hB$ is
\bea
\nabla \wedge \hB & = & \left(  \sigma_1 \, \partial_x + \sigma_2 \, \partial_y + \sigma_3 \, \partial_z \right) 
\wedge  
\left(  \sigma_2 \,\sigma_3 \, B_x + \sigma_3 \, \sigma_1 \, B_y + \sigma_1 \, \sigma_2 \, B_z \right)  \nonumber \\
& = &  \sigma_1 \, \sigma_2 \,\sigma_3\left( \partial_x \, B_x + \partial_y \, B_y\ + \partial_z \, B_z \right)  \nonumber \\
& = & i \nabla \cdot \BB \, .
\eea

Note that 
\alert{the divergence of $\BB$ is a scalar}; when multiplied by $i$ it becomes a pseudoscalar. 
\pause

Also, when performing the external product of nabla with the bivector $\hB$ a pseudoscalar is obtained.

\end{frame}


%=========================================================================
\begin{frame}[fragile]{Divergence of the bivector $\hB$}

The divergence of the bivector $\hB$ is
\bea
\nabla \cdot \hB & = & \left(  \sigma_1 \, \partial_x + \sigma_2 \, \partial_y + \sigma_3 \, \partial_z \right) 
\cdot  
\left(  \sigma_2 \,\sigma_3 \, B_x + \sigma_3 \, \sigma_1 \, B_y + \sigma_1 \, \sigma_2 \, B_z \right)  \nonumber \\
& = & 
- \sigma_3 \, \partial_x \, B_y
+  \sigma_2 \, \partial_x \, B_z
+ \sigma_3 \, \partial_y \, B_x
- \sigma_1 \, \partial_y \, B_z
- \sigma_2 \, \partial_z \, B_x
+ \sigma_1 \, \partial_z \, B_y
 \nonumber \\
& = & - \nabla \times \BB  = i \nabla \wedge \BB\, .
\eea
In summary we have derived the following important identities:
\bea
\nabla \wedge \hB & = &  i \nabla \cdot \BB  \label{nablaestBiv}\\
\nabla \cdot \hB & = & - \nabla \times \BB  = i \, \nabla \wedge \BB\,  \label{nabladotBiv}\\
\nabla \, \hB & = & i \, \nabla \, \BB
\eea

\end{frame}

\section{Geometric Algebra form of Maxwell's Equations}

%=========================================================================
\begin{frame}[fragile]{Geometric Algebra form of Maxwell's Equations}

%
\alert{Faraday's law}

%\paragraph{Faraday's law}
By multiplying both sides of (\ref{Faraday}) times $i$ one obtains:
%
\be
\nabla \wedge \BE = - \partial_t \hB \,.
\ee
%
It is noted that is \alert{an equation of grade 2, i.e. between bivectors}.

\pause

\alert{Ampere's law}

%\paragraph{Ampere's law}
Let us now consider (\ref{Ampere}) and make use of (\ref{nabladotBiv}):
%
\be
\nabla \cdot \hH = - \partial_t \BD - \BJ \,.
\ee
%
This is an \alert{equation of grade 1, i.e. between vectors}.



\end{frame}


%=========================================================================
%=========================================================================
\begin{frame}[fragile]{}
\alert{Gauss' law}
This equation remains unchanged: in fact, by considering (\ref{Gausse}):
%
\be
\nabla \cdot \BD =  \rho_e 
\ee
%
we have an \alert{equation of grade 0, i.e. a scalar equation}.

\pause
\alert{Magnetic flux continuity}

By considering (\ref{Gaussm}), multiplying by $i$ and using (\ref{nablaestBiv}):
%
\be
\nabla \wedge \hB =  0 
\ee
%
we have an \alert{equation of grade 3, i.e. a pseudoscalar equation}.

\end{frame}


%=========================================================================

%=========================================================================
\begin{frame}[fragile]{GA equivalent of Maxwell's equations}

In summary the following form are the GA equivalent of Maxwell's equations local form:

\begin{subequations}
\begin{align}
\nabla \wedge \BE &= - \partial_t \hB \, , &&\text{grade 2}  \label{FaradayGA} \\
\nabla \cdot \hH &= - \partial_t \BD - \BJ  \, , &&\text{grade 1} \label{AmpereGA} \\
\nabla \cdot \BD &=  \rho_e  \, , &&\text{grade 0} \label{GausseGA} \\
\nabla \wedge \hB &=  0 \, , &&\text{grade 3} \label{GaussmGA}
\end{align}
\end{subequations}


\end{frame}


%=========================================================================

%=========================================================================
\begin{frame}[fragile]{GA equivalent of Maxwell's equations with magnetic sources}

Sometimes it is convenient to consider also magnetic sources 
%summary and a piecewise constant medium 
obtaining the following local form of Maxwell's equations:
\begin{subequations}
\begin{align}
\nabla \cdot \BE &=  \frac{\rho_e}{\epsilon}  \, , &&\text{grade 0} \label{GausseGA} \\
\nabla \cdot \left(i \eta \BH\right) &= - \frac{1}{c}\,  \partial_t \BE - \eta \,\BJ  \, , &&\text{grade 1} \label{AmpereGA} \\
\nabla \wedge \BE &= - \frac{1}{c}\,  \partial_t \, \left(i \eta \BH\right) \, -i \, \BM , &&\text{grade 2}  \label{FaradayGA} \\
\nabla \wedge \left(i \eta \BH\right) &=  - i \, c \, \rho_m \, , &&\text{grade 3} \label{GaussmGA}
\end{align}
\end{subequations}

\end{frame}



%=========================================================================
\begin{frame}[shrink=10]{Geometric Algebra Global form of Maxwell's Equations}
%\subsubsection{Geometric Algebra Global form of Maxwell's Equations}
Let us refer to the equations (\ref{intinduction})--(\ref{intGaussm}).
Instead of dealing with the term $\Bn \, dS$ we can now consider the bivector $d\, \hA$ denoting the oriented surface area. In addition note that when a volume term $dv$ is considered, it corresponds to a pseudoscalar as $dv = dx\, dy \, dz$.

\pause
\begin{subequations}
\begin{align}
% \oint_{C} {\BE}({\Br},t) \cdot d{\bf {\ell}}   & =
  \oint_{C} {\BE} \cdot d\,{\Bl}   & =
- \int_{S}\frac{\partial \hB }{ \partial t} \cdot
d \,\hA \,,
  \label{intinductionGA}
\\[1mm]
% \oint_{C} {\BH}({\Br},t) \cdot d{\bf \ell}   & =
  \oint_{C} {\hH} \wedge d\, \Bl   & =
\int_{S} \left( \BJ + \frac{\partial \BD} { \partial t} \right) \wedge
d\,{\hA}   \label{intAmpereGA}
\\[1mm]
 \int_{V} \nabla \cdot {\BD} \, dv   & =
\oint_{S}{\BD} \wedge d\,{\hA}  = \int_{V}
\rho_e  \; dv \,,
  \label{intGausseGA}
\\[1mm]
 \int_{V} \nabla \cdot {\BB} \, dv   & =
\oint_{S}{\hB} \cdot d \; \hA = 0 \,.
  \label{intGaussmGA}
\end{align}
\end{subequations}
\pause
It is noted that, in the above equations, the dot product between two bivectors give rise to a scalar, while the volume integral and the external product of a vector with a bivector produce a pseudoscalar.



\end{frame}




\section{Maxwell's Equations in compact form}
%=========================================================================
\begin{frame}[shrink=20,fragile]{Time--domain Maxwell's Equations in compact form}
 
Time--domain Maxwell's equations are commonly expressed as:
\bea 
%\label{Maxwell--time}
\nabla \times \BE & = & - \frac{\partial \, \BB}{\partial t} \label{curlE} \\
\nabla \times \BH & = &  \frac{\partial \, \BD}{\partial t} + \BJ \label{curlH} \\
\nabla \cdot \BD & = & \rho \label{divD} \\
\nabla \cdot \BH & = & 0 \label{divH} 
\eea
%
It is convenient to express the above equations making use of the light velocity in the medium $v$ and of the medium impedance $\eta$, recalling that:
{\small
\bea
v & = & \frac{1}{\sqrt{\mu \epsilon}} \label{clight} \\
\eta & = &\sqrt{  \frac{\mu}{\epsilon}} \label{etaimp} \\
\mu & = & \frac{\eta}{v} \label{lmu} \\
\epsilon & = &  \frac{1}{v\eta} \label{leps} \, .
\eea
}
%
%and noticing that
%%
%\be
%v\, \BB = \eta \, \BH 
%\ee
%



\end{frame}

%=========================================================================
\begin{frame}[fragile]{}
With a few superficial changes we can make more evident the symmetries in Maxwell equations as,
%
\bea 
%\label{Maxwell--time}
\nabla \times \BE  +  \frac{\partial \, v \BB}{\partial vt} & = &  0\label{curlE2} \\
\nabla \times v \BB - \frac{\partial \, \BE}{\partial vt} & = &   \eta \BJ \label{curlH2} \\
\nabla \cdot \BE & = & \frac{v\rho}{v \epsilon} = \eta v \rho \label{divD2} \\
\nabla \cdot v \BB & = & 0 \label{divH2} 
\eea
%
It is also noted that
\be
v\, \BB = \eta \, \BH 
\ee
so that one can change the expression containing $v \, \BB$ in $\eta \, \BH$ or viceversa.


\end{frame}

%=========================================================================
\begin{frame}[fragile]{}

Equation (\ref{curlE2}) can be multiplied by $i$ and (\ref{curlH2}) can be multiplied by $i^2$, obtaining
%\label{Maxwell--time}
\bea
i \nabla \times \BE  +  \frac{\partial \, i \, \eta \BH}{\partial vt} & = &  0 \label{curlE3} \\
i \nabla \times  i \, \eta \BH + \frac{\partial \, \BE}{\partial vt} & = &   - \eta \BJ \label{curlH3} \\
\nabla \cdot \BE & = & \eta \, v \rho \label{divD3} \\
\nabla \cdot i\, \eta \BH & = & 0 \label{divH3} 
\eea
%


By using the Pauli identity
we can write compactly (\ref{curlE3})--(\ref{divH3}) as
%
\bea
\nabla \,  \BE  +  \frac{\partial \, i \, \eta \BH}{\partial vt} & = &  \eta \, v \rho \label{nablaE} \\
\nabla \left(  i \, \eta \BH \right) + \frac{\partial \, \BE}{\partial vt} & = &   - \eta \BJ  \label{nablaH} \,.
\eea
\end{frame}

%=========================================================================
\begin{frame}[fragile]{}

A few observations are in order:
\begin{itemize}
\item
In every place where $t$ appears, we have arranged things so that $vt$ appears, rather than $t$ alone. The rationale is that $vt$ has the same dimensions as $x, y$, and $z$. 
%
%To say the same thing another way, in spacetime, the partner to $x, y$, and $z$ is not $t$ but rather $vt$.
\item
Similarly, the partner to $\BJ$ is not $\rho$ but rather $v\,\rho$. 
%In spacetime, $v\,\rho$ represents a certain amount of charge that sits at one spatial location and  flows toward the future, whereas $\BJ$ represents charge  flowing from one spatial location to another.
\item
Last but not least, the proper partner for $\BE$ is not $\BH$ but rather $\eta \, \BH$. In every place where $\BH$ appears, we have arranged things so the combination $\eta \, \BH$ appears, rather than $\BH$ alone. This is just an exercise in algebraic re--arrangement, and does not change the meaning of the equations. The rationale is that $\eta \, \BH$ has the same dimensions as $\BE$, and arranging things this way makes the equations more  symmetric. It is also noted that since $\eta \, \BH$ is always multiplied by $i$ it is a bivector while $\BE$ is a vector.
\end{itemize}


\end{frame}


%=========================================================================
\begin{frame}[fragile]{The field multivector: GA approach}

%\subsubsection{The field multivector}
%
The multivector $\cal{F}$, composed by a vector  and a bivector part, is now introduced with the following definition:
\be
%\cal{F} = \BE + \eta \, \hat{\BH} \label{Fpassage:20} \, .
{\cal{F}} = \BE +i \,  \eta \, {\BH} \label{Fpassage:20} \, .
\ee
By summing together (\ref{nablaE}) and (\ref{nablaH}), the well known results that allows to express the four Maxwell equation as a single one is recovered:
\be
\left(\nabla + \frac{1}{v}\partial_t\right) {\cal{F}} =  \eta  \, \left( v \rho  -  \BJ \right) \label{Fpassage:30} \, .
\ee

This expression, while being very synthetic, does not provide the same insight as the Dirac form, that we will introduce next.
\end{frame}


%=========================================================================
\begin{frame}[fragile]{}

The two equations (\ref{nablaE},\ref{nablaH}) in the sourceless case, may be rewritten in terms of Pauli matrices as
%
\bea
\tnabla \,  \tE  + \sigma_0 \frac{\partial \, i \, \eta \tH}{\partial vt} & = & 0 \label{nablaE_pauli} \\
\tnabla \left(  i \, \eta \tH \right) + \sigma_0 \frac{\partial \, \tE}{\partial vt} & = &   0 \label{nablaH_pauli} \,.
\eea
%

By using matrix notation we can write
%
\bea
\begin{pmatrix}
\frac{1}{v}\partial_t \, \sigma_0 & \tnabla \cr 
- \tnabla & -\frac{1}{v}\partial_t \, \sigma_0
\end{pmatrix}
\begin{pmatrix}
 \tE \cr  i \, \eta \tH
\end{pmatrix}
& = & 0
\label{Maxwell_Pauli_Dirac}
\eea
%
where we have changed sign at (\ref{nablaE_pauli}). Equation (\ref{Maxwell_Pauli_Dirac}) is ready to be cast in Dirac form, remembering that
\be
\tnabla   =   \sigma \cdot \nabla = \sigma_1 \partial_x + \sigma_2 \partial_y +\sigma_3 \partial_z  \nonumber
\ee
\end{frame}


%=========================================================================
\section{Maxwell's equations in Dirac form}
%=========================================================================
%=========================================================================
\begin{frame}[fragile]{Maxwell's equations}
It is convenient to introduce the following notation:
%
\bea
x_0 & = & vt \nonumber \\
x_1 & = & x \nonumber \\
x_2 & = & y \nonumber \\
x_3 & = & z
\eea
which is valid for the cartesian coordinate system. Similarly, we use for the derivatives the symbol
\bea
\partial_i & = &\frac{\partial}{\partial x_i} \, .
\eea
By changing the sign of eq. (\ref{nablaE}) it is possible to rewrite the Maxwell equations in a Dirac like notation in terms of the gamma matrices as
\be
\sum_{i=0}^3 \partial_i \gamma^i 
 \begin{pmatrix}{E}_{z}\cr i\,{E}_{y}+{E}_{x}\cr \eta\,i\,{H}_{z}\cr \eta\,\left( i\,{H}_{x}-{H}_{y}\right) \end{pmatrix} =
 -\eta 
 \begin{pmatrix} 
 {J}_{z} \cr
 i\,{J}_{y}+{J}_{x}\cr 
 v\,\rho_e\cr
 0 \end{pmatrix} 
\ee

\end{frame}

%=========================================================================
\begin{frame}[fragile]{Maxwell's equations}
We have already seen eq. (\ref{Maxwell_Pauli_Dirac}), here repeated for convenience for the sourceless case and expressed in the present notation:
%
\bea
\begin{pmatrix}
\partial_0 \, \sigma_0 & \tnabla \cr 
- \tnabla & -\partial_0 \, \sigma_0
\end{pmatrix}
\begin{pmatrix}
 \tE \cr  i \, \eta \tH
\end{pmatrix}
& = & 0
\label{Maxwell_Pauli_Dirac2}
\eea
%
with $\tnabla$ being
%
\be
\tnabla   =   \sigma \cdot \nabla = \sigma_1 \partial_1 + \sigma_2 \partial_2 +\sigma_3 \partial_3  \nonumber \,.
\ee


\end{frame}


%=========================================================================
\begin{frame}[fragile]{}
Therefore, if we write explicitly eq. (\ref{Maxwell_Pauli_Dirac2}) we have
%
\be 
\begin{pmatrix}
{\partial}_{0} & 0 & {\partial}_{3} & \partial_1 -i \partial_2 \cr 
0 & {\partial}_{0} & {\partial}_{1} + i\,{\partial}_{2} & -{\partial}_{3}\cr 
-{\partial}_{3} & - \partial_1 +i \partial_2& -{\partial}_{0} & 0 \cr 
-\partial_1 -i \partial_2 & {\partial}_{3} & 0 & -{\partial}_{0}
\end{pmatrix}
\begin{pmatrix}{E}_{z}\cr i\,{E}_{y}+{E}_{x}\cr \eta\,i\,{H}_{z}\cr \eta\,\left( i\,{H}_{x}-{H}_{y}\right) \end{pmatrix} =0 
\label{expMaxdir}
\ee
In (\ref{expMaxdir}) we have considered the sourceless case and we have used just the first column of the Pauli matrices representing the fields $\tE$
and $i \, \eta \tH$.
By using the Dirac gamma matrices we have
\be
\left(\gamma^0 \partial_0 + \gamma^1 \partial_1 + \gamma^2 \partial_2+ \gamma^3 \partial_3 \right) 
\begin{pmatrix}{E}_{z}\cr i\,{E}_{y}+{E}_{x}\cr \eta\,i\,{H}_{z}\cr \eta\,\left( i\,{H}_{x}-{H}_{y}\right) \end{pmatrix}
= 0
\ee


\end{frame}

%=========================================================================
\begin{frame}[fragile]{Maxwell--Dirac}
By taking into account also of the sources  it is possible to rewrite the Maxwell equations in a Dirac like notation in terms of the gamma matrices as
\be
\sum_{i=0}^3 \partial_i \gamma^i 
 \begin{pmatrix}{E}_{z}\cr i\,{E}_{y}+{E}_{x}\cr \eta\,i\,{H}_{z}\cr \eta\,\left( i\,{H}_{x}-{H}_{y}\right) \end{pmatrix} =
 -\eta 
 \begin{pmatrix} 
 {J}_{z} \cr
 i\,{J}_{y}+{J}_{x}\cr 
 v\,\rho_e\cr
 0 \end{pmatrix} 
 \label{4equations}
\ee
By introducing the Feynman slash notation
\be
{\slashed \partial} = \sum_{i=0}^3 \partial_i \gamma^i 
\ee

\end{frame}

%=========================================================================
\begin{frame}[fragile]{Maxwell--Dirac}
and the shorthand notation for the quadrivectors
\bea
\barF & = & \begin{pmatrix}{E}_{z}\cr i\,{E}_{y}+{E}_{x}\cr \eta\,i\,{H}_{z}\cr \eta\,\left( i\,{H}_{x}-{H}_{y}\right) \end{pmatrix} \nonumber \\
\barJ & = & \begin{pmatrix} 
 {J}_{z} \cr
 i\,{J}_{y}+{J}_{x}\cr 
 v\,\rho_e\cr
 0 \end{pmatrix}
\eea
%
we simply have 
\be
{\slashed \partial} \barF = -\eta \barJ
\label{Maxwell_Feynman}
\ee
%
which  presents a form similar to the Dirac equation for null mass.



\end{frame}


%
%%=========================================================================
%\begin{frame}[fragile]{}
%
%
%\end{frame}

%%=========================================================================
%\begin{frame}[fragile]{}
%
%
%\end{frame}
%
%
%%=========================================================================
%\begin{frame}[fragile]{}
%
%
%\end{frame}
%
%%=========================================================================
%\begin{frame}[fragile]{}
%
%
%\end{frame}

%%=========================================================================
%\begin{frame}[fragile]{}
%
%
%\end{frame}
%
%
%%=========================================================================
%\begin{frame}[fragile]{}
%
%
%\end{frame}
%
%%=========================================================================
%\begin{frame}[fragile]{}
%
%
%\end{frame}


%=========================================================================
\begin{frame}[fragile]{The quadrivector $\barF $}

The quadrivector $\barF $ can also be written in terms of waves
\bea
\barF & = & \begin{pmatrix}{E}_{z}\cr i\,{E}_{y}+{E}_{x}\cr \eta\,i\,{H}_{z}\cr \eta\,\left( i\,{H}_{x}-{H}_{y}\right) \end{pmatrix} 
 = 
\begin{pmatrix}{a}_{0}+{b}_{0}\cr {a}_{1}+{b}_{1}\cr {a}_{0}-{b}_{0}\cr {a}_{1}-{b}_{1}\end{pmatrix} 
 \nonumber 
\eea
%
%we simply have 
%\be
%{\slashed \partial} \barF = -\eta \barJ
%\label{Maxwell_Feynman}
%\ee
%%
%which  presents a form similar to the Dirac equation.


\end{frame}

\begin{frame}[fragile]{Explicit representation of ${\slashed \partial} \barF$}
 The representation of ${\slashed \partial} \barF$ in matrix terms is:
\be
{\slashed \partial} \barF = 
\begin{pmatrix}
{\partial}_{0} & 0 & {\partial}_{3} & \partial_1 -i \partial_2 \cr 
0 & {\partial}_{0} & {\partial}_{1} + i\,{\partial}_{2} & -{\partial}_{3}\cr 
-{\partial}_{3} & - \partial_1 +i \partial_2& -{\partial}_{0} & 0 \cr 
-\partial_1 -i \partial_2 & {\partial}_{3} & 0 & -{\partial}_{0}
\end{pmatrix}
\begin{pmatrix}{E}_{z}\cr i\,{E}_{y}+{E}_{x}\cr \eta\,i\,{H}_{z}\cr \eta\,\left( i\,{H}_{x}-{H}_{y}\right) \end{pmatrix} 
\nonumber
\ee

It is possible to note that the four equations are coupled:  \alert{we need to solve 4 coupled equations}.

It is possible to separate the problem into \alert{two systems of only two coupled equations} by using the Weyl decomposition.
\end{frame}

%=========================================================================
\section{Weyl decomposition}
%=========================================================================
\begin{frame}[fragile]{Weyl decomposition}

Let us consider the case without sources, which applies e.g. to propagation problems.

The four eqs. (\ref{4equations}) are coupled, but it is possible to separate them in two independent sets of two equations.

\pause
To this end it is convenient to introduce the two matrices
%
\bea
A^- & = & \frac{1}{\sqrt{2}} \left(\gamma^4 - \gamma^5 \right) \nonumber \\
A^+ & = & \frac{1}{\sqrt{2}} \left(\gamma^4 + \gamma^5 \right) \,.
\eea
%
\pause

These matrices square to the identity matrix thus being equal to their inverse. 
\be
A^-A^- = A^+A^+ = I_4
\ee

\pause



\end{frame}


%=========================================================================

%=========================================================================
\begin{frame}[fragile]{Weyl decomposition in matrix form}
It is also noted that
%
\be
 \begin{pmatrix}{a}_{0}\cr {a}_{1}\cr {b}_{0}\cr {b}_{1}\end{pmatrix}
 =  \frac{1}{\sqrt{2}}  \, A^- \, 
 \begin{pmatrix}{a}_{0}+{b}_{0}\cr {a}_{1}+{b}_{1}\cr {a}_{0}-{b}_{0}\cr {a}_{1}-{b}_{1}\end{pmatrix}  \, .
\ee
%
We can therefore transform eq. (\ref{Maxwell_Feynman}) as
\be
A^+ \, {\slashed \partial} \, A^- \, \, A^- \,\barF =0
\label{Maxwell_Feynman_mod}
\ee

\end{frame}


%=========================================================================

%=========================================================================
\begin{frame}[fragile]{Weyl decomposition in matrix form}
or,
explicitly, in cartesian coordinates, we have
%
\be
A^+ \, {\slashed \partial} \, A^- = 
\begin{pmatrix}
{\partial}_{3}+{\partial}_{0} & {\partial}_{1}-i\,{\partial}_{2} & 0 & 0\cr 
i\,{\partial}_{2}+{\partial}_{1} & {\partial}_{0}-{\partial}_{3} & 0 & 0\cr 
0 & 0 & {\partial}_{3}-{\partial}_{0} & {\partial}_{1}-i\,{\partial}_{2}\cr 
0 & 0 & i\,{\partial}_{2}+{\partial}_{1} & -{\partial}_{3}-{\partial}_{0}
\end{pmatrix}
\nonumber
\ee
%
\pause
i.e. the problem is decomposed in two systems as following:
%
\bea
\left( \tnabla + \sigma_0 \partial_0\right)  \begin{pmatrix}{a}_{0}\cr {a}_{1}\end{pmatrix}  & = & \tilde{\partial}^+ \begin{pmatrix}{a}_{0}\cr {a}_{1}\end{pmatrix} = 0 \nonumber \\
\left( \tnabla - \sigma_0 \partial_0\right)  \begin{pmatrix}{b}_{0}\cr {b}_{1}\end{pmatrix}  & = &  
 \tilde{\partial}^- \begin{pmatrix}{b}_{0}\cr {b}_{1}\end{pmatrix} =
 0 \, .
\label{nabla_sigma_ab}
\eea
%

\end{frame}


%=========================================================================

%=========================================================================
\begin{frame}[fragile]{An alternative approach}
At the same result we can arrive directly starting from (\ref{nablaE_pauli}), (\ref{nablaH_pauli}) and by introducing the column vectors $u,w$ defined as
%
\bea
u & = & \begin{pmatrix}{E}_{z}\cr i\,{E}_{y}+{E}_{x} \end{pmatrix}
=  \begin{pmatrix}{a}_{0}+{b}_{0}\cr {a}_{1}+{b}_{1}\end{pmatrix} 
  \nonumber \\
w & = &\begin{pmatrix} \eta\,i\,{H}_{z}\cr \eta\,\left( i\,{H}_{x}-{H}_{y}\right) \end{pmatrix} 
=
 \begin{pmatrix} {a}_{0}-{b}_{0}\cr {a}_{1}-{b}_{1}\end{pmatrix} 
\eea
%
\pause
we have
%
%
\bea
 \tnabla   u + \sigma_0 \,  \partial_0 \, w   & = & 0 \nonumber \\
\tnabla   w + \sigma_0 \,  \partial_0 \, u   & = & 0 \, .
\eea
%

\end{frame}


%=========================================================================

%=========================================================================
\begin{frame}[fragile]{Wave propagation equations}
By summing and subtracting the above eqs. we have:
%
\bea
 \tnabla   \left( u+w \right) + \sigma_0 \,  \partial_0 \,  \left( u+w \right)   & = & 0 \nonumber \\
\tnabla    \left( u-w \right) + \sigma_0 \,  \partial_0 \,  \left( u-w \right)   & = & 0 \, 
\eea
%
\pause
and since 
%
\bea
u+w & = & 2\, a = 2 \begin{pmatrix} {a}_{0}\cr {a}_{1}\end{pmatrix} \nonumber \\
u-w & = & 2\, b = 2 \begin{pmatrix} {b}_{0}\cr {b}_{1}\end{pmatrix} 
\eea
% 
\pause
we have obtained eqs. (\ref{nabla_sigma_ab}) or
%
\bea
\left( \tnabla + \sigma_0 \partial_0\right)  \begin{pmatrix}{a}_{0}\cr {a}_{1}\end{pmatrix}  & = & \tilde{\partial}^+ \begin{pmatrix}{a}_{0}\cr {a}_{1}\end{pmatrix} = 0 \nonumber \\
\left( \tnabla - \sigma_0 \partial_0\right)  \begin{pmatrix}{b}_{0}\cr {b}_{1}\end{pmatrix}  & = &  
 \tilde{\partial}^- \begin{pmatrix}{b}_{0}\cr {b}_{1}\end{pmatrix} =
 0 \, .
\label{nabla_sigma_ab}
\eea
%


\end{frame}

%%=========================================================================
%
%%=========================================================================
%\begin{frame}[fragile]{}
%
%
%\end{frame}
%
%
%=========================================================================
%
%
%%=========================================================================
%\begin{frame}[fragile]{}
%
%
%\end{frame}
%
%
%%=========================================================================
%\begin{frame}[fragile]{}
%
%
%\end{frame}
%
%%=========================================================================
%\begin{frame}[fragile]{}
%
%
%\end{frame}
%
%
%%=========================================================================
%
%%=========================================================================
%\begin{frame}[fragile]{}
%
%
%\end{frame}
%
%
%=========================================================================
%
%
%%=========================================================================
%\begin{frame}[fragile]{}
%
%
%\end{frame}


%%=========================================================================
%\begin{frame}[fragile]{}
%
%
%\end{frame}
%
%%=========================================================================
%\begin{frame}[fragile]{}
%
%
%\end{frame}


%
%%=========================================================================
%
%%=========================================================================
%\begin{frame}[fragile]{}
%
%
%\end{frame}


%=========================================================================


%%=========================================================================
%\begin{frame}[fragile]{}
%
%
%\end{frame}
%
%
%%=========================================================================
%\begin{frame}[fragile]{}
%
%
%\end{frame}
%
%%=========================================================================
%\begin{frame}[fragile]{}
%
%
%\end{frame}


%=========================================================================






\end{document}
