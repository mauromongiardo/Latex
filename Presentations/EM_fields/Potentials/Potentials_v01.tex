%\documentclass[10pt]{beamer}
\documentclass[handout,10pt]{beamer}

\usetheme[progressbar=frametitle]{metropolis}

\usepackage{booktabs}
\usepackage[scale=2]{ccicons}


\usepackage{amsmath}
\usepackage{pgfplots}
\usepgfplotslibrary{dateplot}

\usepackage{xspace}
\newcommand{\themename}{\textbf{\textsc{metropolis}}\xspace}

%\usepackage{placeins} %%%
\usepackage{subfig}
\usepackage{physics}
\usepackage{amssymb}


\usepackage{tikz}
\usepackage{circuitikz}
\usepackage{siunitx}


\usepackage{latexsym}
\usepackage{mathtools}
\usepackage{slashed} % for the Feynman slash notation

\usepackage{listings}

\usepackage{balance}


% edited by Mauro 28-12-16
%
%% <local definitions>
\newcommand{\R}{\mathbb{R}}	
\newcommand{\C}{\mathbb{C}}
\newcommand{\HQ}{\mathbb{H}}
\newcommand{\N}{\mathbb{N}}
\newcommand{\be}{\begin{equation}}
\newcommand{\ee}{\end{equation}}	
\newcommand{\bea}{\begin{eqnarray}}
\newcommand{\eea}{\end{eqnarray}}	
\newcommand{\Pin}{\mathrm{Pin}}	
\newcommand{\Spin}{\mathrm{Spin}}
\renewcommand{\O}{\mathrm{O}}
\newcommand{\SO}{\mathrm{SO}}
\renewcommand{\eqref}[1]{(\ref{#1})}
\newcommand{\cl}[1]{\ensuremath{Cl(#1)}} % #1 stands for the values p,q. $\cl{p,q}$ produces 'Cl(p,q)'.
\newcommand{\gvec}[1]{\ensuremath{\mbox{\textbf{\textit{#1}}}}}
\newcommand{\vect}[1]{\ensuremath{\mbox{\textbf{\textit{#1}}}}}
%% </local definitions>

\newcommand{\Ba}[0]{\mathbf{a}}
\newcommand{\Bb}[0]{\mathbf{b}}
\newcommand{\Bc}[0]{\mathbf{c}}
\newcommand{\Bd}[0]{\mathbf{d}}
\newcommand{\Be}[0]{\mathbf{e}}
\newcommand{\Bf}[0]{\mathbf{f}}
\newcommand{\Bg}[0]{\mathbf{g}}
\newcommand{\Bh}[0]{\mathbf{h}}
\newcommand{\Bi}[0]{\mathbf{i}}
\newcommand{\Bj}[0]{\mathbf{j}}
\newcommand{\Bk}[0]{\mathbf{k}}
\newcommand{\Bl}[0]{\mathbf{l}}
\newcommand{\Bm}[0]{\mathbf{m}}
\newcommand{\Bn}[0]{\mathbf{n}}
\newcommand{\Bo}[0]{\mathbf{o}}
\newcommand{\Bp}[0]{\mathbf{p}}
\newcommand{\Bq}[0]{\mathbf{q}}
\newcommand{\Br}[0]{\mathbf{r}}
\newcommand{\Bs}[0]{\mathbf{s}}
\newcommand{\Bt}[0]{\mathbf{t}}
\newcommand{\Bu}[0]{\mathbf{u}}
\newcommand{\Bv}[0]{\mathbf{v}}
\newcommand{\Bw}[0]{\mathbf{w}}
\newcommand{\Bx}[0]{\mathbf{x}}
\newcommand{\By}[0]{\mathbf{y}}
\newcommand{\Bz}[0]{\mathbf{z}}
\newcommand{\BA}[0]{\mathbf{A}}
\newcommand{\BB}[0]{\mathbf{B}}
\newcommand{\BC}[0]{\mathbf{C}}
\newcommand{\BD}[0]{\mathbf{D}}
\newcommand{\BE}[0]{\mathbf{E}}
\newcommand{\BF}[0]{\mathbf{F}}
\newcommand{\BG}[0]{\mathbf{G}}
\newcommand{\BH}[0]{\mathbf{H}}
\newcommand{\BI}[0]{\mathbf{I}}
\newcommand{\BJ}[0]{\mathbf{J}}
\newcommand{\BK}[0]{\mathbf{K}}
\newcommand{\BL}[0]{\mathbf{L}}
\newcommand{\BM}[0]{\mathbf{M}}
\newcommand{\BN}[0]{\mathbf{N}}
\newcommand{\BO}[0]{\mathbf{O}}
\newcommand{\BP}[0]{\mathbf{P}}
\newcommand{\BQ}[0]{\mathbf{Q}}
\newcommand{\BR}[0]{\mathbf{R}}
\newcommand{\BS}[0]{\mathbf{S}}
\newcommand{\BT}[0]{\mathbf{T}}
\newcommand{\BU}[0]{\mathbf{U}}
\newcommand{\BV}[0]{\mathbf{V}}
\newcommand{\BW}[0]{\mathbf{W}}
\newcommand{\BX}[0]{\mathbf{X}}
\newcommand{\BY}[0]{\mathbf{Y}}
\newcommand{\BZ}[0]{\mathbf{Z}}

\newcommand{\ta}[0]{\tilde{a}}
\newcommand{\tb}[0]{\tilde{b}}
\newcommand{\tc}[0]{\tilde{c}}
\newcommand{\td}[0]{\tilde{d}}

\newcommand{\hA}[0]{\hat{A}}
\newcommand{\hB}[0]{\hat{B}}
\newcommand{\hH}[0]{\hat{H}}

\newcommand{\tA}[0]{\tilde{A}}
\newcommand{\tB}[0]{\tilde{B}}
\newcommand{\tF}[0]{\tilde{F}}
\newcommand{\tE}[0]{\tilde{E}}
\newcommand{\tH}[0]{\tilde{H}}

% spinors definition
\newcommand{\barJ}[0]{\bar{J}}
\newcommand{\barF}[0]{\bar{F}}
\newcommand{\barP}[0]{\bar{P}}
\newcommand{\barW}[0]{\bar{W}}



\newcommand{\tnabla}[0]{\tilde{\nabla}}
\newcommand{\tphi}[0]{\tilde{\phi}}
\newcommand{\tpsi}[0]{\tilde{\psi}}

%
\newcommand{\wavep}[0]{\partial^+}
\newcommand{\wavem}[0]{\partial^-}

\newcommand{\wavepp}[0]{\tilde{\partial}^+}
\newcommand{\wavemp}[0]{\tilde{\partial}^-}

\newcommand{\wavepd}[0]{\bar{\partial}^+}
\newcommand{\wavemd}[0]{\bar{\partial}^-}

\newcommand{\pbd}[0]{\bar{\partial}_d}

% frequency

\newcommand{\helmp}[0]{{\underline{\partial}}^+}
\newcommand{\helmm}[0]{{\underline{\partial}}^-}

\newcommand{\helmpp}[0]{{\underline{\tilde{\partial}}}^+}
\newcommand{\helmmp}[0]{{\underline{\tilde{\partial}}}^-}

\newcommand{\helmpd}[0]{{\underline{\bar{\partial}}}^+}
\newcommand{\helmmd}[0]{{\underline{\bar{\partial}}}^-}

\newcommand{\pbfd}[0]{{\underline{\bar{\partial}}}_d}




\def \figname {Figure}
\def \emode {E }
\def \hmode {H }
\def \temode {TE }
\def \tmmode {TM }
\def \temoden {TE${}_n$ }
\def \tmmoden {TM${}_n$ }
\def \temodemn {TE${}_{mn}$ }
\def \tmmodemn {TM${}_{mn}$ }



\newcommand{\iGA}{{i}}
\newcommand{\conjg}[1] {\ensuremath{#1}^*}

\setbeamertemplate{bibliography item}{[\theenumiv]}


\title{Field Potentials}

\date{}

%\subtitle{Maximizing efficiency and power at a fixed frequency}
%\date{\today}
%\author{Alessandra Costanzo, Franco Mastri, Mauro Mongiardo*, Giuseppina Monti}
%\institute{*Department of Engineering,
%University of Perugia, Italy}

\author{ Mauro Mongiardo$^1$}

\institute{
 $^1$ Department of Engineering, University of Perugia, Perugia, Italy.
}

%
\titlegraphic{\hfill\includegraphics[height=1.5cm]{logo}}


\begin{document}

\maketitle

\begin{frame}{Table of contents}
  \setbeamertemplate{section in toc}[sections numbered]
  \tableofcontents[hideallsubsections]
\end{frame}


%=========================================================================
\section{The Field Multivector}
%=========================================================================
\begin{frame}[fragile]{The Field Multivector}

%
We have introduced the multivector $\cal{F}$, composed by a vector  and a bivector part
\be
%\cal{F} = \BE + \eta \, \hat{\BH} \label{Fpassage:20} \, .
{\cal{F}} = \BE +i \,  \eta \, {\BH} \label{Fpassage:20} \, .
\ee
%
\pause
and we have written the four Maxwell equation as a single one:
\be
\left(\nabla + \frac{1}{v}\partial_t\right) {\cal{F}} =  \eta  \, \left( v \rho  -  \BJ \right) \label{Fpassage:30a} \, .
\ee
By using the operator $\partial^+$ as
\be
\partial^+ = \nabla + \frac{1}{v}\partial_t
\ee
and the source multivector $\cal{J} $
\be
{\cal{J}} = \eta  \, \left( v \rho  -  \BJ \right)
\ee
which allows to write (\ref{Fpassage:30a}) as
\be
\partial^+ {\cal{F}}  = {\cal{J}} \, .
\ee

\end{frame}


%=========================================================================
\begin{frame}[shrink=10]{Operators Summary}

%\newpage
\begin{table}[]
\centering
\caption{Table summarizing operator definitions. The symbols with a tilde refer to 2 by 2 matrices, while the symbols with a bar refer to combination of 4 by 4 Dirac matrices. The underlined symbols refer to frequency domain operators.}
\label{operators_definitions}
\begin{tabular}{lllll}
\hline
\hline
\rule{0pt}{3ex} 
$ \partial^\pm$  & = & $\nabla \pm \frac{1}{v}\,\partial_t$ & 
%$\wavep \wavem$ 
&  
\\
\rule{0pt}{3ex} 
$\tilde{\partial}^\pm$ & =  & $\tnabla \pm \frac{1}{v}\,\partial_t \, \sigma_0$  &
%$\wavepp \wavemp$
  & 
   \\
\rule{0pt}{3ex}
$\bar{\partial}^\pm$ & = & 
$\begin{pmatrix}
 \pm \frac{1}{v}\,\partial_t \, \sigma_0 & \tnabla \cr 
\tnabla & \pm \frac{1}{v}\,\partial_t \, \sigma_0 \end{pmatrix}$ &
%$\wavepd \wavemd$
  &  
  \\
\rule{0pt}{3ex} 
  $\bar{\partial}_d$& =  & 
  $ \begin{pmatrix}\tilde{\partial}^+ &0 \cr 0& \tilde{\partial}^-  \end{pmatrix}$ &
  %$\pbd$
    &  \\
 \hline  
 \hline
 \rule{0pt}{3ex}  
% 
${\underline{\partial}}^\pm $ & =  & $ \nabla \pm j\, k\,$  &
%$\helmp \helmm$
  &  \\
\rule{0pt}{3ex} 
${\underline{\tilde{\partial}}}^\pm$  & = & $\tnabla \pm j\, k \, \sigma_0 $  &
%$\helmpp \helmmp$
  &  \\
\rule{0pt}{3ex} 
${\underline{\bar{\partial}}}^\pm$   & =  & $\begin{pmatrix}
 \pm j\, k\, \sigma_0 & \tnabla \cr 
\tnabla & \pm j\, k\, \sigma_0 \end{pmatrix}$ &
%$\helmpd \helmmd$
  &  \\
\rule{0pt}{3ex} 
 ${\underline{\bar{\partial}}}_d$   & = &  $\begin{pmatrix}
{\underline{\tilde{\partial}}}^+ & 0\cr 
0 & {\underline{\tilde{\partial}}}^- \end{pmatrix}$
&  
%$\pbfd$
&  \\
\hline
\hline
\rule{0pt}{3ex} 
$\Box^2$      & = & $\wavep \wavem$ =  $\nabla^2 - \frac{1}{v^2}\partial_t^2$ & &   \\
$\partial^2$ & = & $\helmp \helmm$ = $\nabla^2 +\, k^2$&  & \\
\hline
\hline
\end{tabular}
\end{table}


\end{frame}



%=========================================================================


%=========================================================================
\section{Field Potentials: GA approach}
%=========================================================================


\begin{frame}[fragile]{Field Potentials}


%
It is assumed that the field $\cal{F}$ can be recovered from a scalar potential $\phi$ and a vector potential $\BA$, using the following expression:
\be
{\cal{F}} = \left(\nabla - \frac{1}{v}\partial_t\right)\left( - \phi + v \, \BA \right) \label{Fpassage:40} \, 
\ee
\pause
or, equivalently, by introducing the multivector ${\cal A}$ defined as
\be
{\cal A} = - \phi + v \, \BA
\ee
we can write
\be
{\cal{F}} = \wavem {\cal A}
\ee


\end{frame}



%=========================================================================
\begin{frame}[fragile]{Potential equation}
\pause
When (\ref{Fpassage:40}) is inserted into (\ref{Fpassage:30a}) we recover the following equation:
\be
 \left(\nabla^2 - \frac{1}{v^2}\partial_t^2 \right)\left( - \phi + v \, \BA \right)  = \frac{\rho}{\epsilon} - \eta  \, \BJ\label{Fpassage:40a} \, 
\ee
which may be synthetically expressed as
\be
\Box{\cal A} = \cal{J} \, .
\ee

\pause
By separating the scalar and the vector part, we recover respectively the scalar and vector wave equations
\bea
\left(\nabla^2 - \frac{1}{v^2}\partial_t^2 \right) \phi & = & - \frac{\rho}{\epsilon} \label{scalwave} \\
\left(\nabla^2 - \frac{1}{v^2}\partial_t^2 \right) \BA & = & - \mu \BJ \label{vectwave} \,.
\eea
%

\end{frame}

%=========================================================================
\begin{frame}[fragile]{Potential equations}

Finally, by equating (\ref{Fpassage:40}) and (\ref{Fpassage:20}) we get:
\bea
{\cal{F}} & = & \BE + \eta \, \hat{\BH}  = \wavem {\cal A} \nonumber\\
& = & \left(\nabla - \frac{1}{v}\partial_t\right)\left( - \phi + v \, \BA \right) \nonumber \\
& = & \frac{1}{v} \partial_t \phi + v \nabla \cdot \BA - \nabla \wedge \phi - \partial_t \BA + v \, \nabla \wedge \BA 
 \,.
\eea
Considering the scalar part one gets
\bea
0 & = & \partial_t \phi + v^2 \, \nabla \cdot \BA \label{glorenz}  
%\nonumber
\\
\BE & = & - \nabla \wedge \phi - \partial_t \BA  \label{EfromPotA}
%\nonumber
\\
\eta \, \hat{\BH} & = & v \, \nabla \wedge \BA \label{HfromPotA}
\eea
\end{frame}


%=========================================================================


%=========================================================================
\begin{frame}[fragile]{Lorenz gauge }
We have obtained
\bea
0 & = & \partial_t \phi + v \, \nabla \cdot \BA \label{glorenz}
\eea
This is the Lorenz gauge and has been obtained simply by equating the grades.

\pause
To summarize we have expressed Maxwell's equations as
\be
\partial^+ \cal{F}  = \cal{J} \, ,
\ee
and we have then expressed the field in terms of the potential as
\be
{\cal{F}} = \wavem {\cal A}
\ee
obtaining the following equation
\be
 \cal{J} = \wavep \cal{F} = \wavep \wavem {\cal A} = \Box{\cal A}
\ee

\end{frame}

%=========================================================================
\section{Potentials in spinor form in time--domain}
%=========================================================================
\begin{frame}[fragile]{Potentials in spinor form in time--domain}

%
We start by rewriting (\ref{EfromPotA}) as 
\be
\BE  =  - \nabla \wedge \phi - \partial_t \BA = - \nabla  \phi - \partial_0 v \, \BA 
\ee
where we have used (\ref{glorenz}) and denoted with $\partial_0$
\be
\partial_0 = \frac{1}{v} \partial_t \, .
\ee
%
It is also convenient to express (\ref{HfromPotA}) in a different form noting that
\be
 \nabla \wedge \BA =  \nabla  \BA  - \nabla \cdot \BA = \nabla \, \BA  + \frac{1}{v} \partial_0 \phi
\ee
%
and therefore
\be
i \, \eta \, {\BH} = v \, \nabla \wedge \BA = v \, \nabla  \BA + \partial_0  \phi \,.
\ee
%


\end{frame}


%=========================================================================
\begin{frame}[fragile]{Potentials in terms of Dirac matrices}
So far we have therefore expressed the field in terms of the potentials, taking into account Lorenz condition, as
\bea
\BE & = & \partial_0 \left( - v \BA \right) + \nabla  \left( - \phi \right)\nonumber \\
i \, \eta \, {\BH} & = & -\nabla \left( -v \BA \right) - \partial_0 \left( - \phi \right) 
\label{symmpot_time}
\eea
Is apparent that (\ref{symmpot_time}) exhibit a symmetry suitable for expressing it by means of Dirac matrices.
Let us introduce the following notation:
\bea
\Be & = & 
\begin{pmatrix}  {E}_{z}\cr i\,{E}_{y}+{E}_{x} \end{pmatrix} \nonumber \\
\Bh & = &  
\begin{pmatrix}   \eta\,i\,{H}_{z}\cr \eta\,\left( i\,{H}_{x}-{H}_{y}\right)\end{pmatrix} 
\label{ehvect2t}
\eea


\end{frame}

%=========================================================================
\begin{frame}[fragile]{Field from Potentials in matrix form}

By using  (\ref{ehvect2t}) we have
\bea
 \begin{pmatrix}\Be\cr \Bh \end{pmatrix} & = &
\begin{pmatrix}
\partial_0 \sigma_0 & \tnabla \cr 
- \tnabla & - \partial_0, \sigma_0 \end{pmatrix} \,  \begin{pmatrix} - v\tA \cr -\tphi  \end{pmatrix} \nonumber \\
& = & -
\begin{pmatrix}
\partial_0 \sigma_0 & \tnabla \cr 
- \tnabla & - \partial_0, \sigma_0 \end{pmatrix} \,  \begin{pmatrix}  v\tA \cr \tphi  \end{pmatrix}
\label{MaxDirac4t}
\eea

with the position
\bea
\tA &=&  \begin{pmatrix}A_z \cr A_x + i \, A_y \end{pmatrix} \nonumber \\
\tphi &=&  \begin{pmatrix} {\phi} \cr 0 \end{pmatrix} \, .
\eea
%

\end{frame}


%=========================================================================

%=========================================================================
\begin{frame}[fragile]{Potential quadrivector}
By introducing the quadrivector for the potential
\bea
\barA & = & \begin{pmatrix}v {A}_{z}\cr v\left( i\,{A}_{y}+{A}_{x}\right) \cr \phi \cr 0 \end{pmatrix} 
\eea
%
and noting that the matrix appearing in (\ref{MaxDirac4t}) is the same appearing in Maxwell's equations we have
%
\be
\barF = - {\slashed \partial} \barA
\ee

\end{frame}




%=========================================================================
\begin{frame}[shrink=20]{Expanded version of (\ref{MaxDirac4t})}

The expanded version of (\ref{MaxDirac4t}), in rectangular coordinates, is:
%\small
\be
\begin{pmatrix}{E}_{z}\cr i\,{E}_{y}+{E}_{x}\cr \eta\,i\,{H}_{z}\cr \eta\,\left( i\,{H}_{x}-{H}_{y}\right) \end{pmatrix} = -
\begin{pmatrix} \partial_0 & 0 & {\partial}_{z} & {\partial}_{x}-i\,{\partial}_{y}\cr 
0 & \partial_0 & i\,{\partial}_{y}+{\partial}_{x} & -{\partial}_{z}\cr
- {\partial}_{z} & -{\partial}_{x}+i\,{\partial}_{y} &- \partial_0 & 0\cr 
- i\,{\partial}_{y}-{\partial}_{x} & {\partial}_{z} & 0 & - \partial_0\end{pmatrix}
%
\begin{pmatrix} v{A}_{z}\cr v \left( i\,{A}_{y}+{A}_{x}\right) \cr \phi \cr 0 \end{pmatrix} \nonumber
\label{MaxDirac50}
\ee
\normalsize

It is thus noted that we can write a single equation containing the sources the field and the potential as
\be
-\Box^2 \barA = - {\slashed \partial}^2 \barA = {\slashed \partial} \barF = - \eta \barJ
\ee


\end{frame}

%
%%=========================================================================
%\begin{frame}[fragile]{}
%
%
%\end{frame}
%
%%=========================================================================
%\begin{frame}[fragile]{}
%
%
%\end{frame}












\section{Frequency--domain}


%=========================================================================
\begin{frame}[fragile]{Time--domain Maxwell's equations}

Time--domain Maxwell's equations are commonly expressed as:
\bea 
%\label{Maxwell--time}
\nabla \times \BE & = & - \frac{\partial \, \BB}{\partial t} \label{curlE} \\
\nabla \times \BH & = &  \frac{\partial \, \BD}{\partial t} + \BJ \label{curlH} \\
\nabla \cdot \BD & = & \rho \label{divD} \\
\nabla \cdot \BH & = & 0 \label{divH} 
\eea
%

\end{frame}


%=========================================================================
\begin{frame}[fragile]{Maxwell's equations: frequency domain}

In the following, we make use of equivalence theorems which introduce
magnetic current density,
${\BM}({\Br})$, and magnetic charge distributions,
$\rho_{m}({\Br})$. These quantities, although not
physically present, help in the solution of several boundary
value problems. When considering also magnetic
currents and charges, the frequency--domain Maxwell's equations become
%
\begin{subequations}
\begin{align}
 \nabla \times {\BE}({\Br})  & =    - j\omega {\BB}({\Br})
- {\BM}({\Br})\,,
  \label{fFaraday}
\\[1mm]
 \nabla \times {\BH}({\Br})  & =     j\omega {\BD}({\Br})
+ {\BJ}({\Br})\,,
  \label{fAmpere}
\\[1mm]
  \nabla \cdot {\BD}({\Br})  & =     \rho_e({\Br})\,,
  \label{fGausse}
\\[1mm]
  \nabla \cdot {\BB}({\Br})  & =   - \rho_{m}({\Br})\,.
  \label{fGaussm}
\end{align}
\end{subequations}
 %

\end{frame}




%=========================================================================
\begin{frame}[fragile]{Frequency--domain Maxwell's Equation in compact form}
% \section{Frequency--domain Maxwell's Equation in compact form}

 In the following we initially consider only electric sources and an homogeneous region of space of constant  permittivity and permeability, so that we can use $\BB = \mu \BH$ and $\BD = \epsilon \BE$.
 
We multiply both sides of  (\ref{fFaraday}) times $i$ and applying ($\nabla \wedge \BE  =  i \, \nabla \cross \BE$) obtaining:
\be
\nabla \wedge \BE = - j \omega \mu   i  \,\BH \, . \label{epassage:15}
%+ \frac{\rho}{\epsilon}
\ee
By summing together (\ref{epassage:15}) and the divergence equation one gets
\be
\nabla \,  \BE = - j \omega \mu  \,  i  \,\BH  
+ \frac{\rho}{\epsilon} \label{epassage:25} \, .
\ee
One may rewrite the above eq. as
\be
\nabla \,  \BE = - j \omega \mu  \,   \,\hat{\BH}  
+ \frac{\rho}{\epsilon} \label{epassage:27} \, .
\ee



\end{frame}


%=========================================================================

%=========================================================================
\begin{frame}[fragile]{Frequency--domain Maxwell's Equation in compact form}
We can now consider (\ref{fAmpere}) multiply by $i \, i = -1$ and summing with (\ref{divH}) obtaining
\be
\nabla \,  \hat{\BH }= - j \omega \epsilon     \,\BE  
- \BJ \label{Hpassage:25} \, .
\ee
%

Equations (\ref{epassage:27}) and (\ref{Hpassage:25}) can be expressed as
\bea 
%\label{Maxwell--time}
\nabla \, \BE & = & - j \, k \, \eta \, \hat{\BH}  + \frac{\rho}{\epsilon} \label{nablaEf} \\
\nabla  \eta \, \hat{\BH} & = & -   j \, {k}  \, {\BE}  - \eta  \, \BJ \label{nablaHf} 
\eea
%
%The multivector $F$, composed by a vector  and a bivector part, is now introduced with the following definition:
%\be
%F = \BE + \eta \, \hat{\BH} \label{Fpassage:20f} \, .
%\ee

\end{frame}





%=========================================================================
\begin{frame}[fragile]{Frequency--domain Maxwell's Equation: GA}
By summing together (\ref{nablaEf}) and (\ref{nablaHf}) and using (\ref{Fpassage:20}), the well known results that allows to express the four Maxwells' equation as a single one in the frequency--domain is recovered:
\be
\left(\nabla +  j\, k\,\right) \cal{F} =  \frac{\rho}{\epsilon} - \eta  \, \BJ \label{Fpassage:30f} \, 
\ee
or, synthetically,
\be
\helmp \cal{F} = \cal{J}
\ee



\end{frame}

\section{Frequency--domain potentials: GA approach}

%=========================================================================
\begin{frame}[fragile]{Frequency--domain potentials: GA approach}

%
It is assumed that the field $\cal{F}$ can be recovered from a scalar potential $\phi$ and a vector potential $\BA$, using the following expression:
\be
{\cal{F}} = \left(\nabla - j\, k\, \right)\left( v \, \BA  - \phi \right) \label{Fpassage:40f} \, .
\ee
When (\ref{Fpassage:40f}) is inserted into (\ref{Fpassage:30f}) we recover the following equation:
\be
 \left(\nabla^2 + k^2 \right)\left( v \, \BA  - \phi \right)  = \frac{\rho}{\epsilon} - \eta  \, \BJ\label{Fpassage:43f} \, .
\ee
and, by separating the scalar and the vector part, we recover respectively the scalar and vector wave equations
\bea
\left(\nabla^2 +k^2 \right) \phi & = & - \frac{\rho}{\epsilon} \label{scalwavef} \\
\left(\nabla^2 + k^2 \right) \BA & = & - \mu \BJ \label{vectwavef} \,.
\eea
%
\end{frame}


%=========================================================================

%=========================================================================
\begin{frame}[fragile]{The grade structure}
Finally, by equating (\ref{Fpassage:40f}) and (\ref{Fpassage:20}) we get:
\bea
\cal{F} & = & \BE + \eta \, \hat{\BH} \\
& = & \left(\nabla - j \, k \,\right)\left( v \, \BA  - \phi \right) \\
& = & j \, k \, \phi + v \nabla \cdot \BA - \nabla \wedge \phi - j \, k \, v \, \BA + v \, \nabla \wedge \BA 
 \,.
\eea
Considering the scalar, vector and bivectors  parts one gets
\bea
0 & = & j \, k \, \phi + v \, \nabla \cdot \BA \label{glorenz}  \\
\BE & = & - \nabla \wedge \phi - j \, k \, v \, \BA \\
\eta \, \hat{\BH} & = & v \, \nabla \wedge \BA
\eea

\end{frame}




%=========================================================================
\begin{frame}[fragile]{Potentials: GA}

To summarize:
\bea
\left(\nabla^2 +k^2 \right) \phi & = & - \frac{\rho}{\epsilon} 
%\label{scalwavef} 
\\
\left(\nabla^2 + k^2 \right) \BA & = & - \mu \BJ 
%\label{vectwavef} 
\,.
\eea
\pause
and 
\bea
\phi & = & -\frac{v\, \nabla \cdot \BA}{j\, k} \label{glorenzf} \\
\BE & = &- j \, k \, v \, \BA - \nabla \phi \\
 \hat{\BH}& = & \frac{1}{\mu} \nabla \wedge \BA 
 \eea
 %
Note that the Lorenz condition is not imposed, but it is derived directly from the assumption in (\ref{Fpassage:40f}). 



\end{frame}


\section{Potentials in spinor form}
%=========================================================================
\begin{frame}[fragile]{Potentials in spinor form}

We have seen that in frequency domain the potentials are given by:
\bea
\phi & = & -\frac{v\, \nabla \cdot \BA}{j\, k} \label{glorenzf} \\
\BE & = &- j \, k \, v \, \BA - \nabla \phi \\
 i \, {\BH}& = & \frac{1}{\mu} \nabla \wedge \BA  \label{Hpot}\, .
 \eea
 %
It is convenient to express (\ref{Hpot}) in a different form noting that
\be
 \nabla \wedge \BA =  \nabla  \BA  - \nabla \cdot \BA = \nabla \, \BA  +j \, \frac{k}{v} \phi
\ee
and therefore
\be
i \, \eta \, {\BH} = v \, \nabla  \BA + j\, k \, \phi \,.
\ee
%

\end{frame}

%=========================================================================
\begin{frame}[fragile]{Symmetric form}

So far we have therefore expressed the field in terms of the potentials, taking into account Lorenz condition, as
\bea
\BE & = & v \left(-j\, k  \, \BA - \frac{1}{v}   \nabla \phi \right) \nonumber \\
i \, \eta \, {\BH} & = & v \left[ \nabla  \BA  + \left(- j\,k \right)  \left( - \frac{\phi}{v} \right) \right]
\label{symmpot}
\eea
Is apparent that (\ref{symmpot}) exhibit a symmetry suitable for expressing it by means of Dirac matrices.
Let us introduce the following notation:
\bea
\Be & = & 
\begin{pmatrix}  {E}_{z}\cr i\,{E}_{y}+{E}_{x} \end{pmatrix} \nonumber \\
\Bh & = &  
\begin{pmatrix}   \eta\,i\,{H}_{z}\cr \eta\,\left( i\,{H}_{x}-{H}_{y}\right)\end{pmatrix} 
\label{ehvect2}
\eea

\end{frame}


%=========================================================================

%=========================================================================
\begin{frame}[fragile]{Potentials as quadrivectors}
By using  (\ref{ehvect2}) we have
\be
 \begin{pmatrix}\Be\cr \Bh \end{pmatrix} = v
\begin{pmatrix}
-j \, k\, \sigma_0 & \tnabla \cr 
\tnabla & -j \, k\, \sigma_0 \end{pmatrix} \,  \begin{pmatrix} \tA \cr \tphi  \end{pmatrix} .
\label{MaxDirac4}
\ee
with the position
\bea
\tA &=&  \begin{pmatrix}A_z \cr A_x + i \, A_y \end{pmatrix} \nonumber \\
\tphi &=&  \begin{pmatrix} - \frac{\phi}{v} \cr 0 \end{pmatrix} \, .
\eea

\end{frame}


%=========================================================================


%=========================================================================
\begin{frame}[shrink=20]{Expanded version}

The expanded version of (\ref{MaxDirac4}), in rectangular coordinates, is:

%\small
\be
\begin{pmatrix}{E}_{z}\cr i\,{E}_{y}+{E}_{x}\cr \eta\,i\,{H}_{z}\cr \eta\,\left( i\,{H}_{x}-{H}_{y}\right) \end{pmatrix} = v
\begin{pmatrix} - j k & 0 & {\partial}_{z} & {\partial}_{x}-i\,{\partial}_{y}\cr 
0 & - j k & i\,{\partial}_{y}+{\partial}_{x} & -{\partial}_{z}\cr
 {\partial}_{z} & {\partial}_{x}-i\,{\partial}_{y} &- j k & 0\cr 
 i\,{\partial}_{y}+{\partial}_{x} & -{\partial}_{z} & 0 & - j k \end{pmatrix}
%
\begin{pmatrix}{A}_{z}\cr i\,{A}_{y}+{A}_{x}\cr  - \frac{\phi}{v} \cr 0 \end{pmatrix}
\label{MaxDirac50}
\ee

\end{frame}


%=========================================================================
\begin{frame}[shrink=20]{Magnetic sources }
Sometimes it is convenient to consider also magnetic sources 
%summary 
and, in a piecewise constant medium, we obtain
 the following local form of Maxwell's equations:
\begin{subequations}
\begin{align}
\nabla \cdot \BE &=  \frac{\rho_e}{\epsilon}  \, , &&\text{grade 0} \label{GausseGA} \\
\nabla \cdot \left(i \eta \BH\right) &= - \frac{1}{v}\,  \partial_t \BE - \eta \,\BJ  \, , &&\text{grade 1} \label{AmpereGA} \\
\nabla \wedge \BE &= - \frac{1}{v}\,  \partial_t \, \left(i \eta \BH\right) \, -i \, \BM , &&\text{grade 2}  \label{FaradayGA} \\
\nabla \wedge \left(i \eta \BH\right) &=  - i \, v \, \rho_m \, , &&\text{grade 3} \label{GaussmGA}
\end{align}
\end{subequations}

The spinor containing the excitations is therefore given by:
\be
\begin{pmatrix} 
- \eta {J}_{z} - i \, v \, \rho_m \cr
- \eta \left( i\,{J}_{y}+{J}_{x}\right)\cr 
 \frac{\rho_e}{\epsilon} - i\,{M}_{z}\cr
 - i\,\left( i\,{M}_{y}+{M}_{x}\right) \end{pmatrix}
 =
 \begin{pmatrix}  
\BJ_e \cr\BJ_m
  \end{pmatrix}
\ee
with the vector $\BJ_e$ containing the first two rows and the vector $\BJ_m$ containing the third  and fourth rows.


\end{frame}

%=========================================================================
\begin{frame}[shrink=20]{Potential for magnetic sources only}
By application of superposition it is convenient to consider the case when only  magnetic sources are present. In  frequency domain we have 
%\begin{subequations}
%\begin{align}
\bea
\nabla \cdot \BE &=&  0  \nonumber\\
i \eta  \nabla \cdot \left(\BH\right) &=& -i v \rho_m    \nonumber \\
\nabla \wedge \BE &=& - jk \left(i \eta \BH\right) \, -i \, \BM \nonumber \\
i \eta \nabla  \left(  \BH\right) &=&  - j \, k \, \BE 
\eea
%\end{align}
%\end{subequations}
which can be condensed as
%
\bea 
\nabla \BE &=& - jk \left(i \eta \BH\right) \, -i \, \BM  \nonumber\\
 \nabla  \left(i \eta \BH\right) &=&  - j \, k \, \BE -i v \rho_m
\eea
or, by using the multivector form, yields:
\be
\left(\nabla + j\,k\right) {\cal{F}} =  -i \left(v \rho_m  + \BM\right) \label{FMpassage:30FD} \, .
\ee
%

\end{frame}


%=========================================================================
\begin{frame}[shrink=20]{Potentials for magnetic sources only}
It is assumed that the field $\cal{F}$ can be recovered from a scalar potential $\psi$ and a vector potential $\BF$, using the following expression:
\be
{\cal{F}} = i \left(\nabla - j\, k\, \right)\left( v \, \BF  + \psi \right) \label{FMpassage:40f} \, .
\ee
When (\ref{FMpassage:40f}) is inserted into (\ref{FMpassage:30FD}) we recover the following equation:
\be
 \left(\nabla^2 + k^2 \right)\left( v \, \BF  + \psi \right)  = - \left(v \rho_m  + \BM\right) \label{FMpassage:43f} \, .
\ee
and, by separating the scalar and the vector part, we recover, respectively, the scalar and vector wave equations:
\bea
\left(\nabla^2 +k^2 \right) \psi & = & - v\, \rho_m  \label{Mscalwavef} \nonumber\\
\left(\nabla^2 + k^2 \right) \BF & = & - \frac{1}{v}\, \BM  \label{Mvectwavef} \,.
\eea
%
Finally,  we get:
\bea
\cal{F} & = & \BE + i \, \eta \, {\BH} \nonumber \\
& = &i  \left(\nabla - j \, k \,\right)\left( v \, \BF  + \psi \right) \nonumber \\
& = & i \left( -j \, k \, \psi + v\,  \nabla \cdot \BF + \nabla  \psi - j \, k \, v \, \BF + v \, \nabla \wedge \BF  \right)
 \,.
\eea

\end{frame}


%=========================================================================


%=========================================================================
\begin{frame}[fragile]{Grade separation}
Considering the scalar, vector and bivectors  parts one gets
\bea
0 & = & - j \, k \, \psi + v \, \nabla \cdot \BF \label{Mglorenz}  \nonumber \\
\BE & = & i \left( v \, \nabla \wedge \BF  \right)  \nonumber \\
i \, \eta \, {\BH} & = & i \left( \nabla  \psi - j \, k \, v \, \BF \right)
\eea
and therefore:
\bea
\psi & = & \frac{v\, \nabla \cdot \BF}{j\, k} \label{Mglorenzf} \nonumber \\
\BE & = &i \left( v \, \nabla  \BF - jk \psi  \right) \nonumber \\
 i \eta \, {\BH}& = & i\left( \nabla \psi  - j \, k \, v \, \BF \right)\label{MHpot}\, .
 \eea
 %
The Lorenz condition (\ref{Mglorenzf}) is  derived directly from the assumption in (\ref{FMpassage:40f}). 


\end{frame}

\section{Potentials for magnetic sources in spinor form}

%=========================================================================
\begin{frame}[fragile]{Potentials for magnetic sources in spinor form}
%
It is apparent that (\ref{MHpot}) exhibit a symmetry suitable for expressing it by means of Dirac matrices.
By using  (\ref{ehvect2}) we have
%
\be
 \begin{pmatrix}\Be\cr \Bh \end{pmatrix} = i\,v
\begin{pmatrix}
 j \, k\, \sigma_0 & \tnabla \cr 
- \tnabla & -j \, k\, \sigma_0 \end{pmatrix} \,  \begin{pmatrix}  -\frac{1}{v}\tpsi \cr \tF  \end{pmatrix} .
\label{MaxDirac60}
\ee
with the position
\bea
\tF &=&  \begin{pmatrix}F_z \cr F_x + i \, F_y \end{pmatrix} \nonumber \\
\tpsi &=&  \begin{pmatrix} {\psi} \cr 0 \end{pmatrix} \, .
\eea

\end{frame}

%=========================================================================
\begin{frame}[fragile]{}


\end{frame}


%=========================================================================

%%=========================================================================
%\begin{frame}[fragile]{}
%
%
%\end{frame}
%
%
%%=========================================================================
%\begin{frame}[fragile]{}
%
%
%\end{frame}
%
%%=========================================================================
%\begin{frame}[fragile]{}
%
%
%\end{frame}


%=========================================================================


%
%%=========================================================================
%
%%=========================================================================
%\begin{frame}[fragile]{}
%
%
%\end{frame}


%=========================================================================


%%=========================================================================
%\begin{frame}[fragile]{}
%
%
%\end{frame}
%
%
%%=========================================================================
%\begin{frame}[fragile]{}
%
%
%\end{frame}
%
%%=========================================================================
%\begin{frame}[fragile]{}
%
%
%\end{frame}


%=========================================================================






\end{document}
