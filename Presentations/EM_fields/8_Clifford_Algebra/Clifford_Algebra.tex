\documentclass[10pt]{beamer}
%\documentclass[handout,10pt]{beamer}

\usetheme[progressbar=frametitle]{metropolis}

\usepackage{booktabs}
\usepackage[scale=2]{ccicons}


\usepackage{amsmath}
\usepackage{pgfplots}
\usepgfplotslibrary{dateplot}

\usepackage{xspace}
\newcommand{\themename}{\textbf{\textsc{metropolis}}\xspace}

%\usepackage{placeins} %%%
\usepackage{subfig}
\usepackage{physics}
\usepackage{amssymb}

\usepackage[utf8]{inputenc}

\usepackage{tikz}
\usepackage{circuitikz}
\usepackage{siunitx}


\usepackage{latexsym}
\usepackage{mathtools}
\usepackage{slashed} % for the Feynman slash notation

\usepackage{listings}

\usepackage{balance}


% edited by Mauro 28-12-16
%
%% <local definitions>
\newcommand{\R}{\mathbb{R}}	
\newcommand{\C}{\mathbb{C}}
\newcommand{\HQ}{\mathbb{H}}
\newcommand{\N}{\mathbb{N}}
\newcommand{\be}{\begin{equation}}
\newcommand{\ee}{\end{equation}}	
\newcommand{\bea}{\begin{eqnarray}}
\newcommand{\eea}{\end{eqnarray}}	
\newcommand{\Pin}{\mathrm{Pin}}	
\newcommand{\Spin}{\mathrm{Spin}}
\renewcommand{\O}{\mathrm{O}}
\newcommand{\SO}{\mathrm{SO}}
\renewcommand{\eqref}[1]{(\ref{#1})}
\newcommand{\cl}[1]{\ensuremath{Cl(#1)}} % #1 stands for the values p,q. $\cl{p,q}$ produces 'Cl(p,q)'.
\newcommand{\gvec}[1]{\ensuremath{\mbox{\textbf{\textit{#1}}}}}
\newcommand{\vect}[1]{\ensuremath{\mbox{\textbf{\textit{#1}}}}}
%% </local definitions>

\newcommand{\Ba}[0]{\mathbf{a}}
\newcommand{\Bb}[0]{\mathbf{b}}
\newcommand{\Bc}[0]{\mathbf{c}}
\newcommand{\Bd}[0]{\mathbf{d}}
\newcommand{\Be}[0]{\mathbf{e}}
\newcommand{\Bf}[0]{\mathbf{f}}
\newcommand{\Bg}[0]{\mathbf{g}}
\newcommand{\Bh}[0]{\mathbf{h}}
\newcommand{\Bi}[0]{\mathbf{i}}
\newcommand{\Bj}[0]{\mathbf{j}}
\newcommand{\Bk}[0]{\mathbf{k}}
\newcommand{\Bl}[0]{\mathbf{l}}
\newcommand{\Bm}[0]{\mathbf{m}}
\newcommand{\Bn}[0]{\mathbf{n}}
\newcommand{\Bo}[0]{\mathbf{o}}
\newcommand{\Bp}[0]{\mathbf{p}}
\newcommand{\Bq}[0]{\mathbf{q}}
\newcommand{\Br}[0]{\mathbf{r}}
\newcommand{\Bs}[0]{\mathbf{s}}
\newcommand{\Bt}[0]{\mathbf{t}}
\newcommand{\Bu}[0]{\mathbf{u}}
\newcommand{\Bv}[0]{\mathbf{v}}
\newcommand{\Bw}[0]{\mathbf{w}}
\newcommand{\Bx}[0]{\mathbf{x}}
\newcommand{\By}[0]{\mathbf{y}}
\newcommand{\Bz}[0]{\mathbf{z}}
\newcommand{\BA}[0]{\mathbf{A}}
\newcommand{\BB}[0]{\mathbf{B}}
\newcommand{\BC}[0]{\mathbf{C}}
\newcommand{\BD}[0]{\mathbf{D}}
\newcommand{\BE}[0]{\mathbf{E}}
\newcommand{\BF}[0]{\mathbf{F}}
\newcommand{\BG}[0]{\mathbf{G}}
\newcommand{\BH}[0]{\mathbf{H}}
\newcommand{\BI}[0]{\mathbf{I}}
\newcommand{\BJ}[0]{\mathbf{J}}
\newcommand{\BK}[0]{\mathbf{K}}
\newcommand{\BL}[0]{\mathbf{L}}
\newcommand{\BM}[0]{\mathbf{M}}
\newcommand{\BN}[0]{\mathbf{N}}
\newcommand{\BO}[0]{\mathbf{O}}
\newcommand{\BP}[0]{\mathbf{P}}
\newcommand{\BQ}[0]{\mathbf{Q}}
\newcommand{\BR}[0]{\mathbf{R}}
\newcommand{\BS}[0]{\mathbf{S}}
\newcommand{\BT}[0]{\mathbf{T}}
\newcommand{\BU}[0]{\mathbf{U}}
\newcommand{\BV}[0]{\mathbf{V}}
\newcommand{\BW}[0]{\mathbf{W}}
\newcommand{\BX}[0]{\mathbf{X}}
\newcommand{\BY}[0]{\mathbf{Y}}
\newcommand{\BZ}[0]{\mathbf{Z}}

\newcommand{\ta}[0]{\tilde{a}}
\newcommand{\tb}[0]{\tilde{b}}
\newcommand{\tc}[0]{\tilde{c}}
\newcommand{\td}[0]{\tilde{d}}

\newcommand{\hA}[0]{\hat{A}}
\newcommand{\hB}[0]{\hat{B}}
\newcommand{\hH}[0]{\hat{H}}

\newcommand{\tA}[0]{\tilde{A}}
\newcommand{\tB}[0]{\tilde{B}}
\newcommand{\tF}[0]{\tilde{F}}
\newcommand{\tE}[0]{\tilde{E}}
\newcommand{\tH}[0]{\tilde{H}}

% spinors definition
\newcommand{\barJ}[0]{\bar{J}}
\newcommand{\barF}[0]{\bar{F}}
\newcommand{\barP}[0]{\bar{P}}
\newcommand{\barW}[0]{\bar{W}}



\newcommand{\tnabla}[0]{\tilde{\nabla}}
\newcommand{\tphi}[0]{\tilde{\phi}}
\newcommand{\tpsi}[0]{\tilde{\psi}}

%
\newcommand{\wavep}[0]{\partial^+}
\newcommand{\wavem}[0]{\partial^-}

\newcommand{\wavepp}[0]{\tilde{\partial}^+}
\newcommand{\wavemp}[0]{\tilde{\partial}^-}

\newcommand{\wavepd}[0]{\bar{\partial}^+}
\newcommand{\wavemd}[0]{\bar{\partial}^-}

\newcommand{\pbd}[0]{\bar{\partial}_d}

% frequency

\newcommand{\helmp}[0]{{\underline{\partial}}^+}
\newcommand{\helmm}[0]{{\underline{\partial}}^-}

\newcommand{\helmpp}[0]{{\underline{\tilde{\partial}}}^+}
\newcommand{\helmmp}[0]{{\underline{\tilde{\partial}}}^-}

\newcommand{\helmpd}[0]{{\underline{\bar{\partial}}}^+}
\newcommand{\helmmd}[0]{{\underline{\bar{\partial}}}^-}

\newcommand{\pbfd}[0]{{\underline{\bar{\partial}}}_d}




\def \figname {Figure}
\def \emode {E }
\def \hmode {H }
\def \temode {TE }
\def \tmmode {TM }
\def \temoden {TE${}_n$ }
\def \tmmoden {TM${}_n$ }
\def \temodemn {TE${}_{mn}$ }
\def \tmmodemn {TM${}_{mn}$ }



\newcommand{\iGA}{{i}}
\newcommand{\conjg}[1] {\ensuremath{#1}^*}

\setbeamertemplate{bibliography item}{[\theenumiv]}


\title{Clifford Algebra}

\date{}


\author{ Mauro Mongiardo$^1$}

\institute{ $^1$ Department of Engineering, University of Perugia, Perugia, Italy.
}

%
\titlegraphic{\hfill\includegraphics[height=1.5cm]{logo}}


\begin{document}

\maketitle

\begin{frame}{Table of contents}
  \setbeamertemplate{section in toc}[sections numbered]
  \tableofcontents[hideallsubsections]
\end{frame}

%=========================================================================
%\section{Introduction}
 \section{Clifford Algebra Definition}
%=========================================================================

%Definition of Clifford's or geometric algebra (GA) 
%

%%=========================================================================
%\begin{frame}[fragile]{}
%This non-commutative product and the additional axiom of \textit{associativity} generate the $2^n$-dimensional Clifford geometric algebra $Cl(p,q,r)$.
%The definition of Clifford's GA is fundamentally \textit{coordinate system independent}, i.e. \textit{coordinate free}.
%


%\end{frame}

%=========================================================================
\begin{frame}[fragile]{Clifford basis}
Let us first introduce $Cl_n$ in the following way.
We consider an \alert{orthonormal} basis $e_1, e_2, \ldots, e_n$ such that for $ j=1,\ldots,n$

\alert{
\be
e_j^2 = 1 
\ee
}
\pause
and for $i \ne j$

\alert{
\be
e_i \, e_j +  e_j \, e_i =0
\ee
}
anti--commutativity.
\pause

This basis is a Clifford basis of order $n$.

This is all we need for defining the Clifford algebra!
\end{frame}

%=========================================================================
\begin{frame}[fragile]{Three--dimensional case}

For the three-dimensional case let us introduce the correspondence (for $i=1,2,3$)

\be
e_i = \sigma_i 
\ee

with $\sigma_i$ denoting the Pauli matrices.

\pause

\alert{The Pauli matrices are a basis for the Clifford algebra in 3D}.

Therefore if you know how to operate with the Pauli matrices, you know the Clifford algebra of dimension 3!

\end{frame}

%=========================================================================
\begin{frame}[fragile]{A more general definition $Cl(n,m)$}

A more general definition is obtained by considering, in addition to the $n$ elements of the basis 
$e_1, e_2, \ldots, e_n$ other $m$ elements which \alert{squares to $-1$} 

\be
e_j^2 = -1 
\ee

for $j=n+1,\ldots,n+m$.

An even more general definition can include also elements that square to zero (but we will not use it).
\end{frame}

%=========================================================================
\begin{frame}[fragile]{Formal definition (*)}


Let $\{e_1, e_2, \ldots , e_p, e_{p+1}, \ldots , e_{p+q}, e_{p+q+1}, \ldots, e_n \}$, with $n=p+q+r$, $e_k^2=\varepsilon_k$, $\varepsilon_k = +1$ for $k=1, \ldots , p$, $\varepsilon_k = -1$ for $k=p+1, \ldots , p+q$, $\varepsilon_k = 0$ for $k=p+q+1, \ldots , n$,  be an \textit{orthonormal base} of the inner product vector space $\R^{p,q,r}$ with a geometric product according to the multiplication rules 
\be
  e_k e_l + e_l e_k = 2 \varepsilon_k \delta_{k,l}, 
  \qquad k,l = 1, \ldots n,
\label{eq:mrules}
\ee
where $\delta_{k,l}$ is the Kronecker symbol with $\delta_{k,l}= 1$ for $k=l$, and $\delta_{k,l}= 0$ for $k\neq l$. 


\end{frame}

\section{Clifford's GA in two dimensions $Cl(2,0)$}
%=========================================================================
\begin{frame}[fragile]{Clifford's GA in two dimensions $Cl(2,0)$}
%\subsection{Clifford's GA in two dimensions}

%In order to demonstrate how to compute with Clifford numbers, we begin with a low dimensional example. 

%\subsubsection{Example of $Cl(2,0)$}
A Euclidean plane 
%$\R^2=\R^{2,0}=\R^{2,0,0}$ 
is spanned by $e_1, e_2 $ with 
\begin{gather}
  e_1\cdot e_1 = e_2\cdot e_2 = 1, \quad e_1\cdot e_2 = 0.
\end{gather}
$\{e_1, e_2\}$ is an \textit{orthonormal} vector basis.
% of $\R^2$. 

\pause

Under Clifford's \textit{associative} geometric product we set
%%%%%%%%%%%%%%%%%%%%%%%%%%%
%% Reformatted equation %%%
%%%%%%%%%%%%%%%%%%%%%%%%%%%
\begin{gather}
  e_1^2=e_1e_1 := e_1\cdot e_1=1, 
  \nonumber \\ 
  e_2^2=e_2e_2 := e_2\cdot e_2=1, 
  \\
  \hspace*{-12mm}\text{and}\quad 
  (e_1+e_2)(e_1+e_2) 
  = e_1^2+e_2^2+e_1e_2 + e_2e_1 \\
  = 2 + e_1e_2 + e_2e_1
  := (e_1+e_2)\cdot(e_1+e_2) = 2.
  \nonumber 
\end{gather}
%%%%%%%%%%%%%%%%%%%%%%%%%%%
%% Reformatted equation %%%
%%%%%%%%%%%%%%%%%%%%%%%%%%%
\pause
and
\pause
\be 
  e_1e_2 + e_2e_1 = 0 
  \,\,\,\Leftrightarrow \,\,\,
  e_1e_2 = - e_2e_1,
\ee 



\end{frame}

%=========================================================================
\begin{frame}[fragile]{}
The geometric product of orthogonal vectors forms a new entity, called unit \alert{bivector} $e_{12}=e_1e_2$ by Grassmann, and is \alert{anti-symmetric}. 
\pause
General bivectors in $Cl(2,0)$ are e.g. $\beta e_{12}$.
%, $\forall \, \beta \in \R \setminus \{0\}$. 
For orthogonal vectors the geometric product equals Grassmann's anti-symmetric outer product (exterior product, symbol $\wedge$) 
%%%%%%%%%%%%%%%%%%%%%%%%%%%%
%%% Reformatted equation %%%
%%%%%%%%%%%%%%%%%%%%%%%%%%%%
\begin{align} 
  e_{12} &= e_1e_2 = e_1 \wedge e_2 
  \nonumber \\
  &= - e_2 \wedge e_1 = -e_2e_1 = -e_{21}.
\end{align} 
%%%%%%%%%%%%%%%%%%%%%%%%%%%%
%%% Reformatted equation %%%
%%%%%%%%%%%%%%%%%%%%%%%%%%%%
\pause
Using associativity, we can compute the products
\be   
  e_1 e_{12} = e_1 e_1 e_2 = e_1^2 e_2 = e_2, \quad
  e_2 e_{12} = - e_2 e_{21} = - e_1,
\ee 
which \alert{represent a mathematically \textit{positive} (anti-clockwise) $90^{\circ}$ \textit{rotation}}. 


\end{frame}

%=========================================================================
\begin{frame}[fragile]{}

The opposite order gives
\be 
  e_{12}e_1 = -e_{21}e_1 = -e_2, \quad e_{12}e_2 = e_1,
\ee 
\alert{which represents a mathematically \textit{negative} (clockwise) $90^{\circ}$ \textit{rotation}}. 

\pause

\alert{The bivector $e_{12}$ acts like a \textit{rotation operator}}, and we observe the general anti--commutation property
\be 
  a e_{12} = - e_{12} a, \quad \forall a=a_1e_1+a_2e_2 \in \R^2, \,\,\,a_1,a_2 \in \R.
\ee 
\pause

The square of the unit bivector is $-1$,
\be 
  e_{12}^2 = e_1 e_2 e_{12} = e_1 (-e_1) = -1,
\ee 
just like the imaginary unit $j$ of complex numbers $\C$. 



\end{frame}

%=========================================================================
\begin{frame}[fragile]{Multiplication table}

Table \ref{tb:Cl2mtable} is the complete multiplication table of the Clifford algebra $ Cl(2,0)$ with algebra basis elements $\{1, e_1, e_2, e_{12}\}$. 

\alert{The even subalgebra spanned by $\{1,e_{12}\}$ (closed under geometric multiplication), consisting of even grade scalars (0-vectors) and bivectors (2-vectors), is isomorphic to $\mathbb{C}$. }

\begin{table}
\caption{Multiplication table of plane Clifford algebra $Cl(2,0)$.}
\label{tb:Cl2mtable}
\begin{center}
\begin{tabular}{c|cccc}
       & 1 & $e_1$ & $e_2$ & $e_{12}$\\
 \hline
 1     & 1 & $e_1$ & $e_2$ & $e_{12}$\\
 $e_1$ & $e_1$ & 1 & $e_{12}$ & $e_{2}$\\
 $e_2$ & $e_2$ & $-e_{12}$ & 1 & $-e_{1}$\\
 $e_{12}$ & $e_{12}$ & $-e_{2}$ & $e_1$ & $-1$ 
\end{tabular}
\end{center}
\end{table}


\end{frame}

%=========================================================================
\begin{frame}[fragile]{}
%\subsubsection{Algebraic unification and vector inverse}

The general geometric product of two vectors $a,b \in \R^2$ 
%%%%%%%%%%%%%%%%%%%%%%%%%%%%
%%% Reformatted equation %%%
%%%%%%%%%%%%%%%%%%%%%%%%%%%%
\begin{gather}
  ab=(a_1e_1+a_2e_2)(b_1e_1+b_2e_2) \nonumber \\
  = a_1b_1 + a_2b_2 + (a_1b_2 - a_2b_1) e_{12} \\
  = \frac{1}{2}(ab+ba) + \frac{1}{2}(ab-ba)
  = a \cdot b + a\wedge b,
  \nonumber 
\end{gather} 
%%%%%%%%%%%%%%%%%%%%%%%%%%%%
%%% Reformatted equation %%%
%%%%%%%%%%%%%%%%%%%%%%%%%%%%
\pause
has therefore a scalar \textit{symmetric} inner product part
%%%%%%%%%%%%%%%%%%%%%%%%%%%%
%%% Reformatted equation %%%
%%%%%%%%%%%%%%%%%%%%%%%%%%%%
\begin{align} 
  \frac{1}{2}(ab+ba) 
  &= a \cdot b  
  =a_1b_1 + a_2b_2 
  \nonumber \\
  &= |a| |b| \cos \theta_{a,b},
\end{align} 
%%%%%%%%%%%%%%%%%%%%%%%%%%%%
%%% Reformatted equation %%%
%%%%%%%%%%%%%%%%%%%%%%%%%%%%
\pause
and a bi-vector \textit{skew-symmetric} outer product part
\be 
  \frac{1}{2}(ab-ba)=a\wedge b 
  =(a_1b_2 - a_2b_1) e_{12} = |a| |b| e_{12}\sin \theta_{a,b}. 
\ee

\end{frame}

%=========================================================================
\begin{frame}[fragile]{}
We observe that parallel vectors $(\theta_{a,b}=0)$ commute, $ab = a\cdot b = ba$, and orthogonal vectors $(\theta_{a,b}=90^{\circ})$ anti-commute, $ab = a \wedge b = -ba$.  \pause
The outer product part $a\wedge b$ represents the \textit{oriented area} of the parallelogram spanned by the vectors $a,b$ in the plane of $\R^2$, with oriented magnitude 
\be 
\det(a,b) = |a| |b| \sin \theta_{a,b} = (a\wedge b) e_{12}^{-1},
\ee
where $e_{12}^{-1} = -e_{12}$, because $e_{12}^2=-1$.  

\end{frame}



%=========================================================================
\begin{frame}[fragile]{}


With the \textit{Euler} formula we can rewrite the geometric product as
%%%%%%%%%%%%%%%%%%%%%%%%%%%%
%%% Reformatted equation %%%
%%%%%%%%%%%%%%%%%%%%%%%%%%%%
\begin{align} 
  ab &= |a| |b| (\cos \theta_{a,b} + e_{12}\sin \theta_{a,b}) 
  \nonumber \\
  &= |a| |b| e^{\theta_{a,b} e_{12}} \,,
  \label{eq:Euler}
\end{align} 
%%%%%%%%%%%%%%%%%%%%%%%%%%%%
%%% Reformatted equation %%%
%%%%%%%%%%%%%%%%%%%%%%%%%%%%
again because $e_{12}^2=-1$. 
\pause

The geometric product of vectors is \textit{invertible} for all vectors with non-zero square 
$a^2 \neq 0$
\begin{gather}
  a^{-1} := a/a^2, \quad 
  a a^{-1} = a a/a^2 = 1, \nonumber \\
  a^{-1} a = \frac{a}{a^2}a = a^2/a^2 = 1. 
\end{gather} 
\pause

The inverse vector $a/a^2$ is a rescaled version (reflected at the unit circle) of the vector $a$. 

\pause
This invertibility leads to significant simplifications and ease in computations. 





\end{frame}

%=========================================================================
\begin{frame}[fragile]{Projection and Rejection}

\subsection{Geometric operations and transformations}

For example, the \textit{projection} of one vector $x\in\R^2$ onto another $a \in \R^2$ is
\be 
  x_{\parallel} = | x \,| \cos\theta_{a,x}\frac{a}{|a|} 
  = (x \cdot \frac{a}{|a|})\frac{a}{|a|}
  = (x \cdot a)\frac{a}{|a|^2} 
  = (x \cdot a)a^{-1}.
\ee 
The \textit{rejection} (perpendicular part) is
\begin{gather} 
  x_{\perp} = x-x_{\parallel}
  = x aa^{-1} - (x \cdot a)a^{-1} 
  \nonumber \\
  = (xa-x \cdot a)a^{-1}
  = (x\wedge a)a^{-1}.
\end{gather}



\end{frame}


%=========================================================================
\begin{frame}[fragile]{}

We can now use $x_{\parallel}, x_{\perp}$ to compute the reflection\footnote{Note that reflections at hyperplanes are nothing but the \textit{Householder transformations} of matrix analysis.} of $x=x_{\parallel}+ x_{\perp}$ at the line (hyperplane\footnote{A hyperplane of a $n$D space is a $(n-1)$D subspace, thus a hyperplane of $\R^2$, $n=2$, is a 1D ($2-1=1$) subspace, i.e. a line. Every hyperplane is characterized by a vector normal to the hyperplane.}) with normal vector $a$, which means to reverse $x_{\parallel}\rightarrow -x_{\parallel}$
\begin{gather}
  x' = -x_{\parallel} + x_{\perp} 
  = -a^{-1}a\,x_{\parallel} + a^{-1}a\,x_{\perp} 
  \nonumber \\
  = -a^{-1}x_{\parallel}a - a^{-1}\,x_{\perp}a
  = -a^{-1}(x_{\parallel} +\,x_{\perp})a
  = -a^{-1} x a .
  \label{eq:refl}
\end{gather}



\end{frame}

%=========================================================================
\begin{frame}[fragile]{Rotor}

The combination of two reflections at two lines (hyperplanes) with normals $a,b$
\be 
  x'' = -b^{-1} x' b = b^{-1} a^{-1} x ab = (ab)^{-1} x ab = R^{-1} x R ,
\ee 
gives a rotation. The rotation angle is $\alpha = 2 \theta_{a,b}$ and the \textit{rotor}
\be
  R = e^{\theta_{a,b} e_{12}} = e^{\frac{1}{2}\alpha e_{12}},
\ee 
where the lengths $|a||b|$ of $ab$ cancel against 
$|a|^{-1}|b|^{-1}$ in $(ab)^{-1}$.
The rotor $R$ gives the \textit{spinor} form of rotations, fully replacing rotation matrices, and introducing the same elegance to \textit{real} rotations in $\R^2$, like in the complex plane. 


\end{frame}

%=========================================================================
\begin{frame}[fragile]{Reflections and Rotations}
In 2D, the product of three reflections, i.e. of a rotation and a reflection, leads to another reflection. 

\pause
In 2D the product of an \textit{odd} number of reflections always results in a \textit{reflection}. 

\pause
That the product of an \textit{even} number of reflections leads to a \textit{rotation} is true in general dimensions. 

\pause
These transformations are in Clifford algebra simply described by the products of the vectors normal to the lines (hyperplanes) of reflection and called versors. 

\end{frame}

\section{Geometric algebra in 3D}

%=========================================================================
\begin{frame}[fragile]{Geometric algebra of 3D Euclidean space}

%\subsection{Geometric algebra of 3D Euclidean space}

\alert{The Clifford algebra $Cl(\R^3) = Cl(3,0)$ is  probably the most thoroughly studied and applied GA}.

 In physics it is also known as \textit{Pauli algebra}, since Pauli's spin matrices provide a $2\times 2$ matrix representation. This shows how GA unifies \textit{classical} and \textit{quantum} mechanics and electromagnetism. 
\pause

Given an orthonormal vector basis $\{e_1, e_2, e_3\}$ of $\R^3$, the eight-dimensional ($2^3=8$) Clifford algebra $Cl(\R^3) = Cl(3,0)$ has a basis of one scalar, three vectors, three bivectors and one trivector
\be 
  \{1, e_1, e_2, e_3, e_{23}, e_{31}, e_{12}, e_{123}\},
  \label{eq:Cl3basis}
\ee 
where as before $e_{23}=e_2e_3, e_{123} = e_1e_2e_3$, etc. All basis bivectors square to $-1$, and the product of two basis bivectors gives the third
\be 
  e_{23} e_{31} = e_{21} = - e_{12}, \,\,\, \text{etc.}
\ee 
%




\end{frame}

%=========================================================================
\begin{frame}[fragile]{Quaternions}

\alert{
Therefore the even subalgebra $Cl^+(3,0)$ with basis\footnote{The minus signs are only chosen, to make the product of two bivectors identical to the third, and not minus the third.} $\{1, -e_{23}, -e_{31}, -e_{12}\}$ is indeed found to be isomorphic to quaternions $\{1, \mathbf{i}, \mathbf{j}, \mathbf{k}\}$. 
}
\pause

This isomorphism is not incidental. As we have learned already for $Cl(2,0)$, also in $Cl(3,0)$, the even subalgebra is the algebra of rotors (rotation operators) or spinors, and describes rotations in the same efficient way as do quaternions. 

\pause
We therefore gain a \textit{real geometric} interpretation of quaternions, as the oriented bi-vector side faces of a unit cube, with edge vectors $\{e_1, e_2, e_3\}$. 


\end{frame}

%=========================================================================
\begin{frame}[fragile]{Reflections and rotations in 3D}
In $Cl(3,0)$ a reflection at a plane (=hyperplane) is specified by the plane's normal vector $a\in\R^3$
\be 
  x' = -a^{-1} x a,
\ee 
the proof is identical to the one in (\ref{eq:refl}) for $Cl(2,0)$. 

The combination of two such reflections leads to a rotation by $\alpha = 2 \theta_{a,b}$
%%%%%%%%%%%%%%%%%%%%%%%%%%%%
%%% Reformatted equation %%%
%%%%%%%%%%%%%%%%%%%%%%%%%%%%
\begin{align} 
  x'' &= R^{-1} x R, 
  \\ 
  R &= ab = |a||b| e^{\theta_{a,b}\mathbf{i}_{a,b}}
  = |a||b| e^{\frac{1}{2}\alpha \mathbf{i}_{a,b}},
  \nonumber
\end{align} 
%%%%%%%%%%%%%%%%%%%%%%%%%%%%
%%% Reformatted equation %%%
%%%%%%%%%%%%%%%%%%%%%%%%%%%%
where $\mathbf{i}_{a,b}={a \wedge b}/({|a\wedge b|})$ specifies the oriented unit bivector of the plane spanned by $a,b \in \R^3$. 

\end{frame}

%=========================================================================
\begin{frame}[fragile]{The unit trivector $i=e_{123}$}

\alert{The unit trivector $i=e_{123}$ also squares to $-1$}

\begin{align}
  i^2 &= e_1e_2e_3e_1e_2e_3 
  = - e_1e_2e_1e_3e_2e_3 \nonumber \\
  &= e_1e_2e_1e_2 e_3e_3 = (e_1e_2)^2 (e_3)^2 = -1,
\end{align}
where we only used that the permutation of two orthogonal vectors in the geometric product produces a minus sign. 
Hence $i^{-1} = - i$. 
\pause
We further find, that $i$ commutes with every vector, e.g.
%%%%%%%%%%%%%%%%%%%%%%%%%%%%
%%% Reformatted equation %%%
%%%%%%%%%%%%%%%%%%%%%%%%%%%%
\begin{align} 
  e_1 i &= e_1 e_1e_2e_3 = e_{23}, 
  \\
  i e_1 &= e_1e_2e_3 e_1 = - e_1e_2e_1e_3 = e_1e_1 e_2e_3 = e_{23}, 
  \nonumber 
\end{align} 
%%%%%%%%%%%%%%%%%%%%%%%%%%%%
%%% Reformatted equation %%%
%%%%%%%%%%%%%%%%%%%%%%%%%%%%
\pause
and the like for $e_2 i = i e_2$, $e_3 i = i e_3$. 

\end{frame}

%=========================================================================
\begin{frame}[fragile]{$i$ changes bivectors into orthogonal vectors}
If $i$ commutes with every vector, it also commutes with every bivector $a\wedge b = \frac{1}{2}(ab-ba)$.

 Hence $i$ commutes with every element of $Cl(3,0)$.
 \pause
%, a property which is called \textit{central} in mathematics. The central subalgebra spanned by $\{1,i\}$ is isomorphic to complex numbers $\C$. 

\alert{$i$ changes bivectors into orthogonal vectors}
\be 
  e_{23} i = e_2e_3e_1e_2e_3 = e_1 e_{23}^2 = -e_1\,, \,\,\, \text{etc.}
\ee 

Writing the basis in the simple product form (\ref{eq:Cl3basis}), fully preserves the \textit{geometric interpretation} in terms of \alert{scalars, vectors, bivectors and trivectors}, and allows to \textit{reduce} all products to elementary geometric products of basis vectors.



\end{frame}


%=========================================================================
\begin{frame}[fragile]{Multiplication table and subalgebras of $Cl(3,0)$}
\begin{table}
\caption{Multiplication table of Clifford algebra $Cl(3,0)$ of Euclidean 3D space $\R^3$.}
\label{tb:Cl3mtable}
\begin{center}
\begin{tabular}{c|cccccccc}
       & $1$ & $e_1$ & $e_2$ & $e_3$ & $e_{23}$ & $e_{31}$ & $e_{12}$ & $e_{123}$\\
 \hline
 $1$     & $1$ & $e_1$ & $e_2$ & $e_3$ & $e_{23}$ & $e_{31}$ & $e_{12}$ & $e_{123}$\\
 $e_1$ & $e_1$ & 1 & $e_{12}$ & $-e_{31}$ & $e_{123}$ & $-e_3$ & $e_{2}$ & $e_{23}$\\
 $e_2$ & $e_2$ & $-e_{12}$ & 1 & $e_{23}$ & $e_3$ & $e_{123}$ & $-e_1$ & $e_{31}$ \\
 $e_3$ & $e_3$ & $e_{31}$ & $-e_{23}$ & $1$ & $-e_2$ & $e_1$ & $e_{123}$ & $e_{12}$\\
 $e_{23}$ & $e_{23}$ & $e_{123}$ & $-e_3$ & $e_2$ & $-1$ & $-e_{12}$ & $e_{31}$ & $-e_1$\\
 $e_{31}$ & $e_{31}$ & $e_3$ & $e_{123}$ & $-e_1$ & $e_{12}$ & $-1$ & $-e_{23}$ & $-e_2$\\
 $e_{12}$ & $e_{12}$ & $-e_{2}$ & $e_1$ & $e_{123}$ & $-e_{31}$ & $e_{23}$ & $-1$ & $-e_3$\\
 $e_{123}$ & $e_{123}$ & $e_{23}$ & $e_{31}$ & $e_{12}$ & $-e_1$ & $-e_2$ & $-e_3$ & $-1$
\end{tabular}
\end{center}
\end{table}


%\subsubsection{Multiplication table and subalgebras of $Cl(3,0)$ }


\end{frame}

%=========================================================================
\begin{frame}[fragile]{}

For the full multiplication table of $Cl(3,0)$ we still need the geometric products of vectors and bivectors. By changing labels in Table \ref{tb:Cl2mtable} ($1 \leftrightarrow 3$ or $2 \leftrightarrow 3$), we get that
%%%%%%%%%%%%%%%%%%%%%%%%%%%%
%%% Reformatted equation %%%
%%%%%%%%%%%%%%%%%%%%%%%%%%%%
\begin{align} 
  e_2 e_{23} &= - e_{23}e_2 = e_3, 
  \nonumber \\
  e_3 e_{23} &= - e_{23}e_3 = -e_2
  \\
  e_1 e_{31} &= - e_{31}e_1 = -e_3, 
  \nonumber \\
  e_3 e_{31} &= - e_{31}e_3 = e_1,
\end{align} 
%%%%%%%%%%%%%%%%%%%%%%%%%%%%
%%% Reformatted equation %%%
%%%%%%%%%%%%%%%%%%%%%%%%%%%%
which shows that in general a vector and a bivector, which includes the vector, anti-commute. 


\end{frame}

%=========================================================================
\begin{frame}[fragile]{}

The products of a vector with its orthogonal bivector always gives the trivector $i$
\begin{gather} 
  e_1 e_{23} = e_{23}e_1 = i, \quad
  e_2 e_{31} = e_{31}e_2 = i,
  \nonumber \\
  e_3 e_{12} = e_{12}e_3 = i,
\end{gather} 
which also shows that in general vectors and orthogonal bivectors necessarily commute. 

Commutation relationships therefore clearly depend on both \textit{orthogonality} properties and on the \textit{grades} of the factors, which can frequently be exploited for computations even without the explicit use of coordinates. 

\end{frame}

%=========================================================================
\begin{frame}[fragile]{The grade structure of $Cl(3,0)$ and duality}
%\subsubsection{The grade structure of $Cl(3,0)$ and duality}


A general multivector in $Cl(3,0)$, can be represented as
\begin{gather}
  M = m_0 + m_1 e_1 + m_2 e_2 + m_3 e_3 + m_{23} e_{23}
  + m_{31} e_{31}+ m_{12} e_{12}   \nonumber \\
  + m_{123} e_{123}, \quad
  m_0, \ldots , m_{123} \in \R. 
\end{gather}
\pause
We have 
a scalar part $\langle M \rangle_0$ of grade $0$, 
a vector part $\langle M \rangle_1$ of grade $1$, 
a bivector part $\langle M \rangle_2$ of grade $2$,
and a trivector part $\langle M \rangle_3$ of grade $3$
\begin{align}
  M &= \langle M \rangle_0 + \langle M \rangle_1 + \langle M \rangle_2 + \langle M \rangle_3,
  \\
   \langle M \rangle_0 &= m_0, \quad
  \langle M \rangle_1 = m_1 e_1 + m_2 e_2 + m_3 e_3, \nonumber \\
  \langle M \rangle_2 &= m_{23} e_{23} + m_{31} e_{31}+ m_{12} e_{12},   
  \quad
  \langle M \rangle_3 = m_{123} e_{123}.
  \nonumber 
\end{align}


\end{frame}

%=========================================================================
\begin{frame}[fragile]{}
The set of all grade $k$ elements, $0 \leq k \leq 3$, is denoted $Cl^k(3,0)$.

The multiplication table of $Cl(3,0)$, Table \ref{tb:Cl3mtable}, reveals that multiplication with $i$ (or $i^{-1}=-i$) consistently changes an element of grade $k$, $0 \leq k \leq 3$, into an element of grade $3-k$, i.e. scalars to trivectors (also called pseudoscalars) and vectors to bivectors, and vice versa. 



\end{frame}


\section{The Telegrapher's equations: an example of $Cl(1,1)$}


%=========================================================================
\begin{frame}[fragile]{Telegrapher equations}

%\subsubsection{Telegrapher equations}
%
Let us consider the voltage $V$ and the current $I$ along a transmission line in the $x$ direction.
The  telegrapher's equations, for a lossless line,
%\verb!\href{https://en.wikipedia.org/wiki/Telegrapher%27s_equations}{\textcolor{blue}{telegrapher's equations }} !
are
\bea
\partial_ x V & = & - L \, \partial_t I \nonumber \\
\partial_ x I & = & - C \, \partial_t V 
\label{telegrapher}
\eea
where $L$ and $C$ are the inductances and capacitances per unit length.
\pause
It is convenient to introduce the velocity $v$ and the impedance $\eta$ defined as
%
\bea 
v & = & \frac{1}{\sqrt{LC}} \nonumber \\
\eta & = &  \sqrt{\frac{L}{C}}
\eea
%
\pause
or, equivalently
%
\bea 
L & = & \frac{\eta}{v} \nonumber \\
C & = & \frac{1}{\eta v} \,.
\label{LCetav}
\eea
%




\end{frame}

%=========================================================================
\begin{frame}[fragile]{}

By substituting (\ref{LCetav}) into (\ref{telegrapher}) we get
%
\bea
\partial_ x V & = & -\frac{1}{v} \, \partial_t  \left( \eta \,  I \right) \nonumber \\
\partial_ x \left( \eta \,  I \right) & = & -\frac{1}{v} \, \partial_t V 
\label{telegrapher}
\eea
%

\pause
It is also convenient to introduce the two variables $x_0,x_1$ as 
%
\bea
x_0 & = & v \, t \nonumber \\
x_1 & = & x \label{x0x1}
\eea
%
\pause
so that
%
\bea
\frac{1}{v}  \partial_t & = & \frac{\partial}{\partial x_0} = \partial_0 \nonumber \\
\partial_x  & = & \frac{\partial}{\partial x_1} = \partial_1 \,.
\label{tele01}
\eea
%

\end{frame}

%=========================================================================
\begin{frame}[fragile]{}

The  equations in (\ref{telegrapher}) can be written in matrix form as
%
\be
\begin{pmatrix}
\partial_0\, & \,\partial_1 \cr
\partial_1 \,& \,\partial_0
\end{pmatrix}
%
\begin{pmatrix} 
V \cr
\eta I  
\end{pmatrix}
= \begin{pmatrix} 
0 \cr
0 
\end{pmatrix}
\label{te_matrix_form}
\ee
%
\pause
or, equivalently, by changing the sign in the second row, as
%
\be
\begin{pmatrix}
\partial_0\, & \,\partial_1 \cr
-\partial_1 \,& \,-\partial_0
\end{pmatrix}
%
\begin{pmatrix} 
V \cr
\eta I  
\end{pmatrix}
= \begin{pmatrix} 
0 \cr
0 
\end{pmatrix} \,.
\label{minus_second_row}
\ee


\end{frame}

%=========================================================================
\begin{frame}[fragile]{}
By making use of Pauli matrices we can therefore write the telegrapher equations in (\ref{minus_second_row}) as
%
\be
\left( \sigma_3 \partial_0 + i\, \sigma_2 \partial_1 \right)
\psi
= 0
\label{sigma3_sigma2}
\ee
%
where we have introduced the quantity $\psi$ defined as
\be
\psi = \begin{pmatrix} 
V \cr
\eta I  
\end{pmatrix} \, .
\ee
%
\pause
It is convenient, when the time variable is considered as in this case, to denote the elements of the Clifford basis starting from zero instead of one.
We can now identify the elements of the basis $\{e_0, e_1\}$ as
%
\bea
e_0 & = & \sigma_3  \nonumber \\
e_1 & = & i \, \sigma_2
\label{basesigma2sigma3}
\eea



\end{frame}

%=========================================================================
\begin{frame}[fragile]{}
by noting that
%
\bea
e_0^2=\sigma_3 \sigma_3  & = &  \, \sigma_0 \nonumber \\
e_1^2= - \sigma_2 \sigma_2  & = & -  \, \sigma_0 \nonumber \\
e_0 e_1 = \sigma_3 \, i \, \sigma_2 & = & - i \, \sigma_2 \,  \sigma_3  = - e_1 e_0\,.
\eea
we have therefore realized an example of $Cl(1,1)$.
\pause

Therefore, the Clifford algebra $Cl(1,1)$  with the identification of the basis as in (\ref{basesigma2sigma3}), is well suited to describe the telegrapher's equation.

The telegrapher's equation, in a geometric algebra form, is:
\be
\left( e_0 \partial_0 + e_1 \partial_1 \right) \psi = 0
\ee

This is in a form similar to the Dirac equation.

\end{frame}



%=========================================================================
\begin{frame}[fragile]{}

Naturally, since $e_0$ squares to $1\sigma_0$ and $e_1$ squares to $-1\sigma_0$ and they anticommute we also have
%
\be
\left( e_0 \partial_0 + e_1 \partial_1 \right) \left( e_0 \partial_0 + e_1 \partial_1 \right)  = \left(\partial_0^2 - \partial_1^2 \right) \sigma_0 \label{proctowaveeq}
\ee

\pause
which provides the operator of the wave equation
\be
 \left( \partial_0^2 - \partial_1^2 \right) \sigma_0 \psi = 0 \, . \label{waveeqtl}
\ee

\alert{With GA we have found the square root of the operator of the wave equation! This is not possible in conventional vector algebra.}

\end{frame}

%=========================================================================
\begin{frame}[fragile]{Conventional procedure compared to GA}
%\subsubsection{Conventional procedure compared to GA}
The conventional procedure to find the wave equation corresponding to (\ref{waveeqtl}) is the following. One starts
from

\bea
\partial_0 V + \partial_1 (\eta I)& = &0  \label{feqVI}\\
-\partial_1 V - \partial_0 (\eta I)& = & 0 \label{seqVI}
\eea
\pause
perform a derivative of the first equation (\ref{feqVI}) w.r.t. $\partial_0$. 
\pause
Then perform a derivative of (\ref{seqVI}) w.r.t. $\partial_1$ and then substitute $\partial_0\partial_1 (\eta I)$ so as to obtain the equation in $V$. 
\pause
Then repeat again the procedure for obtaining the other second order equation in $I$.
\pause

\alert{With (\ref{proctowaveeq}) in just one passage we have obtained (\ref{waveeqtl})!}



\end{frame}

%=========================================================================
\begin{frame}[fragile]{Equivalent equations}

\subsubsection{Equivalent equations}
As an alternative we could have considered (\ref{te_matrix_form}) and write it in terms of Pauli matrices as:
%
\be
\left( \sigma_0 \partial_0 +  \sigma_1 \partial_1 \right)
\psi
= 0 \,.
\label{sigma0_sigma1}
\ee
%
\pause
\alert{Note, however, that the $\sigma_0$ matrix is not anti--commutative and therefore cannot be used to create the geometric algebra basis.}
\pause

Nonetheless, it is still feasible  to obtain the operator of the wave equation in the following way:
\be
\left( \sigma_0 \partial_0 - \sigma_1 \partial_1 \right) \left( \sigma_0 \partial_0 + \sigma_1 \partial_1 \right)  = \partial_0^2 - \partial_1^2 \, .
\ee



\end{frame}

%=========================================================================
\begin{frame}[fragile]{Systematic way to generate Dirac--like equations}

But there is one more systematic way to generate Dirac--like equation when $\sigma_0$ is present.

\pause

The expression appearing in (\ref{sigma0_sigma1}) can be transformed without needing to change the sign at one equation, as we did before.

In fact, we can multiply  (\ref{sigma0_sigma1}) by one of the sigma not appearing in the equation (therefore either $\sigma_2$ or $\sigma_3$) and obtain a different equation composed exclusively by anti--commuting matrices.

\end{frame}

%=========================================================================
\begin{frame}[fragile]{An example}
As an example, by pre mutltiplying (\ref{sigma0_sigma1}) with $\sigma_3$ one obtains
%
\be
\sigma_3 \, \left( \sigma_0 \partial_0 +  \sigma_1 \partial_1 \right)
\psi
=  \left( \sigma_3 \partial_0 +  i \sigma_2 \partial_1 \right)
\psi \,.
%\label{sigma3_sigma2}
\ee
%
i.e. (\ref{sigma3_sigma2}). 

\pause

Pre and post multiplication of (\ref{sigma0_sigma1}) with $\sigma_2$ leads to other possible equations as
%
\bea
\sigma_2 \, \left( \sigma_0 \partial_0 +  \sigma_1 \partial_1 \right) & = & \sigma_2 \partial_0 - i  \sigma_3 \partial_1 \nonumber \\
\left( \sigma_0 \partial_0 +  \sigma_1 \partial_1 \right)  \, \sigma_2 & = & \sigma_2 \partial_0 + i  \sigma_3 \partial_1 \, .
\eea
%


\end{frame}

%=========================================================================
\begin{frame}[fragile]{}

It is left as an exercise to perform the following computation:
%
\be
\frac{1}{2}
\left( \sigma_0  +  i \, \sigma_2 \right)
\left( \sigma_2 \partial_0 +  i \, \sigma_3 \partial_1 \right)
\left( \sigma_1 +  \sigma_3 \right) 
= 
i \, \sigma_1 \partial_0 - \sigma_2 \partial_1
\ee
%
and to recognize that the result is an anti--diagonal matrix, thus leading to two separated problems.

\pause

Similarly, if an expression is formed only by employing $\sigma_0$ and $\sigma_3$ will lead again to two separated problems.

\pause
\alert{The technique to obtain such diagonalization is called Weyl decomposition.}



\end{frame}



%=========================================================================
\begin{frame}[fragile]{The Weyl decomposition}

%\subsection{The Weyl decomposition}
Let us introduce the quantities $a,b$ defined as
%
\be
\begin{pmatrix} 
a+b \cr
a-b 
\end{pmatrix} =
\psi = \begin{pmatrix} 
V \cr
\eta I  
\end{pmatrix} \, .
\label{a_b_intro}
\ee
%
\pause
By using (\ref{a_b_intro}) in (\ref{minus_second_row}) we obtain the following two equations
%
\bea
 \partial_0 \left( a+b \right) + \partial_1 \left( a-b \right) &=& 0 \label{a_b_first} \\
  - \partial_1 \left( a+b \right) - \partial_0 \left( a-b \right) &=& 0 \label{a_b_second} \,.
\eea
%


\end{frame}

%=========================================================================
\begin{frame}[fragile]{}

By summing and subtracting (\ref{a_b_first}) and (\ref{a_b_second})
we obtain two independent expressions as
%
\bea
\partial_0  a + \partial_1  a &=& 0 \label{a_second}
\nonumber    \\
 \partial_0 b - \partial_1b  &=& 0 \label{b_first}  
\eea
%
\pause
or, in matrix form,
%
\be
\begin{pmatrix}
\partial_0   + \partial_1 &0 \cr
0& \,\partial_0  - \partial_1
\end{pmatrix}
%
\begin{pmatrix} 
a \cr
b 
\end{pmatrix}
= 0 \,.
\label{ab_matrix}
\ee
%
\pause
The $a$ and $b$ correspond to progressive and regressive waves, respectively.


\end{frame}



%=========================================================================
\begin{frame}[fragile]{}
\alert{What we have obtained here is of \emph{considerable importance}. }
\pause

If we consider our original problem (\ref{minus_second_row}) we have a \emph{system of two coupled equations of the first order}. 

\pause
By contrast, when considering (\ref{ab_matrix}) we have two \emph{independent} first order equations.

\pause
Therefore, the traveling waves (both progressive and regressive) are the natural basis for having uncoupled equations!





\end{frame}

%=========================================================================
\begin{frame}[fragile]{Time--harmonic solution}

%\subsection{Time--harmonic solution}
We have seen that propagation along the transmission line can be described by the equations (\ref{ab_matrix}) here repeated for convenience:
%
\be
\begin{pmatrix}
\partial_0   + \partial_1 &0 \cr
0& \,\partial_0  - \partial_1
\end{pmatrix}
%
\begin{pmatrix} 
a \cr
b 
\end{pmatrix}
= 0 \,.
%\label{ab_matrix}
\ee
%
\pause
\alert{The $a$ and $b$ correspond to progressive and regressive waves, respectively.}
In order to find a solution it is advantageous to apply separation of variables and to consider a time--harmonic solution.
%


\end{frame}

%=========================================================================
\begin{frame}[fragile]{Separation of variables}

%\subsubsection{Separation of variables}
The solution for the propagating wave $a$ can be written in terms of two different functions $a_0,a_1$ as
%
\be
a(x_0,x_1) = a_0(x_0) \, a_1(x_1) \, .
\ee
%
\pause
In addition we can assume a time--harmonic behavior for the $a_0$ part. In particular, it is typically chosen the following expansion
%
\be
a_0(x_0) = A_0 e^{j\omega t} = A_0 e^{j k x_0}
\ee
%
with $k=\omega/v$ and $x_0=vt$ defined in (\ref{x0x1}).
\pause
It is noted that the derivative w.r.t. $x_0$ of $a$ gives
\be
\partial_0 a(x_0,x_1)= j k \, a_0(x_0) a_1(x_1)\, ,
\ee
\pause

and therefore the equation becomes
%
\be
 j k \, a_0(x_0) a_1(x_1) + \partial_1 a_0(x_0) a_1(x_1) =0 \, .
\ee


\end{frame}

%=========================================================================
\begin{frame}[fragile]{}

Since $a_0(x_0)$ is present in all members can be factored out, obtaining the following equation
%
\be
 \partial_1  a_1(x_1) = -  j k \,  a_1(x_1) 
\ee
%
with  solution
%
\be
a_1(x_1) = A_1 e^{-jkx_1} \,.
\ee
%
\pause

Therefore the waves $a$ can be written as
%
\be
a(x_0,x_1) = A_0 e^{j k x_0}  
A_1 e^{-jkx_1} = A e^{jk(x_0 -x_1)}
\ee
% 
with $A=A_0\, A_1$. This solution represent a progressive wave. 
\pause
By a similar procedure the solution for $b$ can be obtained as
%
\be
b(x_0,x_1) = B e^{jk(x_0 + x_1)} \, .
\ee
%


\end{frame}

%=========================================================================
\begin{frame}[fragile]{Interface between two media}

%\subsection{Interface between two media}
Let us consider the case when we have two different transmission lines, the one on the left with an impinging wave denoted by $a_0$ and reference impedance $\eta_0$. 
\pause

The medium on the right has an impedance $\eta_1$. 

\pause
In terms of waves the discontinuity gives rise to a progressive wave in medium 1 (on the right) $a_1$ and to a reflected wave $b_0$ in medium 0 on the left.
\pause

 By placing the discontinuity at $x_1=0$ we have
%
\bea
V_0 & = & a_0+ b_0 \nonumber \\
\eta_0 \, I_0 & = & a_0 - b_0
\eea
%
and
%
\bea
V_1 & = & a_1 \nonumber \\
\eta_1 \, I_1 & = & a_1 \,.
\eea
%

\end{frame}



%=========================================================================
\begin{frame}[fragile]{}
The continuity conditions tell us that
%
\bea
V_1&=&V_0 \nonumber \\
I_1 & = & I_0 \, .
\eea
%
After solving for $a_1, b_0$ we get 
%
\bea
b_0 & = & \Gamma \, a_0 \nonumber \\
a_1 & = & \tau \, a_0
\eea
%



\end{frame}

%=========================================================================
\begin{frame}[fragile]{}
with
%
\bea
\Gamma & = & \frac{{\eta}_{1}-{\eta}_{0}}{{\eta}_{1}+{\eta}_{0}} \nonumber \\
\tau & = & \frac{2\,{\eta}_{1}}{{\eta}_{1}+{\eta}_{0}}
\eea
%
as shown in the code reported next.


\end{frame}

%=========================================================================
\begin{frame}[shrink=50]{}
\newpage
\small
\lstinputlisting{bc1.wxm}
\normalsize
%\newpage



\end{frame}

%%=========================================================================
%\begin{frame}[fragile]{}
%
%
%
%\end{frame}
%
%%=========================================================================
%\begin{frame}[fragile]{}
%
%
%
%\end{frame}

%%=========================================================================
%\begin{frame}[fragile]{}
%
%
%
%\end{frame}
%
%%=========================================================================
%\begin{frame}[fragile]{}
%
%
%
%\end{frame}
%
%%=========================================================================
%\begin{frame}[fragile]{}
%
%
%
%\end{frame}
%
%%=========================================================================
%\begin{frame}[fragile]{}
%
%
%
%\end{frame}
%

%
%%=========================================================================
%\begin{frame}[fragile]{}
%
%
%
%\end{frame}
%
%%=========================================================================
%\begin{frame}[fragile]{}
%
%
%
%\end{frame}
%
%%=========================================================================
%\begin{frame}[fragile]{}
%
%
%
%\end{frame}
%
%%=========================================================================
%\begin{frame}[fragile]{}
%
%
%
%\end{frame}
%
%%=========================================================================
%\begin{frame}[fragile]{}
%
%
%
%\end{frame}
%



\end{document}
